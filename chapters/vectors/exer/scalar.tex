\begin{enumerate}[label=\thesection.\arabic*,ref=\thesection.\theenumi]
\item Let $\vec{a}$ and $\vec{b}$ be two unit vectors and $\theta$ is the angle between them.Then $\vec{a}+\vec{b}$ is a unit vector if
\begin{enumerate}
\item $\theta=\frac{\pi}{4}$
\item $\theta=\frac{\pi}{3}$
\item $\theta=\frac{\pi}{2}$
\item $\theta=\frac{2\pi}{3}$
\end{enumerate}
\item The value of $\hat{i}.(\hat{j}\times\hat{k})+\hat{j}.(\hat{i}\times\hat{k})+\hat{k}.(\hat{i}\times\hat{j})$ is
\begin{enumerate}
\item 0
\item -1
\item 1
\item 3
\end{enumerate}
\item If $\theta$ is the angle between any two vectors $\vec{a}$ and $\vec{b}$,then $|\vec{a}.\vec{b}|=|\vec{a}\times\vec{b}|$ when $\theta$ is equal to
\begin{enumerate}
\item 0
\item $\frac{\pi}{4}$
\item $\frac{\pi}{2}$
\item $\pi$
\end{enumerate}
\item A vector $\vec{r}$ has a magnitude 14 and direction ratios 2,3,-6. Find the direction cosines and components of $\vec{r}$, given that $\vec{r}$ makes an acute angle with x-axis.
\item Find the angle between the vectors $2\hat{i}-\hat{j}+\hat{k}$ $\text{and}$ $3\hat{i}+4\hat{j}-\hat{k}$.
\item If $\vec{a},\vec{b},\vec{c}$ are the three vectors such that $\vec{a}+\vec{b}+\vec{c}=0$ $\text{ and }$ $|\vec{a}|=2$, $|\vec{b}|$=3, $|\vec{c}|$=5, the value of $\vec{a}.\vec{b}+\vec{b}.\vec{c}+\vec{c}.\vec{a}$ is
	\begin{enumerate}
\item 0
\item 1	
\item -19
\item 38
\end{enumerate}
\item If $\vec{a}$, $\vec{b}$, $\vec{c}$ are unit vectors such that $\vec{a}$+$\vec{b}$+$\vec{c}$=0, then the value of $\vec{a}.\vec{b}+\vec{b}.\vec{c}+\vec{c}.\vec{a}$ is
	\begin{enumerate}
\item 1
\item 3
\item $\frac{-3}{2}$
\item None of these
\end{enumerate}
\item The angles between two vectors $\vec{a}$ $\text{and}$ $\vec{b}$ with magnitude $\sqrt{3}$ $\text{ and }$ 4, respectively, and $\vec{a}$, $\vec{b}$= $2\sqrt{3}$ is
	\begin{enumerate}
\item $\frac{\pi}{6}$
\item $\frac{\pi}{3}$
\item $\frac{\pi}{2}$ 
\item $\frac{5\pi}{2}$
\end{enumerate}

\item The vector $\vec{a}+\vec{b}$ bisects the angle between the non-collinear vectors $\vec{a}$ $\text{ and }$ $\vec{b}$ if \rule{1cm}{0.15mm}.
\item The vectors $\vec{a}=3\hat{i}-2\hat{j}+2\hat{k}$ $\text{ and }$ $\vec{b}=\hat{i}-2\hat{k}$ are the adjancent sides of a parallelogram. The acute angle between its diagonals is \rule{1cm}{0.15mm}.
\item If $\vec{a}$ is  any non-zero vector, then $(\vec{a}.\hat{i})\hat{i}$+$(\vec{a}.\hat{j})\hat{j}$+$(\vec{a}.\hat{k})$ $\hat{k}$ equals \rule{1cm}{0.15mm}.
\item If $\vec{a}$ $\text{ and }$ $\vec{b}$ are adjacent sides of a rhombus, then $\vec{a}.\vec{b}$.=0.
\item Find the angle between the lines $$\overrightarrow{r}=3\hat{i}-2\hat{j}+6\hat{k}+\lambda(2\hat{i}+\hat{j}+2\hat{k})\text{ and } \overrightarrow{r}=(2\hat{j}-5\hat{k})+\mu(6\hat{i}+3\hat{j}+2\hat{k})$$
\item Find the angle between the lines whose direction cosines are given by the equations $l+m+n=0$, $l^2+m^2-n^2=0$.
\item If a variable line in two adjacent positions has directions cosines $l, m, n$ and $l+\delta l, m+\delta m, n+\delta n$, show that the small angle $\delta\theta$ between the two positions is given by $$\delta\theta^2=\delta l^2+\delta m^2+\delta n^2$$ 
\item The sine of the angle between the straight line $\dfrac{x-2}{3}=\dfrac{y-3}{4}=\dfrac{z-4}{5}$ and the plane $2x-2y+z=5$ is
\begin{enumerate}
	\item $\dfrac{10}{6\sqrt{5}}$ 
	\item $\dfrac{4}{5\sqrt{2}}$
	\item $\dfrac{2\sqrt{3}}{5}$
	\item $\dfrac{\sqrt{2}}{10}$
\end{enumerate}
\item The plane $2x-3y+6z-11=0$ makes an angle $\sin^{-1}(\alpha)$ with x-axis. The value of $\alpha$ is equal to 
\begin{enumerate}
	\item  $\dfrac{\sqrt{3}}{2}$
	\item  $\dfrac{\sqrt{2}}{3}$
	\item  $\dfrac{2}{7}$
	\item  $\dfrac{3}{7}$
\end{enumerate}
\item The angle between the line $\overrightarrow{r}=(5\hat{i}-\hat{j}-4\hat{k})+\lambda(2\hat{i}-\hat{j}+\hat{k})$ and the plane $\overrightarrow{r} \cdot (3\hat{i}-4\hat{j}-\hat{k})+5=0$ is $\sin^{-1}\brak{\dfrac{5}{2\sqrt{91}}}$.
\item The angle between the planes $\overrightarrow{r} \cdot (2\hat{i}-3\hat{j}+\hat{k})=1$ and $\overrightarrow{r} \cdot (\hat{i}-\hat{j})=4$ is $\cos^{-1} \brak{\dfrac{-5}{\sqrt{58}}}$.
\item Let $\vec{a}$ and $\vec{b}$ be two unit vectors and $\theta$ is the angle between them.Then $\vec{a}+\vec{b}$ is a unit vector if
\begin{enumerate}
\item $\theta=\frac{\pi}{4}$
\item $\theta=\frac{\pi}{3}$
\item $\theta=\frac{\pi}{2}$
\item $\theta=\frac{2\pi}{3}$
\end{enumerate}
\item The value of $\hat{i}.(\hat{j}\times\hat{k})+\hat{j}.(\hat{i}\times\hat{k})+\hat{k}.(\hat{i}\times\hat{j})$ is
\begin{enumerate}
\item 0
\item -1
\item 1
\item 3
\end{enumerate}
\item If $\theta$ is the angle between any two vectors $\vec{a}$ and $\vec{b}$,then $|\vec{a}.\vec{b}|=|\vec{a}\times\vec{b}|$ when $\theta$ is equal to
\begin{enumerate}
\item 0
\item $\frac{\pi}{4}$
\item $\frac{\pi}{2}$
\item $\pi$
\end{enumerate}

\end{enumerate}
