\begin{enumerate}[label=\thesubsection.\arabic*,ref=\thesubsection.\theenumi]
\item Find the angle between two vectors $\overrightarrow{a}$ and $\overrightarrow {b} $ with magnitudes $\sqrt{3}$ and 2 respectively having $\overrightarrow {a}\cdot\overrightarrow {b}=\sqrt{6}$.
		\label{prob:12/10/3/1/inner}
	\\
	\solution
		From the given information,
\begin{align}
\norm{\vec{a}}=\sqrt{3},
\norm{\vec{b}}= 2,
{\vec{a}^{\top}}{\vec{b}}=\sqrt{6}  
\\
\implies \cos\theta=\frac{{\vec{a}^{\top}}{\vec{b}}}{\norm{\vec{a}}\norm{\vec{b}}}
=\frac{1}{\sqrt{2}}\\
	\text{or, }\theta={45}\degree
\end{align}

\item Find the angle between the the vectors $\hat{i}-2\hat{j}+3\hat{k}$ and $3\hat{i}-2\hat{j}+\hat{k}$.
	\\
	\solution
		Let 
\begin{align}
	\vec{a} = \myvec{1\\-2\\3} , \vec{b} = \myvec{3\\ -2 \\ 1},
\end{align}
		From problem \ref{prob:12/10/3/1/inner},
\begin{align}
\cos\theta=\frac{\vec{a}^{\top}\vec{b}}{\norm{\vec{a}}\norm{\vec{b}}}
	= \frac{10}{\sqrt{14}\times \sqrt{14}}= \frac{5}{7}
\end{align}

\item Find $\abs{\overrightarrow {a}}$ and $\abs{\overrightarrow {b}}$, if ($\overrightarrow {a}+\overrightarrow {b})\cdot (\overrightarrow {a}-\overrightarrow {b})=8$ and $\abs{\overrightarrow {a}}=8\abs{\overrightarrow {b}}$.
	\\
	\solution
		\begin{align}
\because \brak{\vec{a}+\vec{b}}^\top\brak{\vec{a}-\vec{b}}=8,
\norm{\vec{a}} &= 8\norm{\vec{b}},\\
\norm{\vec{a}}^2-\norm{\vec{b}}^2&=8\\
\implies\norm{8\vec{b}}^2-\norm{\vec{b}}^2&=8\\
\implies \norm{\vec{b}}&=\frac{2\sqrt{2}}{3\sqrt{7}}
\end{align}
Thus, 
\begin{align}
\norm{\vec{a}}&=8\norm{\vec{b}}
=\frac{16\sqrt{2}}{3\sqrt{7}}
\end{align}

\item Evaluate the product $(3\overrightarrow {a}-5\overrightarrow {b})\cdot (2\overrightarrow {a}+7\overrightarrow {b})$.
	\\
	\solution
		\begin{multline}
    \brak{3\vec{a}-5\vec{b}}^{\top}\brak{2\vec{a}+7\vec{b}}= 3\vec{a}^{\top}\brak{2\vec{a}+7\vec{b}} - 5\vec{b}^{\top}\brak{2\vec{a}+7\vec{b}}
    \\
     =6\norm{\vec{a}}^2-35\norm{\vec{b}}^2+11\vec{a}^{\top}\vec{b}
\end{multline}

\item Find the magnitude of two vectors $\overrightarrow {a}$ and $\overrightarrow {b}$, having the same magnitude and such that the angle between them is $60\degree$ and their scalar product is $\frac{1}{2}$.
	\\
	\solution
		Given 
\begin{align}
	\norm{\vec{a}}= \norm{\vec{b}}, {\cos\theta} = \frac{1}{2}, 
	\vec{a}^{\top}{\vec{b}} = \frac{1}{2},  \\
\implies 
	\frac{1}{2} = \frac{\frac{1}{2}}{\norm{\vec{a}}^2}
\implies \norm{\vec{a}}
= \norm{\vec{b}}=1
\end{align}
by using  the definition of the scalar product.

\item Find $\abs{\overrightarrow {x}}$, if for a unit vector $\overrightarrow {a}, (\overrightarrow {x}-\overrightarrow {a})\cdot (\overrightarrow {x}+\overrightarrow {a}$)=12.
	\\
\solution 
		From the given information,
\begin{align}
  \label{eq:12/10/3/9det2f}
  \brak{\vec{x}-\vec{a}}^\top\brak{\vec{x}+\vec{a}} &= 12 \\
  \implies \norm{\vec{x}}^{2} - \norm{\vec{a}}^{2} &= 12 \\
\implies   
	\norm{\vec{x}} &= \sqrt{13}
\end{align}

\item If the vertices $A,B,C$ of a triangle $ABC$ are (1,2,3), (-1,0,0), (0,1,2), respectively, then find  $\angle{ABC}$.
	\\
	\solution
		From the given information, 
\begin{align}
\vec{A} - \vec{B} &= \myvec{2\\2\\3},
\vec{C} - \vec{B} = \myvec{1\\1\\2}\\
	\implies \angle{ABC} &= \cos^{-1}{\frac{\brak{\vec{A} -\vec{B}}^{\top}\brak{\vec{C}-\vec{B}}}{\norm{\vec{A} -\vec{B}}  \norm{\vec{C}-\vec{B}}}}\\
&= \cos^{-1}{\frac{10}{\sqrt{102}}}\\
\end{align}




\item Find the angle between x-axis and the line joining points (3,-1) and (4,-2).
\label{chapters/11/10/1/10}
\\
\solution 
The direction vector of the given line is 
\begin{align}
	\vec{C}
=\myvec{ -1\\ 1 }
\end{align}
Hence, the desired angle is given by
\begin{align}
	\cos\theta=\frac{\vec{C}^{\top}\vec{e}_1}{\norm{\vec{C}}\norm{\vec{e}_1}}
	&= -\frac{1}{\sqrt{2}}
	\\
	\implies 
	\theta&=135\degree
 \end{align}

	\item The slope of a line is double of the slope of another line. If tangent of the angle between them is 1/3, find the slopes of the lines.
\label{chapters/11/10/1/11}
\\
\solution 
The direction vectors of the lines can be expressed as
\begin{align}
\vec{m}_1=\myvec{1\\m},
\vec{m}_2=\myvec{1\\2m}
\end{align}
If the angle between the lines be $\theta$,
\begin{align}
\tan \theta = \frac{1}{3}
\implies \cos \theta=\frac{3}{\sqrt{10}}
\end{align}
Thus,
\begin{align}
	\frac{3}{\sqrt{10}} = \frac{\vec{m}_1^\top \vec{m}_2}{\norm{\vec{m}_1}\norm{\vec{m}_2}}
	\\
	= \frac{2m^2 +1}{\sqrt{m^2 + 1}\sqrt{4m^2 + 1}}
	\\
	\implies \frac{9}{10}=\frac{4m^4 + 4m^2 +1}{4m^4 + 5m^2 +1}
\\
	\text{or, } 4m^4 - 5m^2 +1 = 0
\end{align}
yielding
\begin{align}
m=\pm \frac{1}{2}, 
\pm 1
\end{align}

\item    Find angle between the lines, $\sqrt{3}x+y=1$ and $x+\sqrt{3}y$=1.
\label{chapters/11/10/3/9}
   \solution 
	From \eqref{eq:normal-form}, 
 the normal vectors of the given lines can be expressed as
   \begin{align}
	   \vec{n}_1=\myvec{\sqrt{3}\\1},\,
	   \vec{n}_2=\myvec{1\\\sqrt{3}}
   \end{align}
The angle between the lines can then be 
obtained as
\begin{align}
	\cos\theta=\frac{\vec{n}_1^\top\vec{n}_2}{\norm{\vec{n}_1}\norm{\vec{n}_2}}
	=\frac{\sqrt{3}}{2} 
	\\
	\text{or, }
\theta=30\degree
\end{align}

\item The scalar product of the vector $\hat{i}+\hat{j}+\hat{k}$ with a unit vector along the sum of vectors $2\hat{i}+4\hat{j}-5\hat{k}$ and $\lambda\hat{i}+2\hat{j}+3\hat{k}$ is equal to one. Find the value of $\lambda$.
\item Let $\vec{a}$ and $\vec{b}$ be two unit vectors and $\theta$ is the angle between them. Then $\vec{a}+\vec{b}$ is a unit vector if
\begin{enumerate}
\item $\theta=\frac{\pi}{4}$
\item $\theta=\frac{\pi}{3}$
\item $\theta=\frac{\pi}{2}$
\item $\theta=\frac{2\pi}{3}$
\end{enumerate}
\item If $\theta$ is the angle between any two vectors $\vec{a}$ and $\vec{b}$, then $|\vec{a} \cdot \vec{b}|=|\vec{a}\times\vec{b}|$ when $\theta$ is equal to
\begin{enumerate}
\item 0
\item $\frac{\pi}{4}$
\item $\frac{\pi}{2}$
\item $\pi$
\end{enumerate}
\item A vector $\vec{r}$ has a magnitude 14 and direction ratios 2, 3, -6. Find the direction cosines and components of $\vec{r}$, given that $\vec{r}$ makes an acute angle with x-axis.
\item Find the angle between the vectors $2\hat{i}-\hat{j}+\hat{k}$ and $3\hat{i}+4\hat{j}-\hat{k}$.
\item If $\vec{a},\vec{b},\vec{c}$ are the three vectors such that $\vec{a}+\vec{b}+\vec{c}=0$ $\text{ and }$ $|\vec{a}|=2$, $|\vec{b}|$=3, $|\vec{c}|$=5, the value of $\vec{a} \cdot \vec{b}+\vec{b} \cdot \vec{c}+\vec{c} \cdot \vec{a}$ is
	\begin{enumerate}
\item 0
\item 1	
\item -19
\item 38
\end{enumerate}
\item If $\vec{a}$, $\vec{b}$, $\vec{c}$ are unit vectors such that $\vec{a}$+$\vec{b}$+$\vec{c}$=0, then the value of $\vec{a} \cdot \vec{b}+\vec{b} \cdot \vec{c}+\vec{c} \cdot \vec{a}$ is
	\begin{enumerate}
\item 1
\item 3
\item $\frac{-3}{2}$
\item None of these
\end{enumerate}
\item The angles between two vectors $\vec{a}, \vec{b}$ with magnitude $\sqrt{3}, 4$ respectively, and $\vec{a} \cdot \vec{b}= 2\sqrt{3}$ is
	\begin{enumerate}
\item $\frac{\pi}{6}$
\item $\frac{\pi}{3}$
\item $\frac{\pi}{2}$ 
\item $\frac{5\pi}{2}$
\end{enumerate}

\item The vector $\vec{a}+\vec{b}$ bisects the angle between the non-collinear vectors $\vec{a}$ $\text{ and }$ $\vec{b}$ if \rule{1cm}{0.15mm}.
\item The vectors $\vec{a}=3\hat{i}-2\hat{j}+2\hat{k}$ $\text{ and }$ $\vec{b}=\hat{i}-2\hat{k}$ are the adjancent sides of a parallelogram. The acute angle between its diagonals is \rule{1cm}{0.15mm}.
\item If $\vec{a}$ is  any non-zero vector, then $(\vec{a}\cdot \hat{i})\hat{i}$+$(\vec{a}\cdot \hat{j})\hat{j}$+$(\vec{a}\cdot \hat{k})$ $\hat{k}$ equals \rule{1cm}{0.15mm}.
\item If $\vec{a}$ $\text{ and }$ $\vec{b}$ are adjacent sides of a rhombus, then $\vec{a}\cdot \vec{b}$=0.
\item Find the angle between the lines 
\begin{align}
	\overrightarrow{r}&=3\hat{i}-2\hat{j}+6\hat{k}+\lambda(2\hat{i}+\hat{j}+2\hat{k})
	\text{ and}
	\\
	\overrightarrow{r}&=(2\hat{j}-5\hat{k})+\mu(6\hat{i}+3\hat{j}+2\hat{k})
\end{align}
%
\item Find the angle between the lines whose direction cosines are given by the equations $l+m+n=0$, $l^2+m^2-n^2=0$.
\item If a variable line in two adjacent positions has directions cosines $l, m, n$ and $l+\delta l, m+\delta m, n+\delta n$, show that the small angle $\delta\theta$ between the two positions is given by 
\begin{align}
	\delta\theta^2=\delta l^2+\delta m^2+\delta n^2
\end{align}
\item The sine of the angle between the straight line $\dfrac{x-2}{3}=\dfrac{y-3}{4}=\dfrac{z-4}{5}$ and the plane $2x-2y+z=5$ is
\begin{enumerate}
	\item $\dfrac{10}{6\sqrt{5}}$ 
	\item $\dfrac{4}{5\sqrt{2}}$
	\item $\dfrac{2\sqrt{3}}{5}$
	\item $\dfrac{\sqrt{2}}{10}$
\end{enumerate}
\item The plane $2x-3y+6z-11=0$ makes an angle $\sin^{-1}(\alpha)$ with x-axis. The value of $\alpha$ is equal to 
\begin{enumerate}
	\item  $\dfrac{\sqrt{3}}{2}$
	\item  $\dfrac{\sqrt{2}}{3}$
	\item  $\dfrac{2}{7}$
	\item  $\dfrac{3}{7}$
\end{enumerate}
\item The angle between the line $\overrightarrow{r}=(5\hat{i}-\hat{j}-4\hat{k})+\lambda(2\hat{i}-\hat{j}+\hat{k})$ and the plane $\overrightarrow{r} \cdot (3\hat{i}-4\hat{j}-\hat{k})+5=0$ is $\sin^{-1}\brak{\dfrac{5}{2\sqrt{91}}}$.
\item The angle between the planes $\overrightarrow{r} \cdot (2\hat{i}-3\hat{j}+\hat{k})=1$ and $\overrightarrow{r} \cdot (\hat{i}-\hat{j})=4$ is $\cos^{-1} \brak{\dfrac{-5}{\sqrt{58}}}$.
\item Let $\vec{a}$ and $\vec{b}$ be two unit vectors and $\theta$ is the angle between them. Then $\vec{a}+\vec{b}$ is a unit vector if
\begin{enumerate}
\item $\theta=\frac{\pi}{4}$
\item $\theta=\frac{\pi}{3}$
\item $\theta=\frac{\pi}{2}$
\item $\theta=\frac{2\pi}{3}$
\end{enumerate}
\item The value of $\hat{i}\cdot (\hat{j}\times\hat{k})+\hat{j}\cdot (\hat{i}\times\hat{k})+\hat{k}\cdot (\hat{i}\times\hat{j})$ is
\begin{enumerate}
\item 0
\item -1
\item 1
\item 3
\end{enumerate}
\item If $\theta$ is the angle between any two vectors $\vec{a}$ and $\vec{b}$, then $|\vec{a}\cdot \vec{b}|=|\vec{a}\times\vec{b}|$ when $\theta$ is equal to
\begin{enumerate}
\item 0
\item $\frac{\pi}{4}$
\item $\frac{\pi}{2}$
\item $\pi$
\end{enumerate}
\item Let $\vec{a}$ and $\vec{b}$ be two unit vectors and $\theta$ the angle between them. Then $\vec{a}+\vec{b}$ is a unit vector if
	\begin{enumerate}
			\itemsep2pt
		\item $\theta = \frac{\pi}{4}$
		\item $\theta = \frac{\pi}{3}$
		\item $\theta = \frac{\pi}{2}$
		\item $\theta = \frac{2\pi}{3}$
			\end{enumerate}
\solution
Given,
\begin{align}
	\norm{\vec{a}}=\norm{\vec{b}}=1\label{eq:12/10/5/17/1}
	\\
	\norm{\vec{a}+\vec{b}}=1\label{eq:12/10/5/17/2}
\end{align}
Squaring both sides of \eqref{eq:12/10/5/17/2}  , we get
\begin{align}
	\norm{\vec{a}+\vec{b}}^2=1^2
\\	
	\implies \norm{\vec{a}}^2 + \norm{\vec{b}}^2 + 2\vec{a}^{\top}\vec{b} = 1\label{eq:12/10/5/17/3}	
\end{align}
Substituting \eqref{eq:12/10/5/17/1} in \eqref{eq:12/10/5/17/3}, we get
\\
\begin{align}
	\implies 1+1+2(\norm{\vec{a}}\norm{\vec{b}}\cos{\theta})=1
	\\
	\implies 2+2(\norm{\vec{a}}\norm{\vec{b}}\cos{\theta})=1
        \\
	\implies 2(\norm{\vec{a}}\norm{\vec{b}}\cos{\theta})=-1
	\\
	\implies (\norm{\vec{a}}\norm{\vec{b}}\cos{\theta})=\frac{-1}{2}\label{eq:12/10/5/17/4}
\end{align}
Subtituting \eqref{eq:12/10/5/17/1} in \eqref{eq:12/10/5/17/4}, we get
\begin{align}
	\implies \cos{\theta}=\frac{-1}{2}
	\\
	\implies \theta=\frac{2\pi}{3}
\end{align}

\item Let $\vec{a}$ and $\vec{b}$ be two unit vectors and $\theta$ is the angle between them.Then $\vec{a}+\vec{b}$ is a unit vector if
\begin{enumerate}
\item $\theta=\frac{\pi}{4}$
\item $\theta=\frac{\pi}{3}$
\item $\theta=\frac{\pi}{2}$
\item $\theta=\frac{2\pi}{3}$
\end{enumerate}
\item The value of $\hat{i}.(\hat{j}\times\hat{k})+\hat{j}.(\hat{i}\times\hat{k})+\hat{k}.(\hat{i}\times\hat{j})$ is
\begin{enumerate}
\item 0
\item -1
\item 1
\item 3
\end{enumerate}
\item If $\theta$ is the angle between any two vectors $\vec{a}$ and $\vec{b}$,then $|\vec{a}.\vec{b}|=|\vec{a}\times\vec{b}|$ when $\theta$ is equal to
\begin{enumerate}
\item 0
\item $\frac{\pi}{4}$
\item $\frac{\pi}{2}$
\item $\pi$
\end{enumerate}
\item A vector $\vec{r}$ has a magnitude 14 and direction ratios 2,3,-6. Find the direction cosines and components of $\vec{r}$, given that $\vec{r}$ makes an acute angle with x-axis.
\item Find the angle between the vectors $2\hat{i}-\hat{j}+\hat{k}$ $\text{and}$ $3\hat{i}+4\hat{j}-\hat{k}$.
\item If $\vec{a},\vec{b},\vec{c}$ are the three vectors such that $\vec{a}+\vec{b}+\vec{c}=0$ $\text{ and }$ $|\vec{a}|=2$, $|\vec{b}|$=3, $|\vec{c}|$=5, the value of $\vec{a}.\vec{b}+\vec{b}.\vec{c}+\vec{c}.\vec{a}$ is
	\begin{enumerate}
\item 0
\item 1	
\item -19
\item 38
\end{enumerate}
\item If $\vec{a}$, $\vec{b}$, $\vec{c}$ are unit vectors such that $\vec{a}$+$\vec{b}$+$\vec{c}$=0, then the value of $\vec{a}.\vec{b}+\vec{b}.\vec{c}+\vec{c}.\vec{a}$ is
	\begin{enumerate}
\item 1
\item 3
\item $\frac{-3}{2}$
\item None of these
\end{enumerate}
\item The angles between two vectors $\vec{a}$ $\text{and}$ $\vec{b}$ with magnitude $\sqrt{3}$ $\text{ and }$ 4, respectively, and $\vec{a}$, $\vec{b}$= $2\sqrt{3}$ is
	\begin{enumerate}
\item $\frac{\pi}{6}$
\item $\frac{\pi}{3}$
\item $\frac{\pi}{2}$ 
\item $\frac{5\pi}{2}$
\end{enumerate}

\item The vector $\vec{a}+\vec{b}$ bisects the angle between the non-collinear vectors $\vec{a}$ $\text{ and }$ $\vec{b}$ if \rule{1cm}{0.15mm}.
\item The vectors $\vec{a}=3\hat{i}-2\hat{j}+2\hat{k}$ $\text{ and }$ $\vec{b}=\hat{i}-2\hat{k}$ are the adjancent sides of a parallelogram. The acute angle between its diagonals is \rule{1cm}{0.15mm}.
\item If $\vec{a}$ is  any non-zero vector, then $(\vec{a}.\hat{i})\hat{i}$+$(\vec{a}.\hat{j})\hat{j}$+$(\vec{a}.\hat{k})$ $\hat{k}$ equals \rule{1cm}{0.15mm}.
\item If $\vec{a}$ $\text{ and }$ $\vec{b}$ are adjacent sides of a rhombus, then $\vec{a}.\vec{b}$.=0.
\item Find the angle between the lines $$\overrightarrow{r}=3\hat{i}-2\hat{j}+6\hat{k}+\lambda(2\hat{i}+\hat{j}+2\hat{k})\text{ and } \overrightarrow{r}=(2\hat{j}-5\hat{k})+\mu(6\hat{i}+3\hat{j}+2\hat{k})$$
\item Find the angle between the lines whose direction cosines are given by the equations $l+m+n=0$, $l^2+m^2-n^2=0$.
\item If a variable line in two adjacent positions has directions cosines $l, m, n$ and $l+\delta l, m+\delta m, n+\delta n$, show that the small angle $\delta\theta$ between the two positions is given by $$\delta\theta^2=\delta l^2+\delta m^2+\delta n^2$$ 
\item The sine of the angle between the straight line $\dfrac{x-2}{3}=\dfrac{y-3}{4}=\dfrac{z-4}{5}$ and the plane $2x-2y+z=5$ is
\begin{enumerate}
	\item $\dfrac{10}{6\sqrt{5}}$ 
	\item $\dfrac{4}{5\sqrt{2}}$
	\item $\dfrac{2\sqrt{3}}{5}$
	\item $\dfrac{\sqrt{2}}{10}$
\end{enumerate}
\item The plane $2x-3y+6z-11=0$ makes an angle $\sin^{-1}(\alpha)$ with x-axis. The value of $\alpha$ is equal to 
\begin{enumerate}
	\item  $\dfrac{\sqrt{3}}{2}$
	\item  $\dfrac{\sqrt{2}}{3}$
	\item  $\dfrac{2}{7}$
	\item  $\dfrac{3}{7}$
\end{enumerate}
\item The angle between the line $\overrightarrow{r}=(5\hat{i}-\hat{j}-4\hat{k})+\lambda(2\hat{i}-\hat{j}+\hat{k})$ and the plane $\overrightarrow{r} \cdot (3\hat{i}-4\hat{j}-\hat{k})+5=0$ is $\sin^{-1}\brak{\dfrac{5}{2\sqrt{91}}}$.
\item The angle between the planes $\overrightarrow{r} \cdot (2\hat{i}-3\hat{j}+\hat{k})=1$ and $\overrightarrow{r} \cdot (\hat{i}-\hat{j})=4$ is $\cos^{-1} \brak{\dfrac{-5}{\sqrt{58}}}$.
\item Let $\vec{a}$ and $\vec{b}$ be two unit vectors and $\theta$ is the angle between them.Then $\vec{a}+\vec{b}$ is a unit vector if
\begin{enumerate}
\item $\theta=\frac{\pi}{4}$
\item $\theta=\frac{\pi}{3}$
\item $\theta=\frac{\pi}{2}$
\item $\theta=\frac{2\pi}{3}$
\end{enumerate}
\item The value of $\hat{i}.(\hat{j}\times\hat{k})+\hat{j}.(\hat{i}\times\hat{k})+\hat{k}.(\hat{i}\times\hat{j})$ is
\begin{enumerate}
\item 0
\item -1
\item 1
\item 3
\end{enumerate}
\item If $\theta$ is the angle between any two vectors $\vec{a}$ and $\vec{b}$,then $|\vec{a}.\vec{b}|=|\vec{a}\times\vec{b}|$ when $\theta$ is equal to
\begin{enumerate}
\item 0
\item $\frac{\pi}{4}$
\item $\frac{\pi}{2}$
\item $\pi$
\end{enumerate}
\item Find the angle between the lines $y(2-\sqrt{3})(x+5)\text{ and }y=(2+\sqrt{3})(x-7)$.
\item Show that the tangent of an angle between the lines $\frac{x}{a}+\frac{y}{b}=1 \text{ and }\frac{x}{a}-\frac{y}{b}=1$ is $\frac{2ab}{a^2-b^2}$.
\item The unit vector normal to the plane $x+2y+3z-6=0$ is $\dfrac{1}{\sqrt{14}}\hat{i} + \dfrac{2}{\sqrt{14}}\hat{j} + \dfrac{3}{\sqrt{14}}\hat{k}$.
\end{enumerate}
