\numberwithin{equation}{subsection}
The equation of the incircle is given by 
		\begin{align}
			 \label{eq:conic_quad_form}
	  {g}\brak{\vec{x}}=
			\vec{x}^{\top}\vec{V}\vec{x}+2\vec{u}^{\top}\vec{x}+f = 0
		\end{align}
		where 
		\begin{align}
			 \label{eq:conic_quad_form-params}
			\vec{V}=\vec{I}, \vec{u}=-\vec{O}, f = \norm{\vec{O}}-r^2,
		\end{align}
		$\vec{O}$ being the incentre and $r$ the inradius.  Here $\vec{I}$ is the identity matrix.
\begin{enumerate}[label=\thesubsection.\arabic*.,ref=\thesubsection.\theenumi]
\numberwithin{equation}{enumi}
\item Compute 
\begin{align}
	\label{eq:incircle-disc-Sigma}
\vec{\Sigma} = 
\brak{\vec{V}\vec{h}+\vec{u}}
	  \brak{\vec{V}\vec{h}+\vec{u}}^{\top}
   -
	  {g}\brak{\vec{h}}\vec{V}
\end{align}
for $\vec{h}=\vec{A}$.
\item Find the roots of the equation
\begin{align}
	\mydet{\lambda \vec{I}-\vec{\Sigma}} = 0
\end{align}
These are known as the eigenvalues of $\vec{\Sigma}$.
\item Find $\vec{p}$  such that 
\begin{align}
	\vec{\Sigma}\vec{p}
	=\lambda\vec{p}
\end{align}
using row reduction.  These are known as the eigenvectors of $\Sigma$.
\item Define
    \begin{align}
      \label{eq:conic_parmas_eig_def-D}
      \vec{D} &= \myvec{\lambda_1 & 0\\ 0 & \lambda_2}, 
      \\
	    \vec{P} &= \myvec{\frac{\vec{p}_1}{\norm{\vec{p}_1}} & \frac{\vec{p}_2}{\norm{\vec{p}_2}}}
      \label{eq:eigevecP}
    \end{align}
    \item Verify that
  \begin{align}
\vec{P}^{\top}=\vec{P}^{-1}.
  \label{eq:orth-mat}
  \end{align}
  $\vec{P}$ is defined to be an orthogonal matrix.
\item Verify that
    \begin{align}
      \label{eq:conic_parmas_eig_def}
      \vec{P}^{\top}\vec{\Sigma}\vec{P} &= \vec{D},
    \end{align} 
		This is known as the spectral (eigenvalue ) decomposition of a symmetric matrix 

\item
	The direction vectors of the tangents from a point 
$\vec{h}$ to the circle in \eqref{eq:conic_quad_form} are given by  
\begin{align}
  \vec{m}&= \vec{P}\myvec{\sqrt{\abs{\lambda_2}} \\[2mm]  \pm \sqrt{\abs{\lambda_1}}}
	  \label{eq:h-tangents-cond-mPlam}
\end{align}
\item The points of contact of the pair 
of tangents 
to the circle in \eqref{eq:conic_quad_form} 
	from 
	a point $\vec{h}$ 
	are given by 
  \begin{align}
  \label{eq:line_dir_pt-lam}
	  \vec{x}  = \vec{h} + \mu \vec{m}
  \end{align}
  where 
  \begin{align}
  \label{eq:line_dir_pt-lam-mu}
	  \mu = -\frac{\vec{m}^{\top}\brak{\vec{V}\vec{h}+\vec{u}}}{\vec{m}^{\top}\vec{V}\vec{m} }
  \end{align}
	for $\vec{m}$ in 
	  \eqref{eq:h-tangents-cond-mPlam}.
  Compute the points of contact. You should get the same points that you obtained in the previous section. 

All codes for this section are available at
\begin{lstlisting}
	codes/triangle/tangpair.py
\end{lstlisting}
\end{enumerate}
