\begin{enumerate}[label=\thesubsection.\arabic*.,ref=\thesubsection.\theenumi]
	\item The matrix of the vertices of the triangle is defined as
		\begin{align}
			\vec{P} = \myvec{\vec{A} & \vec{B} &\vec{C}}
		\end{align}
\item Obtain the direction matrix of the sides of $\triangle ABC$
	defined as 
		\begin{align}
		\vec{M} = 	\myvec{\vec{A}-\vec{B} & \vec{B}-\vec{C} & \vec{C}-\vec{A}}
		\end{align}
	\\
		\solution 

		\begin{align}
			\vec{M} &= \myvec{\vec{A}-\vec{B} & \vec{B}-\vec{C} & \vec{C}-\vec{A}}
			\\
			&= 
			\myvec{\vec{A} & \vec{B} &\vec{C}}
			\myvec{1 & 0 & -1 \\ -1 & 1 & 0 \\ 0 & -1 & 1}
		\end{align}
		where the second matrix above 
		is known as a {\em circulant} matrix.  Note that the 2nd and 3rd row of the above matrix are circular shifts of the 1st row.
	\item Obtain the normal matrix  of the sides of $\triangle ABC$
		\\
		\solution Considering the roation matrix
		\begin{align}
			\vec{R}  = \myvec{0 & -1 \\ 1 & 0},
		\end{align}
		the normal matrix is obtained as
		\begin{align}
			\vec{N} = \vec{R}\vec{M} 
		\end{align}

	\item Obtain $a, b, c$.
		\\
		\solution The sides vector is obtained as
		\begin{align}
			\vec{d} = \sqrt{\text{diag}(\vec{M}^{\top}\vec{M})}
		\end{align}
	\item Obtain the constant terms in the equations of the sides of the triangle. 
		\\
		\solution The constants for the lines can be expressed in vector form as
		\begin{align}
			\vec{c} = \text{diag}\cbrak{\brak{\vec{N}^{\top}\vec{P}}} 
		\end{align}
\item Obtain the mid point matrix for the sides of the triangle
	\\
		\solution
		\begin{align}
			\myvec{\vec{D} & \vec{E} &\vec{F}} &= \frac{1}{2}\myvec{\vec{A} & \vec{B} &\vec{C}}
			\myvec{0 & 1 & 1 \\ 1 & 0 & 1 \\ 1 & 1 & 0}
		\end{align}
	\item Obtain the median direction matrix.
\\
\solution The median direction matrix is given by 
		\begin{align}
			\vec{M}_1 &= \myvec{\vec{A}-\vec{D} & \vec{B}-\vec{E} & \vec{C}-\vec{F}}
			\\
			&= 
			  \myvec{
				  \vec{A}-\frac{\vec{B}+\vec{C}}{2} &
			  \vec{B}-\frac{\vec{C}+\vec{A}}{2} &
			  \vec{C}-\frac{\vec{A}+\vec{B}}{2}} 
			  \\
			  &= \myvec{\vec{A} & \vec{B} &\vec{C}}
			  \myvec{
				  1 & -\frac{1}{2} & -\frac{1}{2}
				  \\
				  -\frac{1}{2} & 1 & -\frac{1}{2}
				  \\
				  -\frac{1}{2} & -\frac{1}{2} & 1
				  }
		\end{align}
	\item Obtain the median normal matrix.
	\item Obtian the median equation constants.
	\item Obtain the centroid by finding the intersection of the medians.

\item Find the normal matrix for the altitudes
	\\
		\solution  The desired matrix is 
\begin{align}
	\vec{M}_2 &= 	\myvec{\vec{B}-\vec{C} & \vec{C}-\vec{A} & \vec{A}-\vec{B} }
	\\
	&= 
	\myvec{\vec{A} & \vec{B} &\vec{C}}
			\myvec{ 0 & -1 & 1 \\ 1 & 0 & -1 \\ -1 & 1 & 0}
		\end{align}

	\item Find the constants vector for the altitudes.
		\\
		\solution The desired vector is 
		\begin{align}
			\vec{c}_2 = \text{diag}\cbrak{\brak{\vec{M}^{\top}\vec{P}}} 
		\end{align}
\item Find the normal matrix for the perpendicular bisectors
	\\
	\solution The normal matrix is $\vec{M}_2$
\item Find the constants vector for the perpendicular bisectors.
		\\
		\solution The desired vector is 
		\begin{align}
			\vec{c}_3 = \text{diag}\cbrak{\vec{M}_2^{\top}\myvec{\vec{D} & \vec{E} &\vec{F}}}
		\end{align}

\item Find the points of contact.
	\\
		\solution The points of contact are given by 
		\begin{align}
			\myvec{			
						\frac{m\vec{C}+n\vec{B}}{m+n}
			&
			\frac{n\vec{A}+p\vec{C}}{n+p}
			&
						\frac{p\vec{B}+m\vec{A}}{p+m}
			}
			= 	\myvec{\vec{A} & \vec{B} &\vec{C}}
			\myvec{
				0 &			\frac{n}{b} & \frac{m}{c}  
				\\
			 \frac{n}{a}& 0 & \frac{p}{c} 
				\\
			\frac{m}{a}	&		\frac{p}{b} & 0  
			}
		\end{align}
All codes for this section are available at
\begin{lstlisting}
	codes/triangle/mat-alg.py
\end{lstlisting}
\end{enumerate}
