\begin{enumerate}[label=\thesubsection.\arabic*.,ref=\thesubsection.\theenumi]
	\item The angle $\theta$ between $\vec{a}, \vec{b}$,
		is given by 
\begin{align}
	\label{eq:angle-inner}
		\cos\theta=\frac{{\vec{a}^{\top}}{\vec{b}}}{\norm{\vec{a}}\norm{\vec{b}}}
\end{align}
\item The equation of a line is given by 
\begin{align}
	\label{eq:param-form}
	\vec{x} = \vec{h} + \kappa \vec{m}
\end{align}
\item 
	For
\begin{align}
	\vec{m}^{\top}\vec{n} = 0,
\end{align}
which means that $\vec{m}\perp \vec{n}$,
	\eqref{eq:param-form} can be expressed as
\begin{align}
	\vec{n}^{\top}\vec{x} &= \vec{n}^{\top}\vec{h} + \kappa \vec{n}^{\top}\vec{m}
	\\
	\label{eq:normal-form}
	\implies	\vec{n}^{\top}\vec{x} &= c
\end{align}
for 
\begin{align}
c = 	\vec{n}^{\top}\vec{h}. 
\end{align}
$\vec{n}$ is defined to be the
{\em normal vector}
		of the line.  
	In 3D, \eqref{eq:normal-form} represents a plane.
	\item Mathematically, 
the projection of $\vec{A}$ on $\vec{B}$ is defined as
		\begin{align}
	\vec{C} = k \vec{B},\, \text{such that}
	\brak{\vec{A}-\vec{C}}^{\top}\vec{C} = 0
\end{align}
yielding
\begin{align}
	\brak{\vec{A}-k\vec{B}}^{\top}\vec{B} = 0
	\\
	\text{or, } k = 
	\frac{\vec{A}^{\top}\vec{B}}{\norm{\vec{B}}^2}
	\implies 
	\vec{C} = 
	\frac{\vec{A}^{\top}\vec{B}}{\norm{\vec{B}}^2}
 \vec{B}
	\label{eq:12/10/3/4/proj}
\end{align}
\item If $\vec{A}, \vec{B}$ are unit vectors, 
\begin{align}
	\brak{\vec{A}-\vec{B}}^{\top} 
	\brak{\vec{A}+\vec{B}} 
	=
\norm{\vec{A}}^2 - \norm{\vec{B}}^2
	= 0
	\label{eq:12/10/3/11/unit}
\end{align}
  \item If 
\begin{align}
	\vec{A}^{\top}\vec{A} =\vec{I},
\label{eq:12/10/3/5/inner}
\end{align}
		then $	\vec{A}$ is an {\em orthogonal} matrix.  This also means that its rows and columns are unit vectors and mutually perpendicular.
	\item The determinant
\begin{align}
  \label{eq:det-2d}
	\mydet{a_1 & b_1 \\ a_2 & b_2} = a_1 b_2 - a_2 b_1.
\end{align}
\item Let 
\begin{align}
  \vec{A} &= \myvec{a_1\\a_2 \\ a_3} \equiv a_1\overrightarrow{i}+a_2\overrightarrow{j}+a_3\overrightarrow{j}, 
  \\
  \vec{B} &= \myvec{b_1\\b_2 \\ b_3}, 
\end{align}
and 
\begin{align}
  \label{eq:cross3d-submat}
\begin{split}
  \vec{A}_{ij} &= \myvec{a_i\\a_j}, 
  \\
  \vec{B}_{ij} &= \myvec{b_i\\b_j}. 
\end{split}
\end{align}

\item The {\em cross product} or {\em vector product} of $\vec{A}, \vec{B}$ is defined as
\begin{align}
  \label{eq:cross3d}
	\vec{A} \times \vec{B} 
	 = \myvec{ \mydet{\vec{A}_{23} & \vec{B}_{23}} \\[1ex] \mydet{\vec{A}_{31} & \vec{B}_{31}} \\[1ex] \mydet{\vec{A}_{12}  & \vec{B}_{12}}}
\end{align}
\item Verify that
\begin{align}
  \label{eq:cross3d-commute}
  \vec{A} \times \vec{B} = -  \vec{B} \times \vec{A} 
  \\
  \label{eq:cross3d-same}
  \vec{A} \times \vec{A} = \vec{0}
\end{align}
\item If 
		\label{prop:lin-dep-cross}
\begin{align}
  \vec{A} \times \vec{B} = \vec{0},
\end{align}
  $\vec{A}$ and $ \vec{B} $ are linearly independent, i.e., they are points on the same line.
  \item 
\begin{align}
	\label{eq:cross-sin}
	\norm{ \vec{A} \times \vec{B} }
	=
	\norm{\vec{A}} \times 	\norm{\vec{B}} \sin \theta
\end{align}
where $\theta$ is the angle between the vectors.
\item 
\begin{align}
	ar\brak{ABCD} = 
         \frac{1}{2}\brak{\brak{\vec{C}-\vec{A}}\times\brak{\vec{D}-\vec{B}}} \\
        \label{eq:11/10/1/1area-diag} 
\end{align}
\item The area of $\triangle ABC$ is 
		\begin{align}
			\label{eq:tri-area-cross}
			\frac{1}{2}\norm{{\brak{\vec{A}-\vec{B}}\times \brak{\vec{A}-\vec{C}}}}
		\end{align}
	\item 
The affine transformation is given by 
\begin{align}
	\label{eq:conic_affine}
	\vec{x} = \vec{P}\vec{y}+\vec{c}
\end{align}
where $\vec{c}$ is the translation vector.
\item The matrix
\begin{align}
\vec{P} =
\myvec{
\cos\theta & -\sin\theta \\
\sin\theta & \cos\theta 
}
\end{align}
is defined to be the rotation matrix. 
\item 
\begin{align}
	\vec{P}^{\top} \vec{P} = \vec{I}
\end{align}
		$\vec{P}$ is known as as {\em orthogonal} matrix.
\item Given vertices $\vec{A}, \vec{C}$ of a square, the other two vertices are given by
\begin{align}
\begin{split}
	\vec{B} = \norm{\vec{C}-\vec{A}}\cos \frac{\pi}{4}\vec{P}\vec{e}_1+\vec{A}
	\\
	\vec{D} = \norm{\vec{C}-\vec{A}}\cos \frac{\pi}{4}\vec{P}\vec{e}_2+\vec{A}
\end{split}
	\label{eq:affine-square-bd}
\end{align}
%-c}}{\norm{\vec{n}}}\vec{n}
\item Code for orthogonality
	\begin{lstlisting}
	codes/book/orth.py
\end{lstlisting}
\item Code for cross product
	\begin{lstlisting}
	codes/book/cross.py
\end{lstlisting}
\iffalse
	\\
		\solution Shifting $\vec{A}$ to the origin and rotating the square clockwise by an angle $\phi$ made by $CA$ with the $x$-axis,
	from \eqref{eq:conic_affine},
\begin{align}
\vec{A} = \vec{P}\vec{0}+\vec{c}
\\
\implies 
\vec{c} = \vec{A}
\\
	\theta =  \phi -\frac{\pi}{4} 
\end{align}
and we obtain a square with the other vertices as
\begin{align}
\begin{split}
	\vec{B}_1 = \norm{\vec{C}-\vec{A}}\cos \frac{\pi}{4}\vec{e}_1
	\\
	\vec{D}_1 = \norm{\vec{C}-\vec{A}}\cos \frac{\pi}{4}\vec{e}_2
\end{split}
	\label{eq:affine-bd}
\end{align}
	From \eqref{eq:conic_affine}
	and 
	\eqref{eq:affine-bd},
	we obtain \eqref{eq:affine-square-bd}.
	\fi
\end{enumerate}
