%\begin{enumerate}[label=\arabic*.,ref=\theenumi]
\begin{enumerate}[label=\thesubsection.\arabic*.,ref=\thesubsection.\theenumi]
	\item 
The affine transformation is given by 
\begin{align}
	\label{eq:conic_affine}
	\vec{x} = \vec{P}^{\top}\vec{y}+\vec{c}
\end{align}
where 
\begin{align}
\vec{P} =
\myvec{
\cos\theta & -\sin\theta \\
\sin\theta & \cos\theta 
}
\end{align}
is the rotation matrix and $\vec{c}$ is the translation vector.
\item Given vertices $\vec{A}, \vec{C}$ of a square, the other two vertices are given by
\begin{align}
\begin{split}
	\vec{B} = \norm{\vec{C}-\vec{A}}\cos \frac{\pi}{4}\vec{P}^{\top}\vec{e}_1+\vec{A}
	\\
	\vec{D} = \norm{\vec{C}-\vec{A}}\cos \frac{\pi}{4}\vec{P}^{\top}\vec{e}_2+\vec{A}
\end{split}
	\label{eq:affine-square-bd}
\end{align}
%-c}}{\norm{\vec{n}}}\vec{n}
	\\
		\solution Shifting $\vec{A}$ to the origin and rotating the square clockwise by an angle $\phi$ made by $CA$ with the $x$-axis,
	from \eqref{eq:conic_affine},
\begin{align}
\vec{A} = \vec{P}\vec{0}+\vec{c}
\\
\implies 
\vec{c} = \vec{A}
\\
	\theta =  \phi -\frac{\pi}{4} 
\end{align}
and we obtain a square with the other vertices as
\begin{align}
\begin{split}
	\vec{B}_1 = \norm{\vec{C}-\vec{A}}\cos \frac{\pi}{4}\vec{e}_1
	\\
	\vec{D}_1 = \norm{\vec{C}-\vec{A}}\cos \frac{\pi}{4}\vec{e}_2
\end{split}
	\label{eq:affine-bd}
\end{align}
	From \eqref{eq:conic_affine}
	and 
	\eqref{eq:affine-bd},
	we obtain \eqref{eq:affine-square-bd}.
\end{enumerate}
