\begin{enumerate}[label=\thesubsubsection.\arabic*.,ref=\thesubsubsection.\theenumi]
	\item Mathematically, 
the projection of $\vec{A}$ on $\vec{B}$ is defined as
		\begin{align}
	\vec{C} = k \vec{B},\, \text{such that}
	\brak{\vec{A}-\vec{C}}^{\top}\vec{C} = 0
\end{align}
yielding
\begin{align}
	\brak{\vec{A}-k\vec{B}}^{\top}\vec{B} = 0
	\\
	\text{or, } k = 
	\frac{\vec{A}^{\top}\vec{B}}{\norm{\vec{B}}^2}
	\implies 
	\vec{C} = 
	\frac{\vec{A}^{\top}\vec{B}}{\norm{\vec{B}}^2}
 \vec{B}
	\label{eq:12/10/3/4/proj}
\end{align}
\item If $\vec{A}, \vec{B}$ are unit vectors, 
\begin{multline}
	\brak{\vec{A}-\vec{B}}^{\top} 
	\brak{\vec{A}+\vec{B}} 
	\\
\norm{\vec{A}}^2 - \norm{\vec{B}}^2
	= 0
	\label{eq:12/10/3/11/unit}
\end{multline}
  \item If $ABCD$ be a parallelogram,
	  \label{prop:two-pgm}
  \begin{align}
	  \label{eq:two-pgm}
 \vec{B}-\vec{A} = \vec{C} -\vec{D}
  \end{align}
  \item 
If $PQRS$ is formed by joining the mid points of $ABCD$, 
\begin{align}
  \vec{P} = \frac{1}{2}\brak{\vec{A}+\vec{B}} 
  ,\,
 \vec{Q} = \frac{1}{2}\brak{\vec{B}+\vec{C}} 
 \\
 \vec{R} = \frac{1}{2}\brak{\vec{C}+\vec{D}}   
  ,\,
 \vec{S} = \frac{1}{2}\brak{\vec{D}+\vec{A}}  
 \\
	\implies 
 \vec{P}-\vec{Q} = \vec{S} -\vec{R}.
  \label{eq:10/7/4/8det2f}
\end{align}
Hence, $PQRS$ is a parallelogram
	  from \eqref{eq:two-pgm}.
  \item If 
\begin{align}
	\vec{A}^{\top}\vec{A} =\vec{I},
\label{eq:12/10/3/5/inner}
\end{align}
		then $	\vec{A}$ is an {\em orthogonal} matrix.
\end{enumerate}
