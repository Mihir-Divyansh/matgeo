\begin{enumerate}[label=\thesubsection.\arabic*.,ref=\thesubsection.\theenumi]
	\item For a symmetric matrix, from \eqref{eq:conic_parmas_eig_def}, we have the eigendecomposition
\begin{align}
	\vec{V}&=\vec{P}\vec{D}
\vec{P}^{\top}
\end{align}
where 
%
\begin{align}
	\vec{P} &= \myvec{\vec{p}_1 & \vec{p}_2}, \quad 
\vec{P}^{\top}\vec{P} = 
	\vec{I}
\label{eq:eig_matrix}
\\
      \label{eq:D}
	  \vec{D} &= \myvec{\lambda_1 & 0 \\ 0 & \lambda_2}
\end{align}
			\item Using the affine transformation in
					\label{app:std-prm-P}
	\eqref{eq:conic_affine},
	the conic in     \eqref{eq:conic_quad_form} can be expressed in standard form 
	%(centre/vertex at the origin, major axis - $x$ axis)
	as
  \begin{align}
    %\begin{aligned}
    \label{eq:conic_simp_temp_nonparab}
	    \vec{y}^{\top}\brak{\frac{\vec{D}}{f_0}}\vec{y} &= 1   &  \abs{\vec{V}} &\ne 0
    \\
	    \vec{y}^{\top}\vec{D}\vec{y} &=  -\eta\vec{e}_1^{\top}\vec{y}   & \abs{\vec{V}} &= 0
    \label{eq:conic_simp_temp_parab}
    %\end{aligned}
    \end{align}
    where
  \begin{align}
      %\begin{split}
      \label{eq:f0}
	  f_0 &=\vec{u}^{\top}\vec{V}^{-1}\vec{u} -f \ne 0
	  \\
      \label{eq:eta}
       \eta &=2\vec{u}^{\top}\vec{p}_1
       \\
       \vec{e}_1 &=\myvec{1 \\ 0}
      \end{align}
  \solution  See \appref{app:parab}.
	  
    \item\leavevmode
		\begin{enumerate}
			\item The directrices for the  standard conic are given by 
				\begin{align}
					\label{eq:dx-ell-hyp}
					\vec{e}_1^{\top}\vec{y} &=  
					%\pm\sqrt{\abs{\frac{f_0\lambda_2}{\lambda_1\brak{\lambda_2-\lambda_1}}}} & e \ne 1
					\pm \frac{1}{e}\sqrt{\frac{\abs{f_0}}{\lambda_2\brak{1-e^2}}} & e \ne 1
					\\
					\vec{e}_1^{\top}\vec{y} &= \frac{\eta}{2\lambda_2} & e = 1
					\label{eq:dx-parab}
				\end{align}
    \item The foci of the standard ellipse and hyperbola are given by 
				\begin{align}
					\label{eq:F-ell-hyp-parab}
					\vec{F} 
=
					\begin{cases}
						\pm e\sqrt{\frac{\abs{f_0}}{\lambda_2\brak{1-e^2}}}\vec{e}_1 & e \ne 1
						\\
						 -\frac{\eta}{4\lambda_2}\vec{e}_1 & e = 1
					\end{cases}
				\end{align}
	
		\end{enumerate}
	%	where, without loss of generality, $f_0 < 0$ for the hyperbola.
    
	\begin{proof}%\leavevmode
  \label{app:foc-dir}
		For the standard conic, 
				\begin{align}
					\label{eq:std-prm-P}
					\vec{P} &= \vec{I}
					\\
					\vec{u} &= 
				\begin{cases}
				0 & e \ne 1
       \\
				\frac{\eta}{2} \vec{e}_1 & e = 1
				\end{cases}
				\label{eq:std-prm-u}
				\\
				\lambda_1 &  
					\begin{cases}
						=0 & e = 1
						\\
						\ne 0 & e \ne 1
					\end{cases}
				\label{eq:std-prm-lam1}
				\end{align}
				where 
				\begin{align}
					\vec{I} = \myvec{\vec{e}_1 & \vec{e}_2}
				\end{align}
				is the identity matrix.
%  \input{appendix.tex}
		\begin{enumerate}
			\item For the standard hyperbola/ellipse in \eqref{eq:conic_simp_temp_nonparab}, from 
					\eqref{eq:std-prm-P},
\eqref{eq:conic_quad_form_nc}
and 
					\eqref{eq:std-prm-u},
				\begin{align}
\label{eq:n-ell-hyp}
					\vec{n} &= \sqrt{\frac{\lambda_2}{f_0}} \vec{e}_1 
					\\
					c &= 
					%\pm \frac{\sqrt{-\lambda_2\brak{e^2-1}\brak{\lambda_2 f_0}}}{\lambda_2e\brak{e^2-1}}
					\pm \frac{\sqrt{-\frac{\lambda_2}{f_0}\brak{e^2-1}\brak{\frac{\lambda_2}{ f_0}}}}{\frac{\lambda_2}{f_0}e\brak{e^2-1}}
					\\
					&=\pm \frac{1}{e\sqrt{1-e^2}}
%					\\
%					&=\pm\sqrt{\abs{\frac{f_0}{\brak{1 - \frac{\lambda_1}{\lambda_2}}\frac{\lambda_1}{\lambda_2}}}}
\label{eq:c-ell-hyp}
				\end{align}
				yielding 
					\eqref{eq:dx-ell-hyp} upon substituting from 
\eqref{eq:conic_quad_form_e} and simplifying.
For the standard parabola in \eqref{eq:conic_simp_temp_parab},  from 
					\eqref{eq:std-prm-P},
\eqref{eq:conic_quad_form_nc}
and 
					\eqref{eq:std-prm-u}, noting that $f = 0$,

				\begin{align}
\label{eq:n-parab}
					\vec{n} &= \sqrt{\lambda_2} \vec{e}_1 
					\\
					c &=
	\frac{\norm{\frac{\eta}{2} \vec{e}_1}^2   }{2\vec{\brak{\frac{\eta}{2}} \brak{\vec{e}_1}^{\top}\vec{n}}} 
\\
					\\
					&= \frac{\eta}{4\sqrt{\lambda_2}}
\label{eq:c-parab}
				\end{align}
				yielding 
					\eqref{eq:dx-parab}.

				\item 	For the standard ellipse/hyperbola, substituting from
\eqref{eq:c-ell-hyp},
\eqref{eq:n-ell-hyp},
\eqref{eq:std-prm-u}
and \eqref{eq:conic_quad_form_e}
in \eqref{eq:conic_quad_form_F},
				\begin{align}
					\vec{F} &= \pm \frac{\brak{\frac{1}{e\sqrt{1-e^2}}}\brak{e^2}\sqrt{\frac{\lambda_2}{f_0}}\vec{e}_1}{\frac{\lambda_2}{f_0}}
					%\pm\sqrt{\abs{\frac{f_0}{\brak{1 - \frac{\lambda_1}{\lambda_2}}\frac{\lambda_1}{\lambda_2}}}}
					%\brak{1 - \frac{\lambda_1}{\lambda_2}}\frac{\sqrt{\lambda_2}}{\lambda_2}\vec{e}_1
 			\end{align}
			yielding
					\eqref{eq:F-ell-hyp-parab}
					after simplification.
					For the standard parabola, substituting from 
\eqref{eq:c-parab},
\eqref{eq:n-parab},
\eqref{eq:std-prm-u}
and \eqref{eq:conic_quad_form_e}
in \eqref{eq:conic_quad_form_F},			
				\begin{align}
	\vec{F}  &= \frac{\brak{\frac{\eta}{4\sqrt{\lambda_2}}}\sqrt{\lambda_2}\vec{e}_1-\vec{\frac{\eta}{2} \vec{e}_1}}{\lambda_2}
\\
				\end{align}
				yielding 
					\eqref{eq:F-ell-hyp-parab} after simplification.

		\end{enumerate}
%		See Appendix \ref{app:foc-dir}.
	\end{proof}
%
    \item The equation of the minor and major  axes for the ellipse/hyperbola are respectively given by 
  \begin{align}
\vec{p}_i^{\top}\brak{\vec{x}-\vec{c}} = 0, i = 1,2
	  \label{eq:major-minor-axis-quad}
  \end{align}
  The axis of symmetry for the parabola is also given by 
	  \eqref{eq:major-minor-axis-quad}.

		\begin{proof}
From		\eqref{corr:axis}, the major/symmetry axis for the hyperbola/ellipse/parabola can be expressed using 
	\eqref{eq:conic_affine}
 as
  \begin{align}
	  \vec{e}_2^{\top}
		  \vec{P}^{\top}\brak{\vec{x}-\vec{c}} &= 0
		  \\
	  \implies 		  \brak{\vec{P}\vec{e}_2}^{\top}\brak{\vec{x}-\vec{c}} &= 0
  \end{align}
yielding	  \eqref{eq:major-minor-axis-quad}, and the proof for the minor axis is similar.
		\end{proof}
	\item
			\label{corr:center}
			The center of the standard ellipse/hyperbola, defined to be the mid point of the line joining the foci, is the origin.
	
	\item
		\label{corr:axis}
			The principal (major) axis of the standard ellipse/hyperbola, defined to be the line joining the two foci   is the $x$-axis.  
	
	\begin{proof}
		From 	\eqref{eq:F-ell-hyp-parab}, it is obvious that the line joining the foci passes through the origin.  Also, the direction vector of this line is $\vec{e}_1$.  Thus, the principal axis is the $x$-axis. 
	\end{proof}
	\item
		\label{corr:minor-axis}
			The minor axis of the standard ellipse/hyperbola, defined to be the line orthogonal to the $x$-axis is the $y$-axis. 
	


	\item
			The axis of symmetry of the standard parabola, defined to be the line perpendicular to the directrix and passing through the focus,  is the $x$- axis.
	
	\begin{proof}
	From \eqref{eq:n-parab} and 	
					\eqref{eq:F-ell-hyp-parab}, 
					the axis of the parabola  can be expressed 
     as 
		\begin{align}
			\vec{e}_2^{\top}\brak{\vec{y}  
			+\frac{\eta}{4\lambda_2}\vec{e}_1} &= 0
			\\
			\implies \vec{e}_2^{\top}\vec{y} &= 0
					\label{eq:axis-std-parab}, 
		\end{align}
		which is the equation of the $x$-axis.
	\end{proof}


	\item
			\label{corr:center-parab}
 The point where the parabola intersects its axis of symmetry is called the vertex. For the standard parabola, the vertex is the origin.
	
	\begin{proof}
					\eqref{eq:axis-std-parab} can be expressed as 
    \begin{align}
			\vec{y}= \alpha \vec{e}_1. 
					\label{eq:axis-std-parab-dir} 
    \end{align}
					Substituting \eqref{eq:axis-std-parab-dir} in 
    \eqref{eq:conic_simp_temp_parab}, 
    \begin{align}
	     \alpha^2 \vec{e}_1^{\top}\vec{D} \vec{e}_1 &=  -\eta\alpha \vec{e}_1^{\top} \vec{e}_1   
	     \\
	     \implies \alpha &=0, \text{ or, } \vec{y} = \vec{0}.
    %\end{aligned}
    \end{align}
	\end{proof}
	\item
			\label{corr:foclen}
	 The {\em focal length} of the standard parabola, , defined to be the distance between the vertex and the focus, measured along the axis of symmetry, is $\abs{\frac{\eta}{4 \lambda_2}}$
    \item For the standard hyperbola/ellipse, the length of the major axis is 
  \begin{align}
\label{eq:chord-len-major}
 2\sqrt{\abs{\frac{
f_0}
{\lambda_1}
	  }}
  \end{align}
  and the minor axis is 
  \begin{align}
\label{eq:chord-len-minor}
 2\sqrt{\abs{\frac{
f_0}
{\lambda_2}
	  }}
  \end{align}
  \solution 
		See \appref{app:major}.
\item
    The latus rectum of a conic section is the chord that passes through the focus and is perpendicular to the major axis.
	The length of the latus rectum for a conic is given by
		\begin{align}
			l =
			\begin{cases}
				2\frac{\sqrt{\abs{f_0\lambda_1}}}{\lambda_2} & e \ne 1
			\\
			\frac{\eta}{\lambda_2} & e = 1
			\end{cases}
			\label{eq:latus-ellipse}
		\end{align}
		\solution 
		See \appref{app:latus}.
%	    $\mydet{\vec{V}} \ne 0$, the lengths of the semi-major and semi-minor axes of the conic in \eqref{eq:conic_quad_form} are given by 
%  \begin{align} 
%    \label{eq:ellipse_axes}
%  %  \begin{aligned}[t]
%    \sqrt{\frac{\vec{u}^{\top}\vec{V}^{-1}\vec{u} -f}{\lambda_1}}, 
%    \sqrt{\frac{\vec{u}^{\top}\vec{V}^{-1}\vec{u} -f}{\lambda_2}}. \quad \brak{\text{ellipse}}
%    \\
%%
%       \sqrt{\frac{\vec{u}^{\top}\vec{V}^{-1}\vec{u} -f}{\lambda_1}}, 
%       \sqrt{\frac{f-\vec{u}^{\top}\vec{V}^{-1}\vec{u}}{\lambda_2}}, \quad \brak{\text{hyperbola }}
%%
%  %\end{aligned}
%  \label{eq:hyper_axes}
%\end{align} 
%\solution For \begin{align} \abs{\vec{V}} > 0, \quad \text{or, } \lambda_1 > 0, \lambda_2 > 0 
%  \end{align} and \eqref{eq:conic_simp_temp_nonparab} becomes 
%  \begin{align} 
%	  \lambda_1y_1^2 +\lambda_2y_2^2 = 
%  \vec{u}^{\top}\vec{V}^{-1}\vec{u} -f 
%	  \label{eq:hyper-pair-cond}
%  \end{align} 
%  yielding        \eqref{eq:ellipse_axes}.  Similarly, \eqref{eq:hyper_axes} can be obtained for 
%  \begin{align} 
%    \label{eq:conic_hyper_cond}
%    \abs{\vec{V}} 
%    < 0, \quad \text{or, } \lambda_1 > 0, \lambda_2 < 0 \end{align}
\end{enumerate}
