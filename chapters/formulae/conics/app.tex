\begin{enumerate}[label=\thesection.\arabic*.,ref=\thesection.\theenumi]
\item
  Let $\vec{q}$ be a point such that the ratio of its distance from a fixed point $\vec{F}$ and the distance ($d$) from a fixed line 
	\begin{align}
L: \vec{n}^{\top}\vec{x}=c 
	\end{align}
		is constant, given by 
\label{conics/30/def}
\begin{align}
\frac{\norm{\vec{q}-\vec{F}}}{d} = e    
\end{align}
The locus of $\vec{q}$ is known as a conic section. The line $L$ is known as the directrix and the point $\vec{F}$ is the focus. $e$ is defined to be 
the eccentricity of the conic.  
\begin{enumerate}
    \item For $e = 1$, the conic is a parabola
    \item For $e < 1$, the conic is an ellipse
    \item For $e > 1$, the conic is a hyperbola
\end{enumerate}

\item
The equation of  a conic with directrix $\vec{n}^{\top}\vec{x} = c$, eccentricity $e$ and focus $\vec{F}$ is given by 
\begin{align}
    \label{eq:conic_quad_form}
	\text{g}\brak{\vec{x}} = \vec{x}^{\top}\vec{V}\vec{x}+2\vec{u}^{\top}\vec{x}+f=0
    \end{align}
where     
\begin{align}
  \label{eq:conic_quad_form_v}
\vec{V} &=\norm{\vec{n}}^2\vec{I}-e^2\vec{n}\vec{n}^{\top}, 
\\
\label{eq:conic_quad_form_u}
\vec{u} &= ce^2\vec{n}-\norm{\vec{n}}^2\vec{F}, 
\\
\label{eq:conic_quad_form_f}
f &= \norm{\vec{n}}^2\norm{\vec{F}}^2-c^2e^2
    \end{align}
    \solution
  Using Definition \ref{conics/30/def} and 
			\eqref{eq:PQ-final},
for any point $\vec{x}$ on the conic,
\begin{multline}
\norm{\vec{x}-\vec{F}}^2=e^2 \frac{\brak{{\vec{n}^{\top}\vec{x} - c}}^2}{\norm{\vec{n}}^2}\label{conics/30/eq:1} \\
\implies \norm{\vec{n}}^2\brak{\vec{x}-\vec{F}}^{\top}\brak{\vec{x}-\vec{F}}=e^2\brak{\vec{n}^{\top}\vec{x} - c}^2
\\
\implies \norm{\vec{n}}^2\brak{\vec{x}^{\top}\vec{x}-2\vec{F}^{\top}\vec{x}+\norm{\vec{F}}^2}
	\\
	=e^2\brak{c^2+\brak{\vec{n}^{\top}\vec{x} }^2-2c\vec{n}^{\top}\vec{x}} \\
=e^2\brak{c^2+\brak{\vec{x}^{\top}\vec{n}\vec{n}^{\top}\vec{x} }-2c\vec{n}^{\top}\vec{x}}
\end{multline}
%
which can be expressed as \eqref{eq:conic_quad_form} after simplification.
\item
  The eccentricity, directrices and foci of \eqref{eq:conic_quad_form} are given by 
\begin{align}
  \label{eq:conic_quad_form_e} 
  e&= \sqrt{1-\frac{\lambda_1}{\lambda_2}}
\\
\label{eq:conic_quad_form_nc} 
	\begin{split}
  \vec{n}&= \sqrt{\lambda_2}\vec{p}_1,  
  \\
	c &= 
  \begin{cases}
    \frac{e\vec{u}^{\top}\vec{n} \pm \sqrt{e^2\brak{\vec{u}^{\top}\vec{n}}^2-\lambda_2\brak{e^2-1}\brak{\norm{\vec{u}}^2 - \lambda_2 f}}}{\lambda_2e\brak{e^2-1}} & e \ne 1
    \\
    \frac{\norm{\vec{u}}^2 - \lambda_2 f   }{2\vec{u}^{\top}\vec{n}} & e = 1
  \end{cases}
	\end{split}
  \\
  \label{eq:conic_quad_form_F} 
  \vec{F}  &= \frac{ce^2\vec{n}-\vec{u}}{\lambda_2}
\end{align}  
	\label{app:conic-parameters}
	\solution
	From \eqref{eq:conic_quad_form_v}, using the fact that $\vec{V}$ is symmetric with $\vec{V} = \vec{V}^{\top}$,
  \begin{align}
	  \vec{V}^{\top} \vec{V}&=\brak{\norm{\vec{n}}^2\vec{I}-e^2\vec{n}\vec{n}^{\top}}^{\top}
	  \brak{\norm{\vec{n}}^2\vec{I}-e^2\vec{n}\vec{n}^{\top}}
    \\
	  \implies \vec{V}^{2} &= \norm{\vec{n}}^4\vec{I}+e^4\vec{n}\vec{n}^{\top}\vec{n}\vec{n}^{\top}
	  -2e^2\norm{\vec{n}}^2\vec{n}\vec{n}^{\top}
    \\
	  &= \norm{\vec{n}}^4\vec{I} + e^4\norm{\vec{n}}^2\vec{n}\vec{n}^{\top}
	%  \\
	  - 2e^2\norm{\vec{n}}^2\vec{n}\vec{n}^{\top}
    \\
	  &= \norm{\vec{n}}^4\vec{I} + e^2\brak{e^2 - 2}\norm{\vec{n}}^2\vec{n}\vec{n}^{\top}
    \\
	  &= \norm{\vec{n}}^4\vec{I} + \brak{e^2 - 2}\norm{\vec{n}}^2\brak{\norm{\vec{n}}^2\vec{I}- \vec{V}}
    \end{align}
%    
which can be expressed as
\begin{align}
  \vec{V}^{2} + \brak{e^2 - 2}\norm{\vec{n}}^2\vec{V} - \brak{e^2 - 1}\norm{\vec{n}}^4\vec{I}=0
  \label{eq:conic_quad_form_e_cayley}
\end{align}
	Using the Cayley-Hamilton theorem,
	\eqref{eq:conic_quad_form_e_cayley} results in the characteristic equation, 
\begin{align}
  \lambda^{2} - \brak{2-e^2}\norm{\vec{n}}^2\lambda + \brak{1-e^2 }\norm{\vec{n}}^4=0
\end{align}
which can be expressed as
\begin{align}
\brak{\frac{\lambda}{\norm{\vec{n}}^2}}^2 - \brak{2-e^2 }\brak{\frac{\lambda}{\norm{\vec{n}}^2}} 
	+ \brak{1-e^2 } = 0
	\\
	\implies \frac{\lambda}{\norm{\vec{n}}^2} = 1-e^2, 1
  \\
	\text{or, }\lambda_2 = \norm{\vec{n}}^2, \lambda_1 = \brak{1-e^2}\lambda_2 
  \label{eq:conic_quad_form_lam_cayley}
\end{align}
From   \eqref{eq:conic_quad_form_lam_cayley}, the eccentricity of \eqref{eq:conic_quad_form} is given by 
\eqref{eq:conic_quad_form_e}.   
%
Multiplying both sides of    \eqref{eq:conic_quad_form_v} by $\vec{n}$,
\begin{align}
\vec{V} \vec{n}&=\norm{\vec{n}}^2\vec{n}-e^2\vec{n}\vec{n}^{\top}\vec{n} 
\\
&=\norm{\vec{n}}^2\brak{1-e^2}\vec{n} 
 \\
  &=\lambda_1 \vec{n} 
	\\
  \label{eq:eigevecn}
\end{align}  
from \eqref{eq:conic_quad_form_lam_cayley}.
Thus,  $\lambda_1$ is the corresponding eigenvalue for $\vec{n}$.  From       \eqref{eq:eigevecP} and \eqref{eq:eigevecn}, this implies that 
\begin{align}  
	\vec{p}_1 &= \frac{\vec{n}}{\norm{\vec{n}}} 
	\\
	\text{or, }
   \vec{n}&= \norm{\vec{n}}\vec{p}_1  = \sqrt{\lambda_2}\vec{p}_1 
\end{align}  
from   \eqref{eq:conic_quad_form_lam_cayley} .
From \eqref{eq:conic_quad_form_u} and \eqref{eq:conic_quad_form_lam_cayley},
\begin{align}
\vec{F}  &= \frac{ce^2\vec{n}-\vec{u}}{\lambda_2}
 \\
 \implies \norm{\vec{F}}^2  &= \frac{\brak{ce^2\vec{n}-\vec{u}}^{\top}\brak{ce^2\vec{n}-\vec{u}}}{\lambda_2^2}
 \\
 \implies \lambda_2^2\norm{\vec{F}}^2  &= c^2e^4\lambda_2-2ce^2\vec{u}^{\top}\vec{n}+\norm{\vec{u}}^2
 \label{eq:conic_quad_form_u_temp}
    \end{align}
    Also, \eqref{eq:conic_quad_form_f} can be expressed as
    \begin{align}
    \lambda_2\norm{\vec{F}}^2 &= f+c^2e^2
    \label{eq:conic_quad_form_f_temp}
\end{align}
From  \eqref{eq:conic_quad_form_u_temp} and     \eqref{eq:conic_quad_form_f_temp},
\begin{align}
c^2e^4\lambda_2-2ce^2\vec{u}^{\top}\vec{n}+\norm{\vec{u}}^2 = \lambda_2\brak{f+c^2e^2}
\end{align}
\begin{align}
\implies \lambda_2e^2\brak{e^2-1}c^2-2ce^2\vec{u}^{\top}\vec{n}
	+\norm{\vec{u}}^2 - \lambda_2 f = 0
\end{align}
yielding
  \eqref{eq:conic_quad_form_F}. 
\item
\eqref{eq:conic_quad_form} represents 
	\begin{enumerate}
		\item a parabola for $\mydet{\vec{V}} = 0 $,
		\item ellipse for $\mydet{\vec{V}} > 0 $ and 
		\item hyperbola for $\mydet{\vec{V}} < 0 $.
	\end{enumerate}
\solution
  From \eqref{eq:conic_quad_form_e},
\begin{align}
  \frac{\lambda_1}{\lambda_2} = 1 - e^2
\end{align}
Also, 
\begin{align}
	\mydet{\vec{V}} =   \lambda_1\lambda_2 
\end{align}
	yielding \tabref{table:det}.
\begin{table}[!h]
\centering
\resizebox{\columnwidth}{!}{%
%%%%%%%%%%%%%%%%%%%%%%%%%%%%%%%%%%%%%%%%%%%%%%%%%%%%%%%%%%%%%%%%%%%%%%
%%                                                                  %%
%%  This is the header of a LaTeX2e file exported from Gnumeric.    %%
%%                                                                  %%
%%  This file can be compiled as it stands or included in another   %%
%%  LaTeX document. The table is based on the longtable package so  %%
%%  the longtable options (headers, footers...) can be set in the   %%
%%  preamble section below (see PRAMBLE).                           %%
%%                                                                  %%
%%  To include the file in another, the following two lines must be %%
%%  in the including file:                                          %%
%%        \def\inputGnumericTable{}                                 %%
%%  at the beginning of the file and:                               %%
%%        \input{name-of-this-file.tex}                             %%
%%  where the table is to be placed. Note also that the including   %%
%%  file must use the following packages for the table to be        %%
%%  rendered correctly:                                             %%
%%    \usepackage[latin1]{inputenc}                                 %%
%%    \usepackage{color}                                            %%
%%    \usepackage{array}                                            %%
%%    \usepackage{longtable}                                        %%
%%    \usepackage{calc}                                             %%
%%    \usepackage{multirow}                                         %%
%%    \usepackage{hhline}                                           %%
%%    \usepackage{ifthen}                                           %%
%%  optionally (for landscape tables embedded in another document): %%
%%    \usepackage{lscape}                                           %%
%%                                                                  %%
%%%%%%%%%%%%%%%%%%%%%%%%%%%%%%%%%%%%%%%%%%%%%%%%%%%%%%%%%%%%%%%%%%%%%%



%%  This section checks if we are begin input into another file or  %%
%%  the file will be compiled alone. First use a macro taken from   %%
%%  the TeXbook ex 7.7 (suggestion of Han-Wen Nienhuys).            %%
\def\ifundefined#1{\expandafter\ifx\csname#1\endcsname\relax}


%%  Check for the \def token for inputed files. If it is not        %%
%%  defined, the file will be processed as a standalone and the     %%
%%  preamble will be used.                                          %%
\ifundefined{inputGnumericTable}

%%  We must be able to close or not the document at the end.        %%
	\def\gnumericTableEnd{\end{document}}


%%%%%%%%%%%%%%%%%%%%%%%%%%%%%%%%%%%%%%%%%%%%%%%%%%%%%%%%%%%%%%%%%%%%%%
%%                                                                  %%
%%  This is the PREAMBLE. Change these values to get the right      %%
%%  paper size and other niceties.                                  %%
%%                                                                  %%
%%%%%%%%%%%%%%%%%%%%%%%%%%%%%%%%%%%%%%%%%%%%%%%%%%%%%%%%%%%%%%%%%%%%%%

	\documentclass[12pt%
			  %,landscape%
                    ]{report}
       \usepackage[latin1]{inputenc}
       \usepackage{fullpage}
       \usepackage{color}
       \usepackage{array}
       \usepackage{longtable}
       \usepackage{calc}
       \usepackage{multirow}
       \usepackage{hhline}
       \usepackage{ifthen}

	\begin{document}


%%  End of the preamble for the standalone. The next section is for %%
%%  documents which are included into other LaTeX2e files.          %%
\else

%%  We are not a stand alone document. For a regular table, we will %%
%%  have no preamble and only define the closing to mean nothing.   %%
    \def\gnumericTableEnd{}

%%  If we want landscape mode in an embedded document, comment out  %%
%%  the line above and uncomment the two below. The table will      %%
%%  begin on a new page and run in landscape mode.                  %%
%       \def\gnumericTableEnd{\end{landscape}}
%       \begin{landscape}


%%  End of the else clause for this file being \input.              %%
\fi

%%%%%%%%%%%%%%%%%%%%%%%%%%%%%%%%%%%%%%%%%%%%%%%%%%%%%%%%%%%%%%%%%%%%%%
%%                                                                  %%
%%  The rest is the gnumeric table, except for the closing          %%
%%  statement. Changes below will alter the table's appearance.     %%
%%                                                                  %%
%%%%%%%%%%%%%%%%%%%%%%%%%%%%%%%%%%%%%%%%%%%%%%%%%%%%%%%%%%%%%%%%%%%%%%

\providecommand{\gnumericmathit}[1]{#1} 
%%  Uncomment the next line if you would like your numbers to be in %%
%%  italics if they are italizised in the gnumeric table.           %%
%\renewcommand{\gnumericmathit}[1]{\mathit{#1}}
\providecommand{\gnumericPB}[1]%
{\let\gnumericTemp=\\#1\let\\=\gnumericTemp\hspace{0pt}}
 \ifundefined{gnumericTableWidthDefined}
        \newlength{\gnumericTableWidth}
        \newlength{\gnumericTableWidthComplete}
        \newlength{\gnumericMultiRowLength}
        \global\def\gnumericTableWidthDefined{}
 \fi
%% The following setting protects this code from babel shorthands.  %%
 \ifthenelse{\isundefined{\languageshorthands}}{}{\languageshorthands{english}}
%%  The default table format retains the relative column widths of  %%
%%  gnumeric. They can easily be changed to c, r or l. In that case %%
%%  you may want to comment out the next line and uncomment the one %%
%%  thereafter                                                      %%
\providecommand\gnumbox{\makebox[0pt]}
%%\providecommand\gnumbox[1][]{\makebox}

%% to adjust positions in multirow situations                       %%
\setlength{\bigstrutjot}{\jot}
\setlength{\extrarowheight}{\doublerulesep}

%%  The \setlongtables command keeps column widths the same across  %%
%%  pages. Simply comment out next line for varying column widths.  %%
\setlongtables

\setlength\gnumericTableWidth{%
	80pt+%
	65pt+%
	70pt+%
	70pt+%
0pt}
\def\gumericNumCols{4}
\setlength\gnumericTableWidthComplete{\gnumericTableWidth+%
         \tabcolsep*\gumericNumCols*2+\arrayrulewidth*\gumericNumCols}
\ifthenelse{\lengthtest{\gnumericTableWidthComplete > \linewidth}}%
         {\def\gnumericScale{1*\ratio{\linewidth-%
                        \tabcolsep*\gumericNumCols*2-%
                        \arrayrulewidth*\gumericNumCols}%
{\gnumericTableWidth}}}%
{\def\gnumericScale{1}}

%%%%%%%%%%%%%%%%%%%%%%%%%%%%%%%%%%%%%%%%%%%%%%%%%%%%%%%%%%%%%%%%%%%%%%
%%                                                                  %%
%% The following are the widths of the various columns. We are      %%
%% defining them here because then they are easier to change.       %%
%% Depending on the cell formats we may use them more than once.    %%
%%                                                                  %%
%%%%%%%%%%%%%%%%%%%%%%%%%%%%%%%%%%%%%%%%%%%%%%%%%%%%%%%%%%%%%%%%%%%%%%

\ifthenelse{\isundefined{\gnumericColA}}{\newlength{\gnumericColA}}{}\settowidth{\gnumericColA}{\begin{tabular}{@{}p{90pt*\gnumericScale}@{}}x\end{tabular}}
\ifthenelse{\isundefined{\gnumericColB}}{\newlength{\gnumericColB}}{}\settowidth{\gnumericColB}{\begin{tabular}{@{}p{65pt*\gnumericScale}@{}}x\end{tabular}}
\ifthenelse{\isundefined{\gnumericColC}}{\newlength{\gnumericColC}}{}\settowidth{\gnumericColC}{\begin{tabular}{@{}p{70pt*\gnumericScale}@{}}x\end{tabular}}
\ifthenelse{\isundefined{\gnumericColD}}{\newlength{\gnumericColD}}{}\settowidth{\gnumericColD}{\begin{tabular}{@{}p{75pt*\gnumericScale}@{}}x\end{tabular}}

%\begin{longtable}[c]{%
\begin{tabular}[c]{%
	b{\gnumericColA}%
	b{\gnumericColB}%
	b{\gnumericColC}%
	b{\gnumericColD}%
	}

%%%%%%%%%%%%%%%%%%%%%%%%%%%%%%%%%%%%%%%%%%%%%%%%%%%%%%%%%%%%%%%%%%%%%%
%%  The longtable options. (Caption, headers... see Goosens, p.124) %%
%	\caption{The Table Caption.}             \\	%
% \hline	% Across the top of the table.
%%  The rest of these options are table rows which are placed on    %%
%%  the first, last or every page. Use \multicolumn if you want.    %%

%%  Header for the first page.                                      %%
%	\multicolumn{4}{c}{The First Header} \\ \hline 
%	\multicolumn{1}{c}{colTag}	%Column 1
%	&\multicolumn{1}{c}{colTag}	%Column 2
%	&\multicolumn{1}{c}{colTag}	%Column 3
%	&\multicolumn{1}{c}{colTag}	\\ \hline %Last column
%	\endfirsthead

%%  The running header definition.                                  %%
%	\hline
%	\multicolumn{4}{l}{\ldots\small\slshape continued} \\ \hline
%	\multicolumn{1}{c}{colTag}	%Column 1
%	&\multicolumn{1}{c}{colTag}	%Column 2
%	&\multicolumn{1}{c}{colTag}	%Column 3
%	&\multicolumn{1}{c}{colTag}	\\ \hline %Last column
%	\endhead

%%  The running footer definition.                                  %%
%	\hline
%	\multicolumn{4}{r}{\small\slshape continued\ldots} \\
%	\endfoot

%%  The ending footer definition.                                   %%
%	\multicolumn{4}{c}{That's all folks} \\ \hline 
%	\endlastfoot
%%%%%%%%%%%%%%%%%%%%%%%%%%%%%%%%%%%%%%%%%%%%%%%%%%%%%%%%%%%%%%%%%%%%%%

\hhline{|-|-|-|-}
	 \multicolumn{1}{|p{\gnumericColA}|}%
	{\gnumericPB{\centering}\textbf{Eccentricity}}
	&\multicolumn{1}{p{\gnumericColB}|}%
	{\gnumericPB{\centering}\textbf{Conic}}
	&\multicolumn{1}{p{\gnumericColC}|}%
	{\gnumericPB{\centering}\textbf{Eigenvalue}}
	&\multicolumn{1}{p{\gnumericColD}|}%
	{\gnumericPB{\centering}\textbf{Determinant}}
\\
\hhline{|----|}
	 \multicolumn{1}{|p{\gnumericColA}|}%
	{\gnumericPB{\centering}$e = 1$}
	&\multicolumn{1}{p{\gnumericColB}|}%
	{\gnumericPB{\centering}Parabola}
	&\multicolumn{1}{p{\gnumericColC}|}%
	{\gnumericPB{\centering}$\lambda_1 =0$}
	&\multicolumn{1}{p{\gnumericColD}|}%
	{\gnumericPB{\centering}$\mydet{\vec{V}} = 0$}
\\
\hhline{|----|}
	 \multicolumn{1}{|p{\gnumericColA}|}%
	{\gnumericPB{\centering}$e < 1$}
	&\multicolumn{1}{p{\gnumericColB}|}%
	{\gnumericPB{\centering}Ellipse}
	&\multicolumn{1}{p{\gnumericColC}|}%
	{\gnumericPB{\centering}$\lambda_1 > 0, \lambda_2 > 0$}
	&\multicolumn{1}{p{\gnumericColD}|}%
	{\gnumericPB{\centering}$\mydet{\vec{V}} > 0$}
\\
\hhline{|----|}
	 \multicolumn{1}{|p{\gnumericColA}|}%
	{\gnumericPB{\centering}$e > 1$}
	&\multicolumn{1}{p{\gnumericColB}|}%
	{\gnumericPB{\centering}Hyperbola}
	&\multicolumn{1}{p{\gnumericColC}|}%
	{\gnumericPB{\centering}$ \lambda_1 < 0, \lambda_2 > 0$}
	&\multicolumn{1}{p{\gnumericColD}|}%
		{\gnumericPB{\centering}$\mydet{\vec{V}} < 0$}
\\
\hhline{|-|-|-|-|}
\end{tabular}
%\end{longtable}

\ifthenelse{\isundefined{\languageshorthands}}{}{\languageshorthands{\languagename}}
\gnumericTableEnd

	}
	\caption{}
\label{table:det}
\end{table}
\item
  \label{app:parab}
	Using 
\eqref{eq:conic_affine},
\eqref{eq:conic_quad_form} can be expressed as
\begin{align}
\brak{\vec{P}\vec{y}+\vec{c}}^{\top}\vec{V}\brak{\vec{P}\vec{y}+\vec{c}}+2\vec{u}^{\top}\brak{\vec{P}\vec{y}+\vec{c}}+ f
	= 0, 
\end{align}
yielding 
\begin{align}
\vec{y}^{\top}\vec{P}^{\top}\vec{V}\vec{P}\vec{y}+2\brak{\vec{V}\vec{c}+\vec{u}}^{\top}\vec{P}\vec{y}
+  \vec{c}^{\top}\vec{V}\vec{c} + 2\vec{u}^{\top}\vec{c} + f= 0
\label{eq:conic_simp_one}
\end{align}
%
From \eqref{eq:conic_simp_one} and \eqref{eq:conic_parmas_eig_def},
\begin{align}
\vec{y}^{\top}\vec{D}\vec{y}+2\brak{\vec{V}\vec{c}+\vec{u}}^{\top}\vec{P}\vec{y}
+  \vec{c}^{\top}\brak{\vec{V}\vec{c} + \vec{u}}+ \vec{u}^{\top}\vec{c} + f= 0
\label{eq:conic_simp}
\end{align}
When $\vec{V}^{-1}$ exists, choosing
\begin{align}
%\begin{split}
\vec{V}\vec{c}+\vec{u} &= \vec{0}, \quad \text{or}, \vec{c} = -\vec{V}^{-1}\vec{u},
\label{eq:conic_parmas_c_def}
\end{align}
%
%%From \eqref{eq:conic_parmas_k_def} and 
%%
and substituting \eqref{eq:conic_parmas_c_def}
in \eqref{eq:conic_simp}
yields \eqref{eq:conic_simp_temp_nonparab}. 
  %See Appendix \ref{app:parab}.
When $\abs{\vec{V}} = 0, \lambda_1 = 0$ and 
\begin{align}
\vec{V}\vec{p}_1 = 0, 
\vec{V}\vec{p}_2 = \lambda_2\vec{p}_2.
\label{eq:conic_parab_eig_prop} 
\end{align}
Substituting \eqref{eq:eig_matrix}
in \eqref{eq:conic_simp},
\begin{multline}
	\vec{y}^{\top}\vec{D}\vec{y}+2\brak{\vec{c}^{\top}\vec{V}+\vec{u}^{\top}}\myvec{\vec{p}_1 & \vec{p}_2}\vec{y}
	\\
	+  \vec{c}^{\top}\brak{\vec{V}\vec{c} + \vec{u}}+ \vec{u}^{\top}\vec{c} + f= 0
\\
\implies \vec{y}^{\top}\vec{D}\vec{y}
+2\myvec{\brak{\vec{c}^{\top}\vec{V}+\vec{u}^{\top}}\vec{p}_1  \brak{\vec{c}^{\top}\vec{V}+\vec{u}^{\top}}\vec{p}_2}\vec{y}
\\
	+  \vec{c}^{\top}\brak{\vec{V}\vec{c} + \vec{u}}+ \vec{u}^{\top}\vec{c} + f= 0
\\
\implies \vec{y}^{\top}\vec{D}\vec{y}
+2\myvec{\vec{u}^{\top}\vec{p}_1 & \brak{\lambda_2\vec{c}^{\top}+\vec{u}^{\top}}\vec{p}_2}\vec{y}
\\
	+  \vec{c}^{\top}\brak{\vec{V}\vec{c} + \vec{u}}+ \vec{u}^{\top}\vec{c} + f= 0
\end{multline}
upon substituting from 
 \eqref{eq:conic_parab_eig_prop}, yielding
\begin{multline}
\lambda_2y_2^2+2\brak{\vec{u}^{\top}\vec{p}_1}y_1+  2y_2\brak{\lambda_2\vec{c}+\vec{u}}^{\top}\vec{p}_2
\\
	+  \vec{c}^{\top}\brak{\vec{V}\vec{c} + \vec{u}}+ \vec{u}^{\top}\vec{c} + f= 0
\label{eq:conic_parab_foc_len_temp} 
\end{multline}
%which is the equation of a parabola. 
Thus, \eqref{eq:conic_parab_foc_len_temp} 
can be expressed as \eqref{eq:conic_simp_temp_parab} by choosing
\begin{align}
%\label{eq:eta}
\eta = 2\vec{u}^{\top}\vec{p}_1
\end{align}
%Choosing 
%\begin{align}
%\vec{u} + \lambda_2\vec{c} = 0,
%\vec{c}^{\top}\brak{\vec{V}\vec{c} + \vec{u}}+ \vec{u}^{\top}\vec{c} + f = 0,
%\end{align}
% the above equation becomes
%\begin{align}
%y_2^2= -\frac{2\vec{u}^{\top}\vec{p}_1}{ \lambda_2} \brak{y_1
%+  \frac{\vec{u}^{\top}\vec{V}\vec{u} - 2\lambda_2\vec{u}^{\top}\vec{u} + f\lambda_2^2}{2\vec{u}^{\top}\vec{p}_1\lambda_2^2}}
%\\
%or \eta = 2\vec{u}^{\top}\vec{p}_1
%%\label{eq:conic_simp_parab_new}
%\end{align}
and $\vec{c}$ in \eqref{eq:conic_simp} such that
\begin{align}
\label{eq:conic_parab_one}
2\vec{P}^{\top}\brak{\vec{V}\vec{c}+\vec{u}} &= \eta\myvec{1\\0}
\\
\vec{c}^{\top}\brak{\vec{V}\vec{c} + \vec{u}}+ \vec{u}^{\top}\vec{c} + f&= 0
\label{eq:conic_parab_two}
\end{align}
%we obtain  \eqref{eq:conic_simp_temp_parab}.
	\item
		The center/vertex of a conic section are given by
  \begin{align}
    \label{eq:conic_nonparab_c}
	    \vec{c} &= - \vec{V}^{-1}\vec{u}  & \mydet{\vec{V}} \ne 0
    \\
	    \myvec{ \vec{u}^{\top}+\frac{\eta}{2}\vec{p}_1^{\top} \\ \vec{v}}\vec{c} &= \myvec{-f \\ \frac{\eta}{2}\vec{p}_1-\vec{u}}  
& \mydet{\vec{V}} = 0
    \label{eq:conic_parab_c}
    \end{align}	
\solution	
$\because
\vec{P}^{\top}\vec{P} = \vec{I}$,
multiplying \eqref{eq:conic_parab_one} by $\vec{P}$ yields
\begin{align}
\label{eq:conic_parab_one_eig}
	\brak{\vec{V}\vec{c}+\vec{u}} &= \frac{\eta}{2}\vec{p}_1,
\end{align}
which, upon substituting in \eqref{eq:conic_parab_two}
results in 
\begin{align}
\frac{\eta}{2}\vec{c}^{\top}\vec{p}_1 + \vec{u}^{\top}\vec{c} + f&= 0
\label{eq:conic_parab_two_eig}
\end{align}
\eqref{eq:conic_parab_one_eig} and \eqref{eq:conic_parab_two_eig} can be clubbed together to obtain \eqref{eq:conic_parab_c}.
\iffalse
			In 
			\eqref{eq:conic_affine}, substituting $\vec{y} = \vec{0}$, the center/vertex for the quadratic form is obtained as
    \begin{align}
	    \vec{x} = \vec{c}, 
    \end{align}
			where $\vec{c}$ is derived as 
    \eqref{eq:conic_nonparab_c}
    and 
    \eqref{eq:conic_parab_c}
in Appendix  \ref{app:parab}.
\fi
\iffalse
	  \item
		For the standard conic, 
				\begin{align}
					\label{eq:std-prm-P}
					\vec{P} &= \vec{I}
					\\
					\vec{u} &= 
				\begin{cases}
				0 & e \ne 1
       \\
				\frac{\eta}{2} \vec{e}_1 & e = 1
				\end{cases}
				\label{eq:std-prm-u}
				\\
				\lambda_1 &  
					\begin{cases}
						=0 & e = 1
						\\
						\ne 0 & e \ne 1
					\end{cases}
				\label{eq:std-prm-lam1}
				\end{align}
				where 
				\begin{align}
					\vec{I} = \myvec{\vec{e}_1 & \vec{e}_2}
				\end{align}
				is the identity matrix.
	  
    \item\leavevmode
		\begin{enumerate}
			\item The directrices for the  standard conic are given by 
				\begin{align}
					\label{eq:dx-ell-hyp}
					\vec{e}_1^{\top}\vec{y} &=  
					%\pm\sqrt{\abs{\frac{f_0\lambda_2}{\lambda_1\brak{\lambda_2-\lambda_1}}}} & e \ne 1
					\pm \frac{1}{e}\sqrt{\frac{\abs{f_0}}{\lambda_2\brak{1-e^2}}} & e \ne 1
					\\
					\vec{e}_1^{\top}\vec{y} &= \frac{\eta}{2\lambda_2} & e = 1
					\label{eq:dx-parab}
				\end{align}
    \item The foci of the standard ellipse and hyperbola are given by 
				\begin{align}
					\label{eq:F-ell-hyp-parab}
					\vec{F} 
=
					\begin{cases}
						\pm e\sqrt{\frac{\abs{f_0}}{\lambda_2\brak{1-e^2}}}\vec{e}_1 & e \ne 1
					%	\pm \sqrt{\abs{\frac{f_0}{\lambda_1}\brak{1 - \frac{\lambda_1}{\lambda_2}}}}\vec{e}_1 & e \ne 1
						\\
						 -\frac{\eta}{4\lambda_2}\vec{e}_1 & e = 1
					\end{cases}
				\end{align}
	
		\end{enumerate}
	%	where, without loss of generality, $f_0 < 0$ for the hyperbola.
    
	\begin{proof}%\leavevmode
  \label{app:foc-dir}
%  \input{appendix.tex}
		\begin{enumerate}
			\item For the standard hyperbola/ellipse in \eqref{eq:conic_simp_temp_nonparab}, from 
					\eqref{eq:std-prm-P},
\eqref{eq:conic_quad_form_nc}
and 
					\eqref{eq:std-prm-u},
				\begin{align}
\label{eq:n-ell-hyp}
					\vec{n} &= \sqrt{\frac{\lambda_2}{f_0}} \vec{e}_1 
					\\
					c &= 
					%\pm \frac{\sqrt{-\lambda_2\brak{e^2-1}\brak{\lambda_2 f_0}}}{\lambda_2e\brak{e^2-1}}
					\pm \frac{\sqrt{-\frac{\lambda_2}{f_0}\brak{e^2-1}\brak{\frac{\lambda_2}{ f_0}}}}{\frac{\lambda_2}{f_0}e\brak{e^2-1}}
					\\
					&=\pm \frac{1}{e\sqrt{1-e^2}}
%					\\
%					&=\pm\sqrt{\abs{\frac{f_0}{\brak{1 - \frac{\lambda_1}{\lambda_2}}\frac{\lambda_1}{\lambda_2}}}}
\label{eq:c-ell-hyp}
				\end{align}
				yielding 
					\eqref{eq:dx-ell-hyp} upon substituting from 
\eqref{eq:conic_quad_form_e} and simplifying.
For the standard parabola in \eqref{eq:conic_simp_temp_parab},  from 
					\eqref{eq:std-prm-P},
\eqref{eq:conic_quad_form_nc}
and 
					\eqref{eq:std-prm-u}, noting that $f = 0$,

				\begin{align}
\label{eq:n-parab}
					\vec{n} &= \sqrt{\lambda_2} \vec{e}_1 
					\\
					c &=
	\frac{\norm{\frac{\eta}{2} \vec{e}_1}^2   }{2\vec{\brak{\frac{\eta}{2}} \brak{\vec{e}_1}^{\top}\vec{n}}} 
\\
					\\
					&= \frac{\eta}{4\sqrt{\lambda_2}}
\label{eq:c-parab}
				\end{align}
				yielding 
					\eqref{eq:dx-parab}.

				\item 	For the standard ellipse/hyperbola, substituting from
\eqref{eq:c-ell-hyp},
\eqref{eq:n-ell-hyp},
\eqref{eq:std-prm-u}
and \eqref{eq:conic_quad_form_e}
in \eqref{eq:conic_quad_form_F},
				\begin{align}
					\vec{F} &= \pm \frac{\brak{\frac{1}{e\sqrt{1-e^2}}}\brak{e^2}\sqrt{\frac{\lambda_2}{f_0}}\vec{e}_1}{\frac{\lambda_2}{f_0}}
					%\pm\sqrt{\abs{\frac{f_0}{\brak{1 - \frac{\lambda_1}{\lambda_2}}\frac{\lambda_1}{\lambda_2}}}}
					%\brak{1 - \frac{\lambda_1}{\lambda_2}}\frac{\sqrt{\lambda_2}}{\lambda_2}\vec{e}_1
 			\end{align}
			yielding
					\eqref{eq:F-ell-hyp-parab}
					after simplification.
					For the standard parabola, substituting from 
\eqref{eq:c-parab},
\eqref{eq:n-parab},
\eqref{eq:std-prm-u}
and \eqref{eq:conic_quad_form_e}
in \eqref{eq:conic_quad_form_F},			
				\begin{align}
	\vec{F}  &= \frac{\brak{\frac{\eta}{4\sqrt{\lambda_2}}}\sqrt{\lambda_2}\vec{e}_1-\vec{\frac{\eta}{2} \vec{e}_1}}{\lambda_2}
\\
				\end{align}
				yielding 
					\eqref{eq:F-ell-hyp-parab} after simplification.

		\end{enumerate}
%		See Appendix \ref{app:foc-dir}.
	\end{proof}
%
    \item The equation of the minor and major  axes for the ellipse/hyperbola are respectively given by 
  \begin{align}
\vec{p}_i^{\top}\brak{\vec{x}-\vec{c}} = 0, i = 1,2
	  \label{eq:major-minor-axis-quad}
  \end{align}
  The axis of symmetry for the parabola is also given by 
	  \eqref{eq:major-minor-axis-quad}.

		\begin{proof}
From		\eqref{corr:axis}, the major/symmetry axis for the hyperbola/ellipse/parabola can be expressed using 
	\eqref{eq:conic_affine}
 as
  \begin{align}
	  \vec{e}_2^{\top}
		  \vec{P}^{\top}\brak{\vec{x}-\vec{c}} &= 0
		  \\
	  \implies 		  \brak{\vec{P}\vec{e}_2}^{\top}\brak{\vec{x}-\vec{c}} &= 0
  \end{align}
yielding	  \eqref{eq:major-minor-axis-quad}, and the proof for the minor axis is similar.
		\end{proof}
	\item
			\label{corr:center}
			The center of the standard ellipse/hyperbola, defined to be the mid point of the line joining the foci, is the origin.
	
	\item
		\label{corr:axis}
			The principal (major) axis of the standard ellipse/hyperbola, defined to be the line joining the two foci   is the $x$-axis.  
	
	\begin{proof}
		From 	\eqref{eq:F-ell-hyp-parab}, it is obvious that the line joining the foci passes through the origin.  Also, the direction vector of this line is $\vec{e}_1$.  Thus, the principal axis is the $x$-axis. 
	\end{proof}
	\item
		\label{corr:minor-axis}
			The minor axis of the standard ellipse/hyperbola, defined to be the line orthogonal to the $x$-axis is the $y$-axis. 
	


	\item
			The axis of symmetry of the standard parabola, defined to be the line perpendicular to the directrix and passing through the focus,  is the $x$- axis.
	
	\begin{proof}
	From \eqref{eq:n-parab} and 	
					\eqref{eq:F-ell-hyp-parab}, 
					the axis of the parabola  can be expressed 
     as 
		\begin{align}
			\vec{e}_2^{\top}\brak{\vec{y}  
			+\frac{\eta}{4\lambda_2}\vec{e}_1} &= 0
			\\
			\implies \vec{e}_2^{\top}\vec{y} &= 0
					\label{eq:axis-std-parab}, 
		\end{align}
		which is the equation of the $x$-axis.
	\end{proof}


	\item
			\label{corr:center-parab}
 The point where the parabola intersects its axis of symmetry is called the vertex. For the standard parabola, the vertex is the origin.
	
	\begin{proof}
					\eqref{eq:axis-std-parab} can be expressed as 
    \begin{align}
			\vec{y}= \alpha \vec{e}_1. 
					\label{eq:axis-std-parab-dir} 
    \end{align}
					Substituting \eqref{eq:axis-std-parab-dir} in 
    \eqref{eq:conic_simp_temp_parab}, 
    \begin{align}
	     \alpha^2 \vec{e}_1^{\top}\vec{D} \vec{e}_1 &=  -\eta\alpha \vec{e}_1^{\top} \vec{e}_1   
	     \\
	     \implies \alpha &=0, \text{ or, } \vec{y} = \vec{0}.
    %\end{aligned}
    \end{align}
	\end{proof}
	\item
			\label{corr:foclen}
	 The {\em focal length} of the standard parabola, , defined to be the distance between the vertex and the focus, measured along the axis of symmetry, is $\abs{\frac{\eta}{4 \lambda_2}}$
	 \fi

 \item
	 \label{prop:chord}
  Substituting \eqref{eq:conic_tangent}
in \eqref{eq:conic_quad_form}, 
\begin{align}
\brak{\vec{h} + \kappa \vec{m}}^{\top}\vec{V}\brak{\vec{h} + \kappa \vec{m}}  + 2 \vec{u}^{\top}\brak{\vec{h} + \kappa \vec{m}}+f &= 0
\\
\implies \kappa^2\vec{m}^{\top}\vec{V}\vec{m} + 2 \kappa\vec{m}^{\top}\brak{\vec{V}\vec{h}+\vec{u}} 
+ \vec{h}^{\top}\vec{V}\vec{h} + 2\vec{u}^{\top}\vec{h} +f &= 0
	\\
	\text{or, }
\kappa^2\vec{m}^{\top}\vec{V}\vec{m} + 2 \kappa\vec{m}^{\top}\brak{\vec{V}\vec{h}+\vec{u}} 
	+ \text{g}\brak{\vec{h}} &=0
	%^{\top}\vec{V}\vec{h} + 2\vec{u}^{\top}\vec{h} +f &= 0
\label{eq:conic_intercept}
\end{align}
for g defined in \eqref{eq:conic_quad_form}.
Solving the above quadratic in \eqref{eq:conic_intercept}
yields \eqref{eq:tangent_roots}.

	\item
		The length of the chord in 
\eqref{eq:conic_tangent}
is given by 
\begin{align}
 \frac{2\sqrt{
\sbrak{
\vec{m}^{\top}\brak{\vec{V}\vec{h}+\vec{u}}
}^2
-
\brak
{
\vec{h}^{\top}\vec{V}\vec{h} + 2\vec{u}^{\top}\vec{h} +f
}
\brak{\vec{m}^{\top}\vec{V}\vec{m}}
}
}
{
\vec{m}^{\top}\vec{V}\vec{m}
}\norm{\vec{m}}
\label{eq:chord-len}
  \end{align}
	
\begin{proof}
The distance between the points in 
	\eqref{eq:chord-pts}
is given by 
\begin{align}
	\norm{\vec{x}_1-\vec{x}_2} =  \abs{\kappa_1-\kappa_2} \norm{\vec{m}}
\label{eq:conic_tangent_pts_dist}
\end{align}
Substituing $\kappa_i$ from 
\eqref{eq:tangent_roots} in
\eqref{eq:conic_tangent_pts_dist}
yields
	\eqref{eq:chord-len}.
\end{proof}
	\item
 The affine transform for the conic section, preserves the norm.  This implies that the length of any chord of a conic
	is invariant to translation and/or rotation.
	
	\begin{proof}
	Let 
%From \eqref{eq:conic_affine}, 
\begin{align}
\vec{x}_i = \vec{P}\vec{y}_i+\vec{c} 
\label{eq:conic_affine_pts}
\end{align}
be any two points on the conic.  Then the distance between the points is given by 
\begin{align}
	\norm{\vec{x}_1-\vec{x}_2 } &= \norm{\vec{P}\brak{	\vec{y}_1 -\vec{y}_2 }}
\end{align}
which can be expressed as 
\begin{align}
	\norm{\vec{x}_1-\vec{x}_2 }^2 &= 		\brak{\vec{y}_1 -\vec{y}_2 }^{\top}\vec{P}^{\top}\vec{P}\brak{\vec{y}_1 -\vec{y}_2 }
	\\
	&= 		\norm{\vec{y}_1 -\vec{y}_2 }^2
\label{eq:conic_affine_norm_preserve}
\end{align}
since 
\begin{align}
	\vec{P}^{\top}\vec{P} = \vec{I}
\end{align}
	\end{proof}
\item 
%	See Appendix \ref{app:major}
		\label{app:major}
		Since the major axis passes through the origin, 
  \begin{align}
	  \vec{q} =			\vec{0} 
\end{align}  
Further, from Corollary  
		\eqref{corr:axis},
  \begin{align}
  \vec{m}&= \vec{e}_2,  
\end{align} and
from 
    \eqref{eq:conic_simp_temp_nonparab},
  \begin{align}
	  \vec{V} =     \frac{\vec{D} }{f_0}, 
	   \vec{u} = 0, 
	   f = -1
	    \label{eq:latus_rectum_ellipse_param}
\end{align}  
Substituting the above in
\eqref{eq:chord-len}, 
\begin{align}
 \frac{2\sqrt{
\vec{e}_1^{\top}\frac{\vec{D}}{f_0}\vec{e}_1
}
}
{
\vec{e}_1^{\top}\frac{\vec{D}}{f_0}\vec{e}_1
}\norm{\vec{e}_1}
  \end{align}
  yielding 
\eqref{eq:chord-len-major}.
Similarly, for the minor axis, the only different parameter is 
  \begin{align}
  \vec{m}&= \vec{e}_2,  
\end{align} 
Substituting the above in
\eqref{eq:chord-len}, 
\begin{align}
 \frac{2\sqrt{
\vec{e}_2^{\top}\frac{\vec{D}}{f_0}\vec{e}_2
}
}
{
\vec{e}_2^{\top}\frac{\vec{D}}{f_0}\vec{e}_2
}\norm{\vec{e}_2}
  \end{align}
  yielding 
\eqref{eq:chord-len-minor}.
	\item 
%			See Appendix \ref{app:latus}.
		%\section{}
		\label{app:latus}
			The latus rectum is perpendicular to the major axis for the standard conic.  Hence, from Corollary  
		\eqref{corr:axis},
  \begin{align}
  \vec{m}&= \vec{e}_2,  
\end{align}  
Since it passes through the focus, from 
					\eqref{eq:F-ell-hyp-parab}
  \begin{align}
	  \vec{q} =			\vec{F} 
=
					 \pm e\sqrt{\frac{f_0}{\lambda_2\brak{1-e^2}}} \vec{e }_1
%					 \frac{e}{\sqrt{f_0\lambda_2\brak{1-e^2}}}\vec{e }_1
\end{align}  
for the standard hyperbola/ellipse.  Also, 
from 
    \eqref{eq:conic_simp_temp_nonparab},
  \begin{align}
	  \vec{V} =     \frac{\vec{D} }{f_0}, 
	   \vec{u} = 0, 
	   f = -1
	    \label{eq:latus_rectum_ellipse_param-new}
\end{align}  
Substituting the above in
\eqref{eq:chord-len}, 
\begin{align}
 \frac{2\sqrt{
\sbrak{
\vec{e}_2^{\top}\brak{\frac{\vec{D}}{f_0} e\sqrt{\frac{f_0}{\lambda_2\brak{1-e^2}}} \vec{e }_1}
}^2
-
\brak
{
 e\sqrt{\frac{f_0}{\lambda_2\brak{1-e^2}}} \vec{e }_1^{\top}\frac{\vec{D}}{f_0} e\sqrt{\frac{f_0}{\lambda_2\brak{1-e^2}}} \vec{e }_1 -1 
}
\brak{\vec{e}_2^{\top}\frac{\vec{D}}{f_0}\vec{e}_2}
}
}
{
\vec{e}_2^{\top}\frac{\vec{D}}{f_0}\vec{e}_2
}\norm{\vec{e}_2}
\label{eq:chord-len-sub-ell}
  \end{align}
  Since 
  \begin{align}
\vec{e}_2^{\top}\vec{D}\vec{e}_1 = 0, 
%\vec{e}_2^{\top}\vec{e}_2 = 0,
\vec{e}_1^{\top}\vec{D}\vec{e}_1 = \lambda_1,
\vec{e}_1^{\top}\vec{e}_1 = 1,
	  \norm{\vec{e}_2} = 1,
\vec{e}_2^{\top}\vec{D}\vec{e}_2 = \lambda_2,
  \end{align}
\eqref{eq:chord-len-sub-ell} can be expressed as 
  \begin{align}
	&		\frac{2\sqrt{\brak{1-\frac{\lambda_1e^2}{{\lambda_2\brak{1-e^2}}}}\brak{\frac{\lambda_2}{f_0}}}}
{
	\frac{\lambda_2}{f_0}
	} 	
	\\
	&=		2\frac{\sqrt{
		f_0\lambda_1}}{\lambda_2}
 & \brak{ \because e^2 = 1-\frac{\lambda_1}{\lambda_2}}
		   \end{align}
For the standard parabola, the parameters in 
\eqref{eq:chord-len} are
\begin{align}  
	\vec{q} =\vec{F} =  -\frac{\eta}{4\lambda_2}\vec{e}_1, \vec{m} = \vec{e}_1, \vec{V} = \vec{D},
	\vec{u} = \frac{\eta}{2}\vec{e}_1^{\top}, f = 0
\end{align}  

Substituting the above in
\eqref{eq:chord-len}, 
%			from \eqref{eq:conic_simp_temp_nonparab},  
%					from \eqref{eq:F-ell-hyp-parab}
%and 						 \\
the length of the latus rectum  can be expressed as
{\footnotesize
\begin{align}
 \frac{2\sqrt{
\sbrak{
\vec{e}_2^{\top}\brak{\vec{D}\brak{-\frac{\eta}{4\lambda_2}\vec{e}_1}+\frac{\eta}{2}\vec{e}_1}
}^2
-
\brak
{
\brak{-\frac{\eta}{4\lambda_2}\vec{e}_1}^{\top}\vec{D}\brak{-\frac{\eta}{4\lambda_2}\vec{e}_1} + 2\frac{\eta}{2}\vec{e}_1^{\top}\brak{-\frac{\eta}{4\lambda_2}\vec{e}_1} 
}
\brak{\vec{e}_2^{\top}\vec{D}\vec{e}_2}
}
}
{
\vec{e}_2^{\top}\vec{D}\vec{e}_2
}\norm{\vec{e}_2}
\label{eq:chord-len-sub}
  \end{align}
  }
  Since 
  \begin{align}
\vec{e}_2^{\top}\vec{D}\vec{e}_1 = 0, 
\vec{e}_2^{\top}\vec{e}_2 = 0,
\vec{e}_1^{\top}\vec{D}\vec{e}_1 = 0,
\vec{e}_1^{\top}\vec{e}_1 = 1,
	  \norm{\vec{e}_1} = 1,
\vec{e}_2^{\top}\vec{D}\vec{e}_2 = \lambda_2,
  \end{align}
\eqref{eq:chord-len-sub} can be expressed as 
  \begin{align}
	  2 \frac{\sqrt{\frac{\eta^2}{4\lambda_2}\lambda_2}}{\lambda_2}
	  = \frac{\eta}{\lambda_2}
  \end{align}
\iffalse
\item
Using the affine transformation in
\eqref{eq:conic_affine},
	the conic in     \eqref{eq:conic_quad_form} can be expressed in standard form 
	%(centre/vertex at the origin, major axis - $x$ axis)
	as
  \begin{align}
    %\begin{aligned}
    \label{eq:conic_simp_temp_nonparab}
	    \vec{y}^{\top}\brak{\frac{\vec{D}}{f_0}}\vec{y} &= 1   &  \abs{\vec{V}} &\ne 0
    \\
	    \vec{y}^{\top}\vec{D}\vec{y} &=  -\eta\vec{e}_1^{\top}\vec{y}   & \abs{\vec{V}} &= 0
    \label{eq:conic_simp_temp_parab}
    %\end{aligned}
    \end{align}
    where
  \begin{align}
      %\begin{split}
      \label{eq:f0}
	  f_0 &=\vec{u}^{\top}\vec{V}^{-1}\vec{u} -f \ne 0
	  \\
      \label{eq:eta}
       \eta &=2\vec{u}^{\top}\vec{p}_1
       \\
       \vec{e}_1 &=\myvec{1 \\ 0}
      \end{align}
     
    
\begin{proof}
  \label{app:parab}
	Using 
\eqref{eq:conic_affine}
%such that 
\eqref{eq:conic_quad_form} can be expressed as

%\item  
%Substituting \eqref{eq:conic_affine} in \eqref{eq:conic_quad_form}
\begin{align}
\brak{\vec{P}\vec{y}+\vec{c}}^{\top}\vec{V}\brak{\vec{P}\vec{y}+\vec{c}}+2\vec{u}^{\top}\brak{\vec{P}\vec{y}+\vec{c}}+ f
	= 0, 
\end{align}
yielding 
\begin{align}
\vec{y}^{\top}\vec{P}^{\top}\vec{V}\vec{P}\vec{y}+2\brak{\vec{V}\vec{c}+\vec{u}}^{\top}\vec{P}\vec{y}
+  \vec{c}^{\top}\vec{V}\vec{c} + 2\vec{u}^{\top}\vec{c} + f= 0
\label{eq:conic_simp_one}
\end{align}
%
From \eqref{eq:conic_simp_one} and \eqref{eq:conic_parmas_eig_def},
\begin{align}
\vec{y}^{\top}\vec{D}\vec{y}+2\brak{\vec{V}\vec{c}+\vec{u}}^{\top}\vec{P}\vec{y}
+  \vec{c}^{\top}\brak{\vec{V}\vec{c} + \vec{u}}+ \vec{u}^{\top}\vec{c} + f= 0
\label{eq:conic_simp}
\end{align}
When $\vec{V}^{-1}$ exists, choosing
\begin{align}
%\begin{split}
\vec{V}\vec{c}+\vec{u} &= \vec{0}, \quad \text{or}, \vec{c} = -\vec{V}^{-1}\vec{u},
\label{eq:conic_parmas_c_def}
\end{align}
%
%%From \eqref{eq:conic_parmas_k_def} and 
%%
and substituting \eqref{eq:conic_parmas_c_def}
in \eqref{eq:conic_simp}
yields \eqref{eq:conic_simp_temp_nonparab}. 
  %See Appendix \ref{app:parab}.
When $\abs{\vec{V}} = 0, \lambda_1 = 0$ and 
\begin{align}
\vec{V}\vec{p}_1 = 0, 
\vec{V}\vec{p}_2 = \lambda_2\vec{p}_2.
\label{eq:conic_parab_eig_prop} 
\end{align}
where $\vec{p}_1,\vec{p}_2$ are the eigenvectors of $\vec{V}$ such that  \eqref{eq:conic_parmas_eig_def}
%
\begin{align}
\vec{P} = \myvec{\vec{p}_1 & \vec{p}_2},
\label{eq:eig_matrix}
\end{align}
Substituting \eqref{eq:eig_matrix}
in \eqref{eq:conic_simp},
\begin{align}
	\vec{y}^{\top}\vec{D}\vec{y}+2\brak{\vec{c}^{\top}\vec{V}+\vec{u}^{\top}}\myvec{\vec{p}_1 & \vec{p}_2}\vec{y}
	+  \vec{c}^{\top}\brak{\vec{V}\vec{c} + \vec{u}}+ \vec{u}^{\top}\vec{c} + f&= 0
\\
\implies \vec{y}^{\top}\vec{D}\vec{y}
+2\myvec{\brak{\vec{c}^{\top}\vec{V}+\vec{u}^{\top}}\vec{p}_1  \brak{\vec{c}^{\top}\vec{V}+\vec{u}^{\top}}\vec{p}_2}\vec{y}
	+  \vec{c}^{\top}\brak{\vec{V}\vec{c} + \vec{u}}+ \vec{u}^{\top}\vec{c} + f&= 0
\\
\implies \vec{y}^{\top}\vec{D}\vec{y}
+2\myvec{\vec{u}^{\top}\vec{p}_1 & \brak{\lambda_2\vec{c}^{\top}+\vec{u}^{\top}}\vec{p}_2}\vec{y}
	+  \vec{c}^{\top}\brak{\vec{V}\vec{c} + \vec{u}}+ \vec{u}^{\top}\vec{c} + f&= 0
\end{align}
upon substituting from 
 \eqref{eq:conic_parab_eig_prop} yielding
\begin{align}
\lambda_2y_2^2+2\brak{\vec{u}^{\top}\vec{p}_1}y_1+  2y_2\brak{\lambda_2\vec{c}+\vec{u}}^{\top}\vec{p}_2
	+  \vec{c}^{\top}\brak{\vec{V}\vec{c} + \vec{u}}+ \vec{u}^{\top}\vec{c} + f= 0
\label{eq:conic_parab_foc_len_temp} 
\end{align}
%which is the equation of a parabola. 
Thus, \eqref{eq:conic_parab_foc_len_temp} 
can be expressed as \eqref{eq:conic_simp_temp_parab} by choosing
\begin{align}
%\label{eq:eta}
\eta = 2\vec{u}^{\top}\vec{p}_1
\end{align}
%Choosing 
%\begin{align}
%\vec{u} + \lambda_2\vec{c} = 0,
%\vec{c}^{\top}\brak{\vec{V}\vec{c} + \vec{u}}+ \vec{u}^{\top}\vec{c} + f = 0,
%\end{align}
% the above equation becomes
%\begin{align}
%y_2^2= -\frac{2\vec{u}^{\top}\vec{p}_1}{ \lambda_2} \brak{y_1
%+  \frac{\vec{u}^{\top}\vec{V}\vec{u} - 2\lambda_2\vec{u}^{\top}\vec{u} + f\lambda_2^2}{2\vec{u}^{\top}\vec{p}_1\lambda_2^2}}
%\\
%or \eta = 2\vec{u}^{\top}\vec{p}_1
%%\label{eq:conic_simp_parab_new}
%\end{align}
and $\vec{c}$ in \eqref{eq:conic_simp} such that
\begin{align}
\label{eq:conic_parab_one}
2\vec{P}^{\top}\brak{\vec{V}\vec{c}+\vec{u}} &= \eta\myvec{1\\0}
\\
\vec{c}^{\top}\brak{\vec{V}\vec{c} + \vec{u}}+ \vec{u}^{\top}\vec{c} + f&= 0
\label{eq:conic_parab_two}
\end{align}
%we obtain  \eqref{eq:conic_simp_temp_parab}.
$\because
\vec{P}^{\top}\vec{P} = \vec{I}$,
multiplying \eqref{eq:conic_parab_one} by $\vec{P}$ yields
\begin{align}
\label{eq:conic_parab_one_eig}
	\brak{\vec{V}\vec{c}+\vec{u}} &= \frac{\eta}{2}\vec{p}_1,
\end{align}
which, upon substituting in \eqref{eq:conic_parab_two}
results in 
\begin{align}
\frac{\eta}{2}\vec{c}^{\top}\vec{p}_1 + \vec{u}^{\top}\vec{c} + f&= 0
\label{eq:conic_parab_two_eig}
\end{align}
\eqref{eq:conic_parab_one_eig} and \eqref{eq:conic_parab_two_eig} can be clubbed together to obtain \eqref{eq:conic_parab_c}.
  \end{proof}
	  \item
		For the standard conic, 
				\begin{align}
					\label{eq:std-prm-P}
					\vec{P} &= \vec{I}
					\\
					\vec{u} &= 
				\begin{cases}
				0 & e \ne 1
       \\
				\frac{\eta}{2} \vec{e}_1 & e = 1
				\end{cases}
				\label{eq:std-prm-u}
				\\
				\lambda_1 &  
					\begin{cases}
						=0 & e = 1
						\\
						\ne 0 & e \ne 1
					\end{cases}
				\label{eq:std-prm-lam1}
				\end{align}
				where 
				\begin{align}
					\vec{I} = \myvec{\vec{e}_1 & \vec{e}_2}
				\end{align}
				is the identity matrix.
	  
    \item\leavevmode
		\begin{enumerate}
			\item The directrices for the  standard conic are given by 
				\begin{align}
					\label{eq:dx-ell-hyp}
					\vec{e}_1^{\top}\vec{y} &=  
					%\pm\sqrt{\abs{\frac{f_0\lambda_2}{\lambda_1\brak{\lambda_2-\lambda_1}}}} & e \ne 1
					\pm \frac{1}{e}\sqrt{\frac{\abs{f_0}}{\lambda_2\brak{1-e^2}}} & e \ne 1
					\\
					\vec{e}_1^{\top}\vec{y} &= \frac{\eta}{2\lambda_2} & e = 1
					\label{eq:dx-parab}
				\end{align}
    \item The foci of the standard ellipse and hyperbola are given by 
				\begin{align}
					\label{eq:F-ell-hyp-parab}
					\vec{F} 
=
					\begin{cases}
						\pm e\sqrt{\frac{\abs{f_0}}{\lambda_2\brak{1-e^2}}}\vec{e}_1 & e \ne 1
					%	\pm \sqrt{\abs{\frac{f_0}{\lambda_1}\brak{1 - \frac{\lambda_1}{\lambda_2}}}}\vec{e}_1 & e \ne 1
						\\
						 -\frac{\eta}{4\lambda_2}\vec{e}_1 & e = 1
					\end{cases}
				\end{align}
	
		\end{enumerate}
	%	where, without loss of generality, $f_0 < 0$ for the hyperbola.
    
	\begin{proof}%\leavevmode
  \label{app:foc-dir}
%  \input{appendix.tex}
		\begin{enumerate}
			\item For the standard hyperbola/ellipse in \eqref{eq:conic_simp_temp_nonparab}, from 
					\eqref{eq:std-prm-P},
\eqref{eq:conic_quad_form_nc}
and 
					\eqref{eq:std-prm-u},
				\begin{align}
\label{eq:n-ell-hyp}
					\vec{n} &= \sqrt{\frac{\lambda_2}{f_0}} \vec{e}_1 
					\\
					c &= 
					%\pm \frac{\sqrt{-\lambda_2\brak{e^2-1}\brak{\lambda_2 f_0}}}{\lambda_2e\brak{e^2-1}}
					\pm \frac{\sqrt{-\frac{\lambda_2}{f_0}\brak{e^2-1}\brak{\frac{\lambda_2}{ f_0}}}}{\frac{\lambda_2}{f_0}e\brak{e^2-1}}
					\\
					&=\pm \frac{1}{e\sqrt{1-e^2}}
%					\\
%					&=\pm\sqrt{\abs{\frac{f_0}{\brak{1 - \frac{\lambda_1}{\lambda_2}}\frac{\lambda_1}{\lambda_2}}}}
\label{eq:c-ell-hyp}
				\end{align}
				yielding 
					\eqref{eq:dx-ell-hyp} upon substituting from 
\eqref{eq:conic_quad_form_e} and simplifying.
For the standard parabola in \eqref{eq:conic_simp_temp_parab},  from 
					\eqref{eq:std-prm-P},
\eqref{eq:conic_quad_form_nc}
and 
					\eqref{eq:std-prm-u}, noting that $f = 0$,

				\begin{align}
\label{eq:n-parab}
					\vec{n} &= \sqrt{\lambda_2} \vec{e}_1 
					\\
					c &=
	\frac{\norm{\frac{\eta}{2} \vec{e}_1}^2   }{2\vec{\brak{\frac{\eta}{2}} \brak{\vec{e}_1}^{\top}\vec{n}}} 
\\
					\\
					&= \frac{\eta}{4\sqrt{\lambda_2}}
\label{eq:c-parab}
				\end{align}
				yielding 
					\eqref{eq:dx-parab}.

				\item 	For the standard ellipse/hyperbola, substituting from
\eqref{eq:c-ell-hyp},
\eqref{eq:n-ell-hyp},
\eqref{eq:std-prm-u}
and \eqref{eq:conic_quad_form_e}
in \eqref{eq:conic_quad_form_F},
				\begin{align}
					\vec{F} &= \pm \frac{\brak{\frac{1}{e\sqrt{1-e^2}}}\brak{e^2}\sqrt{\frac{\lambda_2}{f_0}}\vec{e}_1}{\frac{\lambda_2}{f_0}}
					%\pm\sqrt{\abs{\frac{f_0}{\brak{1 - \frac{\lambda_1}{\lambda_2}}\frac{\lambda_1}{\lambda_2}}}}
					%\brak{1 - \frac{\lambda_1}{\lambda_2}}\frac{\sqrt{\lambda_2}}{\lambda_2}\vec{e}_1
 			\end{align}
			yielding
					\eqref{eq:F-ell-hyp-parab}
					after simplification.
					For the standard parabola, substituting from 
\eqref{eq:c-parab},
\eqref{eq:n-parab},
\eqref{eq:std-prm-u}
and \eqref{eq:conic_quad_form_e}
in \eqref{eq:conic_quad_form_F},			
				\begin{align}
	\vec{F}  &= \frac{\brak{\frac{\eta}{4\sqrt{\lambda_2}}}\sqrt{\lambda_2}\vec{e}_1-\vec{\frac{\eta}{2} \vec{e}_1}}{\lambda_2}
\\
				\end{align}
				yielding 
					\eqref{eq:F-ell-hyp-parab} after simplification.

		\end{enumerate}
%		See Appendix \ref{app:foc-dir}.
	\end{proof}
	\fi
\item
  Given the point of contact $\vec{q}$, the equation of a tangent to \eqref{eq:conic_quad_form} is 
  \begin{align}
  \brak{\vec{V}\vec{q}+\vec{u}}^{\top}\vec{x}+\vec{u}^{\top}\vec{q}+f = 0
  \end{align}

\begin{proof}
  The normal vector is obtained from \eqref{eq:conic_tangent_mq} 
  as
  %
  \begin{align}
	  \kappa \vec{n} = \vec{V}\vec{q}+\vec{u}, \kappa \in \mathbb{R}
  \end{align}  
  From \eqref{eq:conic_normal_vec}, the equation of the tangent is
	\begin{align}
    \brak{\vec{V}\vec{q}+\vec{u}}^{\top}\brak{\vec{x}-\vec{q}} &=0
    \\
    \implies \brak{\vec{V}\vec{q}+\vec{u}}^{\top}\vec{x}-\vec{q}^{\top}\vec{V}\vec{q}-\vec{u}^{\top}\vec{q} &= 0
    \end{align}
    which, upon substituting from \eqref{eq:conic_tangent_qquad} and simplifying yields 
  \eqref{eq:conic_tangent_final}
%	\eqref{eq:conic_tangent}.
\end{proof}
\item
  Given the point of contact $\vec{q}$, the equation of the normal to \eqref{eq:conic_quad_form} is 
  \begin{align}
    \brak{\vec{V}\vec{q}+\vec{u}}^{\top}\vec{R}\brak{\vec{x}-\vec{q}} =0
  \end{align}

\begin{proof}
  The direction vector of the tangent is obtained from 
  \eqref{eq:conic_normal_vec} as
  as
  %
  \begin{align}
	  \vec{m} = \vec{R}\brak{\vec{V}\vec{q}+\vec{u}}, 
  \end{align}  
  where $\vec{R}$ is the rotation matrix.
  From \eqref{eq:conic_tangent_vec}, the equation of the normal is
  given by 
  \eqref{eq:conic_normal_final}
\end{proof}
\item Given the tangent 
\begin{align}
  \label{eq:conic_tangent_eq}
\vec{n}^{\top}\vec{x} = c,
\end{align}
the point of  contact to the conic in \eqref{eq:conic_quad_form} is given by 
\begin{align}
  \label{eq:conic_tangent_contact}
        \myvec{\vec{n}^{\top} \\ \vec{m}^{\top}\vec{V}} \vec{q} = \myvec{c\\ -\vec{m}^{\top}\vec{u}}
\end{align}
		\begin{proof}
			From
  \eqref{eq:conic_tangent_mq},
\begin{align}
	\vec{m}^{\top}(\vec{V}\vec{q}+\vec{u})&=0
	\\
	\implies        \vec{m}^{\top}\vec{V}\vec{q} &= -\vec{m}^{\top}\vec{u}
\end{align}
Combining 
  \eqref{eq:conic_tangent_eq}
  and 
  \eqref{eq:conic_tangent_contact_eq}, 
  \eqref{eq:conic_tangent_contact} is obtained.

		\end{proof}

\item
  If $\vec{V}^{-1}$ exists, given the normal vector $\vec{n}$, the tangent points of contact to \eqref{eq:conic_quad_form} are given by
\begin{align}
  \begin{split}
\vec{q}_i &= \vec{V}^{-1}\brak{\kappa_i \vec{n}-\vec{u}}, i = 1,2
\\
\text{where }\kappa_i &= \pm \sqrt{
\frac{
f_0
%\vec{u}^{\top}\vec{V}^{-1}\vec{u}-f
}
{
\vec{n}^{\top}\vec{V}^{-1}\vec{n}
}
}
  \end{split}
\end{align}

\begin{proof}
  From \eqref{eq:conic_normal_vec},
\begin{align}
 \vec{q} = \vec{V}^{-1}\brak{\kappa \vec{n}-\vec{u}}, \quad \kappa \in \mathbb{R}
\end{align}
Substituting \eqref{eq:conic_normal_vec_q}
in \eqref{eq:conic_tangent_qquad},
\begin{align}
\brak{\kappa \vec{n}-\vec{u}}^{\top}\vec{V}^{-1}\brak{\kappa \vec{n}-\vec{u}} 
%\\
+ 2\vec{u}^{\top}\vec{V}^{-1}\brak{\kappa \vec{n}-\vec{u}} +f &= 0
\\
\implies 
\kappa^2 \vec{n}^{\top}\vec{V}^{-1}\vec{n} - \vec{u}^{\top}\vec{V}^{-1}\vec{u} + f &=0
 \\
 \text{or, } \kappa = \pm \sqrt{\frac{
	 %\vec{u}^{\top}\vec{V}^{-1}\vec{u}-f
	f_0 
 }{\vec{n}^{\top}\vec{V}^{-1}\vec{n}}} &
\end{align}
%
%yileding 
Substituting \eqref{eq:conic_normal_k} in \eqref{eq:conic_normal_vec_q}
yields \eqref{eq:conic_tangent_qk}.
%
\end{proof}
\item For a conic/hyperbola, a line with normal vector $\vec{n}$ cannot be a tangent if 
\begin{align}
\frac{
\vec{u}^{\top}\vec{V}^{-1}\vec{u}-f
}
{
\vec{n}^{\top}\vec{V}^{-1}\vec{n}
} < 0
\end{align}

\item
	\label{eq:conic-p-contact-parab}
  If $\vec{V}$ is not invertible,  given the normal vector $\vec{n}$, the point of contact to \eqref{eq:conic_quad_form} is given by the matrix equation
\begin{align}
\label{eq:conic_tangent_q_eigen}
\myvec{
\vec{\brak{u+\kappa \vec{n}}}^{\top} \\ \vec{V}
}
\vec{q} &= 
\myvec{
-f
\\
\kappa\vec{n}-\vec{u}
}
\\
\text{where }  \kappa = \frac{\vec{p}_1^{\top}\vec{u}}{\vec{p}_1^{\top}\vec{n}}, \quad \vec{V}\vec{p}_1 &= 0
\label{eq:conic_tangent_qk_eigen}
\end{align}


\begin{proof}
  If $\vec{V}$ is non-invertible, it has a zero eigenvalue.  If the corresponding eigenvector is $\vec{p}_1$, then,
\begin{align}
\vec{V}\vec{p}_1 = 0
\label{eq:conic_zero_eigen}
\end{align}
From \eqref{eq:conic_normal_vec},
\begin{align}
\label{eq:conic_zero_eigen_normal}
\kappa \vec{n} &= \vec{V} \vec{q}+\vec{u}, \quad \kappa \in \mathbb{R}
\\
\implies \kappa \vec{p}_1^{\top}\vec{n} &= \vec{p}_1^{\top}\vec{V} \vec{q}+\vec{p}_1^{\top}\vec{u}
\\
\text{or, } \kappa \vec{p}_1^{\top}\vec{n} &= \vec{p}_1^{\top}\vec{u},  \quad \because \vec{p}_1^{\top} \vec{V} = 0, 
%\\
\quad 
\brak{\text{ from } \eqref{eq:conic_zero_eigen}}
%\label{eq:conic_normal_vec_q}
\end{align}
yielding $\kappa$ in \eqref{eq:conic_tangent_qk_eigen}. From \eqref{eq:conic_zero_eigen_normal},
\begin{align}
\kappa \vec{q}^{\top}\vec{n} &= \vec{q}^{\top}\vec{V} \vec{q}+\vec{q}^{\top}\vec{u}
\\
\implies \kappa \vec{q}^{\top}\vec{n} &= -f-\vec{q}^{\top}\vec{u} \quad \text{from } \eqref{eq:conic_tangent_qquad},
\\
\text{or, } \brak{\kappa \vec{n}+\vec{u}}^{\top}\vec{q} &= -f
\label{eq:conic_zero_eigen_normal_fq}
\end{align}
\eqref{eq:conic_zero_eigen_normal} can be expressed as
\begin{align}
\label{eq:conic_zero_eigen_normal_vq}
\vec{V} \vec{q} = \kappa \vec{n} - \vec{u}.
\end{align}
\eqref{eq:conic_zero_eigen_normal_fq} and \eqref{eq:conic_zero_eigen_normal_vq} clubbed together result in \eqref{eq:conic_tangent_q_eigen}.
\end{proof}
\item
	The asymptotes of the hyperbola in 
    \eqref{eq:conic_simp_temp_nonparab}, defined to be the lines that do not intersect the hyperbola, are given by 
    \begin{align} 
    \label{eq:pair-std}
    \myvec{\sqrt{\abs{\lambda_1}} & \pm \sqrt{\abs{\lambda_2}}}\vec{y} = 0
    \end{align} 
%  \begin{align}
%	  \myvec{\lambda_1 & \pm \lambda_2}\vec{y} = 0   
%  \end{align}
  
  \begin{proof}
	  From 
\eqref{eq:conic_simp_temp_nonparab},
it is obvious that 
the pair of lines represented by 
  \begin{align}
	    \vec{y}^{\top}\vec{D}\vec{y} = 0   
      \label{eq:pair-conic}
  \end{align}
  do not intersect the conic 
  \begin{align}
	    \vec{y}^{\top}\vec{D}\vec{y} =  f_0  
  \end{align}
  Thus, 
      \eqref{eq:pair-conic}
      represents the asysmptotes of the hyperbola in 
\eqref{eq:conic_simp_temp_nonparab} and can be expressed as 
  \begin{align} 
    \lambda_1y_1^2 +\lambda_2y_1^2 = 0, 
    \label{eq:quad_form_hyper}
    \end{align}
%    \eqref{eq:quad_form_hyper}
which can then be simplified  
using the steps in 
	\eqref{eq:incircle-disc-v}-
	\eqref{eq:incircle-disc-v-lam}
to obtain
    \eqref{eq:pair-std}.
  \end{proof}
  \item
\eqref{eq:conic_quad_form} represents a pair of straight lines if 
  \begin{align} 
	  \label{eq:pair-cond}
%	  \lambda_1y_1^2 +\lambda_2y_2^2 = 
  \vec{u}^{\top}\vec{V}^{-1}\vec{u} -f  = 0
  \end{align} 
  
  \item
	  \label{them:pair-mat-sing}
\eqref{eq:conic_quad_form} represents a pair of straight lines if 
the matrix 
  \begin{align} 
	  \myvec{\vec{V} & \vec{u}\\ \vec{u}^{\top} & f}  
	  \label{eq:pair-mat-sing}
  \end{align} 
  is singular.
  
  \begin{proof}
Let 
  \begin{align} 
	  \myvec{\vec{V} & \vec{u}\\ \vec{u}^{\top} & f}  \vec{x} =\vec{0}
  \end{align} 
  Expressing 
  \begin{align} 
	  \vec{x} =\myvec{\vec{y} \\ y_3}, 
  \end{align} 
  \begin{align} 
	  \myvec{\vec{V} & \vec{u}\\ \vec{u}^{\top} & f}   
	  \myvec{\vec{y} \\ y_3} &= \vec{0}
	  \\
	  \implies
	  \label{eq:pair-mat-sing-1}
	  \vec{V} \vec{y} + y_3\vec{u} &= \vec{0} \quad \text{and}
	  \\
	  \vec{u}^{\top}\vec{y} + fy_3 &=0
	  \label{eq:pair-mat-sing-2}
  \end{align} 
  From 
	  \eqref{eq:pair-mat-sing-1} we obtain,
  \begin{align} 
	  \vec{y}^{\top}  \vec{V} \vec{y} + y_3\vec{y}^{\top}\vec{u} &= \vec{0} 
	  \\
	  \implies 
	  \vec{y}^{\top}  \vec{V} \vec{y} + y_3\vec{u}^{\top}\vec{y} &= \vec{0} 
  \end{align} 
  yielding 
	  \eqref{eq:pair-cond} upon substituting from 
	  \eqref{eq:pair-mat-sing-2}.
  \end{proof}
  \item
	  Using the affine transformation, 
    \eqref{eq:pair-std}
 can be expressed as the lines 
%
\begin{align} 
\label{eq:quad_form_pair}
\myvec{\sqrt{\abs{\lambda_1}} & \pm \sqrt{\abs{\lambda_2}}}\vec{P}^{\top}\brak{\vec{x}-\vec{c}} = 0
\end{align} 
  
   \item
	   The angle between the asymptotes can be expressed as
\begin{align} 
\label{eq:quad_form_pair_ang}
\cos\theta=\frac{\abs{\lambda_1}-\abs{\lambda_2}}
{\abs{\lambda_1}+\abs{\lambda_2}}
\end{align} 
  
  \begin{proof}
The normal vectors of the lines in \eqref{eq:quad_form_pair} are 
  \begin{align} 
  \label{eq:quad_form_pair_normvecs}
  \begin{split}
  \vec{n}_1 &= \vec{P}\myvec{\sqrt{\abs{\lambda_1}} \\[2mm]  \sqrt{\abs{\lambda_2}}}
  \\
  \vec{n}_2 &= \vec{P}\myvec{\sqrt{\abs{\lambda_1}} \\[2mm] - \sqrt{\abs{\lambda_2}}}
  \end{split}
  \end{align} 
  The angle between the asymptotes is given by 
\begin{align} 
\label{eq:quad_form_pair_ang_exp}
\cos\theta=\frac{\vec{n_1}^{\top}\vec{n_2}}{\norm{\vec{n_1}}\norm{\vec{n_2}}}
\end{align} 
The orthogonal matrix $\vec{P}$ preserves the norm, i.e.
\begin{align} 
	\norm{\vec{n_1}} &= \norm{\vec{P}\myvec{\sqrt{\abs{\lambda_1}} \\[2mm]  \sqrt{\abs{\lambda_2}}}}
	=\norm{\myvec{\sqrt{\abs{\lambda_1}} \\[2mm]  \sqrt{\abs{\lambda_2}}}}
	\\
	&=\sqrt{\abs{\lambda_1}+\abs{\lambda_2}} = \norm{\vec{n_2}}
\end{align} 
It is easy to verify that 
\begin{align} 
\vec{n_1}^{\top}\vec{n_2} = \abs{\lambda_1}-\abs{\lambda_2}
\end{align} 
%
Thus, the angle between the asymptotes is obtained from \eqref{eq:quad_form_pair_ang_exp} as \eqref{eq:quad_form_pair_ang}.
  \end{proof}
\item
  If $L$ in \eqref{eq:conic_tangent} touches \eqref{eq:conic_quad_form} at exactly one point $\vec{q}$, 
  \begin{align}
  \vec{m}^{\top}\brak{\vec{V}\vec{q}+\vec{u}} = 0
  \end{align}
\begin{proof}
  In this case, \eqref{eq:conic_intercept} has exactly one root.  Hence, 
  in \eqref{eq:tangent_roots}
  \begin{align}
  \sbrak{
  \vec{m}^{\top}\brak{\vec{V}\vec{q}+\vec{u}}
  }^2 -\brak{\vec{m}^{\top}\vec{V}\vec{m}}
	  \text{g}\brak
  {
  \vec{q}
%  \vec{q}^{\top}\vec{V}\vec{q} + 2\vec{u}^{\top}\vec{q} +f
  } = 0                                                                                             
  \label{eq:conic_tangent_disc}
  \end{align}                    
  $\because \vec{q}$ is the point of contact,
	%$\vec{q}$ satisfies \eqref{eq:conic_quad_form}
%  and 
  \begin{align}
	  \text{g}\brak{  \vec{q}} = 0
%  \vec{q}^{\top}\vec{V}\vec{q} + 2\vec{u}^{\top}\vec{q} +f = 0
  \label{eq:conic_tangent_qquad}
  \end{align}
  Substituting \eqref{eq:conic_tangent_qquad} in \eqref{eq:conic_tangent_disc} and simplifying, we obtain \eqref{eq:conic_tangent_mq}.
\end{proof}
\item
  Given the point of contact $\vec{q}$, the equation of a tangent to \eqref{eq:conic_quad_form} is 
  \begin{align}
  \brak{\vec{V}\vec{q}+\vec{u}}^{\top}\vec{x}+\vec{u}^{\top}\vec{q}+f = 0
  \end{align}
\begin{proof}
  The normal vector is obtained from \eqref{eq:conic_tangent_mq} 
  as
  %
  \begin{align}
  \label{eq:conic_normal_vec}
	  \kappa \vec{n} = \vec{V}\vec{q}+\vec{u}, \kappa \in \mathbb{R}
  \end{align}  
  From \eqref{eq:conic_normal_vec}, the equation of the tangent is\begin{align}
    \brak{\vec{V}\vec{q}+\vec{u}}^{\top}\brak{\vec{x}-\vec{q}} &=0
    \\
    \implies \brak{\vec{V}\vec{q}+\vec{u}}^{\top}\vec{x}-\vec{q}^{\top}\vec{V}\vec{q}-\vec{u}^{\top}\vec{q} &= 0
    \end{align}
    which, upon substituting from \eqref{eq:conic_tangent_qquad} and simplifying yields 
  \eqref{eq:conic_tangent_final}
%	\eqref{eq:conic_tangent}.
\end{proof}
\item
  Given the point of contact $\vec{q}$, the equation of the normal to \eqref{eq:conic_quad_form} is 
  \begin{align}
    \brak{\vec{V}\vec{q}+\vec{u}}^{\top}\vec{R}\brak{\vec{x}-\vec{q}} =0
  \end{align}
\begin{proof}
  The direction vector of the tangent is obtained from 
  \eqref{eq:conic_normal_vec} as
  as
  %
  \begin{align}
  \label{eq:conic_tangent_vec}
	  \vec{m} = \vec{R}\brak{\vec{V}\vec{q}+\vec{u}}, 
  \end{align}  
  where $\vec{R}$ is the rotation matrix.
  From \eqref{eq:conic_tangent_vec}, the equation of the normal is
  given by 
  \eqref{eq:conic_normal_final}
\end{proof}
\item Given the tangent 
\begin{align}
\vec{n}^{\top}\vec{x} = c,
\end{align}
the point of  contact to the conic in \eqref{eq:conic_quad_form} is given by 
\begin{align}
        \myvec{\vec{n}^{\top} \\ \vec{m}^{\top}\vec{V}} \vec{q} = \myvec{c\\ -\vec{m}^{\top}\vec{u}}
\end{align}
		\begin{proof}
			From
  \eqref{eq:conic_tangent_mq},
\begin{align}
	\vec{m}^{\top}(\vec{V}\vec{q}+\vec{u})&=0
	\\
	\implies        \vec{m}^{\top}\vec{V}\vec{q} &= -\vec{m}^{\top}\vec{u}
  \label{eq:conic_tangent_contact_eq}
\end{align}
Combining 
  \eqref{eq:conic_tangent_eq}
  and 
  \eqref{eq:conic_tangent_contact_eq}, 
  \eqref{eq:conic_tangent_contact} is obtained.

		\end{proof}

\item
  If $\vec{V}^{-1}$ exists, given the normal vector $\vec{n}$, the tangent points of contact to \eqref{eq:conic_quad_form} are given by
\begin{align}
  \begin{split}
\vec{q}_i &= \vec{V}^{-1}\brak{\kappa_i \vec{n}-\vec{u}}, i = 1,2
\\
\text{where }\kappa_i &= \pm \sqrt{
\frac{
f_0
%\vec{u}^{\top}\vec{V}^{-1}\vec{u}-f
}
{
\vec{n}^{\top}\vec{V}^{-1}\vec{n}
}
}
  \end{split}
\end{align}
\begin{proof}
  From \eqref{eq:conic_normal_vec},
\begin{align}
\label{eq:conic_normal_vec_q}
 \vec{q} = \vec{V}^{-1}\brak{\kappa \vec{n}-\vec{u}}, \quad \kappa \in \mathbb{R}
\end{align}
Substituting \eqref{eq:conic_normal_vec_q}
in \eqref{eq:conic_tangent_qquad},
\begin{align}
\brak{\kappa \vec{n}-\vec{u}}^{\top}\vec{V}^{-1}\brak{\kappa \vec{n}-\vec{u}} 
%\\
+ 2\vec{u}^{\top}\vec{V}^{-1}\brak{\kappa \vec{n}-\vec{u}} +f &= 0
\\
\implies 
\kappa^2 \vec{n}^{\top}\vec{V}^{-1}\vec{n} - \vec{u}^{\top}\vec{V}^{-1}\vec{u} + f &=0
 \\
 \text{or, } \kappa = \pm \sqrt{\frac{
	 %\vec{u}^{\top}\vec{V}^{-1}\vec{u}-f
	f_0 
 }{\vec{n}^{\top}\vec{V}^{-1}\vec{n}}} &
	\label{eq:conic_normal_k}
\end{align}
%
%yileding 
Substituting \eqref{eq:conic_normal_k} in \eqref{eq:conic_normal_vec_q}
yields \eqref{eq:conic_tangent_qk}.
%
\end{proof}
\item For a conic/hyperbola, a line with normal vector $\vec{n}$ cannot be a tangent if 
\begin{align}
\frac{
\vec{u}^{\top}\vec{V}^{-1}\vec{u}-f
}
{
\vec{n}^{\top}\vec{V}^{-1}\vec{n}
} < 0
\end{align}
\item For a circle, the points of contact are
	\begin{align}
	\vec{q}_{ij} &= \brak{\pm r \frac{\vec{n}_j}{\norm{\vec{n}_j}}-\vec{u}}, \quad i,j = 1,2
\end{align}
\begin{proof}
	From 
\eqref{eq:conic_tangent_qk},
and 
	\eqref{eq:circ-cr},
\begin{align}
\kappa_{ij} &= \pm 
\frac{r
}
{
	\norm{\vec{n}_j}
}
\end{align}
\end{proof}
\item A point $\vec{h}$ lies on a normal to the conic in \eqref{eq:conic_quad_form} 
	if
\begin{multline}
	\brak{ {\vec{m}^\top(\vec{Vh}+\vec{u})}}^2\brak{\vec{n}^{\top}\vec{V}\vec{n}} 
	\\
	- 2\brak{\vec{m}^\top\vec{V}\vec{n}} \brak{ {\vec{m}^\top(\vec{Vh}+\vec{u})}\vec{n}^{\top}\brak{\vec{V}\vec{h}+\vec{u}}} 
	\\
+  \text{g}\brak{
  \vec{h}
	  }\brak{\vec{m}^\top\vec{V}\vec{n}}^2
%	\vec{h}^{\top}\vec{V}\vec{h} + 2\vec{u}^{\top}\vec{h} +f 
	= 0
\end{multline}
\begin{proof}
The point of contact for the normal passing through a point $\vec{h}$ is given by 
\begin{align}
	\label{eq:point_of_tangency}
	\vec{q} = \vec{h} + \mu\vec{n}
	%\vec{q} = \vec{h} + \mu\vec{R}\vec{m}
\end{align}
From 
  \eqref{eq:conic_tangent_mq},
	the tangent at $\vec{q}$ satisfies 
\begin{align}
	\label{eq:tangency_condition}
	\vec{m}^\top(\vec{Vq}+\vec{u}) = 0
\end{align}
Substituting \eqref{eq:point_of_tangency} in \eqref{eq:tangency_condition},
\begin{align}
	\label{eq:normal_simp_1}
	\vec{m}^\top(\vec{V}(\vec{h}+\mu\vec{n})+\vec{u}) = 0\\
	%\vec{m}^\top(\vec{V}(\vec{h}+\mu\vec{R}\vec{m})+\vec{u}) = 0\\
	\label{eq:normal_simp_2}
	\implies \mu\vec{m}^\top\vec{V}\vec{n} = -\vec{m}^\top(\vec{Vh}+\vec{u})
	%\implies \mu\vec{m}^\top\vec{V}\vec{R}\vec{m} = -\vec{m}^\top(\vec{Vh}+\vec{u})
\end{align}
yielding 
\begin{align}
	\label{eq:point_of_tangency-mu}
	\mu &=- \frac  {\vec{m}^\top(\vec{Vh}+\vec{u})}{\vec{m}^\top\vec{V}\vec{n}},
\end{align}
%	\eqref{eq:point_of_tangency-mu}.
	From 
\eqref{eq:conic_intercept},
\begin{align}
	\label{eq:normal_simp_2-quad}
\mu^2\vec{n}^{\top}\vec{V}\vec{n} + 2 \mu\vec{n}^{\top}\brak{\vec{V}\vec{h}+\vec{u}} 
+  \text{g}\brak{
  \vec{h}
	  }
%	\vec{h}^{\top}\vec{V}\vec{h} + 2\vec{u}^{\top}\vec{h} +f 
	&= 0
\end{align}
From 
	\eqref{eq:point_of_tangency-mu},
	\eqref{eq:normal_simp_2-quad} can be expressed
	as
\begin{multline}
%	\label{eq:normal_simp_2-quad}
\brak{- \frac{\vec{m}^\top(\vec{Vh}+\vec{u})}{\vec{m}^\top\vec{V}\vec{n}}}^2\vec{n}^{\top}\vec{V}\vec{n} 
	\\
	+ 2 \brak{- \frac{\vec{m}^\top(\vec{Vh}+\vec{u})}{\vec{m}^\top\vec{V}\vec{n}}}\vec{n}^{\top}\brak{\vec{V}\vec{h}+\vec{u}} 
+  \text{g}\brak{
  \vec{h}
	  }
	= 0
\end{multline}
	yielding
	\eqref{eq:point_of_tangency-m}.
\end{proof}
\item  A point $\vec{h}$ lies on a tangent to the conic in \eqref{eq:conic_quad_form} if 
\begin{align}
  \vec{m}^{\top}  \sbrak{\brak{\vec{V}\vec{h}+\vec{u}}
	  \brak{\vec{V}\vec{h}+\vec{u}}^{\top}
   -\vec{V}
	  \text{g}\brak{
  \vec{h}
	  }
	  }\vec{m} 
	  &= 0                                                                                             
\end{align}
\begin{proof}
 From \eqref{eq:tangent_roots}
 and
  \eqref{eq:conic_tangent_disc}
  \begin{align}
  \sbrak{
  \vec{m}^{\top}\brak{\vec{V}\vec{h}+\vec{u}}
  }^2 -\brak{\vec{m}^{\top}\vec{V}\vec{m}}
	  \text{g}\brak{
  \vec{h}
	  }
	  &= 0                                                                                             
	  \iffalse
  \\
	  \implies 
  \brak
  {
  \vec{h}^{\top}\vec{V}\vec{h} + 2\vec{u}^{\top}\vec{h} +f
  }
\fi
  \label{eq:conic_tangent_disc-h}
  \end{align}                    
  yielding
	  \eqref{eq:h-tangents-cond}.
\end{proof}
\item
	The normal vectors of the tangents 
to the conic in \eqref{eq:conic_quad_form} 
	from 
	a point $\vec{h}$ 
	are given by 
  \begin{align} 
  \label{eq:quad_form_pair_normvecs-sigma}
  \begin{split}
  \vec{n}_1 &= \vec{P}\myvec{\sqrt{\abs{\lambda_1}} \\[2mm]  \sqrt{\abs{\lambda_2}}}
  \\
  \vec{n}_2 &= \vec{P}\myvec{\sqrt{\abs{\lambda_1}} \\[2mm] - \sqrt{\abs{\lambda_2}}}
  \end{split}
  \end{align} 
  where $\lambda_i, \vec{P}$ are the eigenparameters of 
  \begin{align} 
		\bm{\Sigma} &= 
	   \brak{\vec{V}\vec{h}+\vec{u}}
	  \brak{\vec{V}\vec{h}+\vec{u}}^{\top}
   -
  \brak
  {
	  \text{g}\brak{
  \vec{h}
	  }
%  \vec{h}^{\top}\vec{V}\vec{h} + 2\vec{u}^{\top}\vec{h} +f
  }\vec{V}.
	  \label{eq:h-tangents-sigma}
  \end{align}                    
  \begin{proof}
	  From 
	  \eqref{eq:h-tangents-cond} we obtain
	  \eqref{eq:h-tangents-sigma}.  Consequently, from 
  \eqref{eq:quad_form_pair_normvecs}, 
  \eqref{eq:quad_form_pair_normvecs-sigma}
  can be obtained.
  \end{proof}

	
\end{enumerate}
