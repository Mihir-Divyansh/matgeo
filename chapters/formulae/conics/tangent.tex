\begin{enumerate}[label=\thesubsection.\arabic*,ref=\thesubsection.\theenumi]
\item
The equation of  a conic  is given by
		\begin{align}
    \label{eq:conic_quad_form}
	\text{g}\brak{\vec{x}} = \vec{x}^{\top}\vec{V}\vec{x}+2\vec{u}^{\top}\vec{x}+f=0
    \end{align}
	\item The equation of the {\em incircle} is given by 
		\begin{align}
			\label{eq:incircle}
			\norm{\vec{x}-\vec{O}}^2 = r^2
		\end{align}
		which can be expressed as 
			 \eqref{eq:conic_quad_form}
			 using 
			 \eqref{eq:conic_quad_form-params}.
		 \item 	In \figref{fig:incircle}, 
let 
  \eqref{eq:line_dir_pt-lam}
  be the equation of $AB$.  Then, the intersection of 
  \eqref{eq:line_dir_pt-lam}
  and 
			 \eqref{eq:conic_quad_form}
			 can be expressed as 
\begin{align}
\brak{\vec{h} + \mu{\vec{m}}}^{\top}
\vec{V}
\brak{\vec{h} + \mu{\vec{m}}}
			+2\vec{u}^{\top}\brak{\vec{h} + \mu{\vec{m}}}+f &= 0
			\\
\implies \mu^2\vec{m}^{\top} \vec{V}\vec{m} + 2\mu \vec{m}^{\top}\brak{\vec{V}\vec{h}+\vec{u}}+g\brak{\vec{h}} &= 0 
	\label{eq:incircle-quad}
\end{align}
For 	\eqref{eq:incircle-quad} to have exactly one root, the discriminant
\begin{align}
 \cbrak{\vec{m}^{\top}\brak{\vec{V}\vec{h}+\vec{u}}}^2 -g\brak{\vec{h}}\vec{m}^{\top} \vec{V}\vec{m}  &= 0 
	\label{eq:incircle-disc}
\end{align}
and 
  \eqref{eq:line_dir_pt-lam-mu}
  is obtained.
  \item 
	\eqref{eq:incircle-disc}
	can be expressed as
\begin{align}
\vec{m}^{\top}\brak{\vec{V}\vec{h}+\vec{u}}^{\top}\brak{\vec{V}\vec{h}+\vec{u}}\vec{m}-g\brak{\vec{h}}\vec{m}^{\top} \vec{V}\vec{m}  &= 0 
\\
\implies \vec{m}^{\top}\vec{\Sigma}\vec{m} &= 0
	\label{eq:incircle-disc-Sigma-new}
\end{align}
for $\vec{\Sigma}$ defined in 
	\eqref{eq:incircle-disc-Sigma-new}.
      Substituting \eqref{eq:conic_parmas_eig_def}
	in \eqref{eq:incircle-disc-Sigma-new},
\begin{align}
\vec{m}^{\top}\vec{P}\vec{D}\vec{P}^{\top}\vec{m} &= 0
\\
\implies 
\vec{v}^{\top}\vec{D}\vec{v} &= 0
	\label{eq:incircle-disc-v}
\end{align}
where 
\begin{align}
	\label{eq:incircle-disc-v-lam-P}
\vec{v} = \vec{P}^{\top}\vec{m}
\end{align}
	\eqref{eq:incircle-disc-v}
	can be expressed as 
\begin{align}
\lambda_1 v_1^2
-\lambda_2 v_2^2 &= 0
\\
\implies \vec{v} = \myvec{\sqrt{\abs{\lambda_2}} \\[2mm]  \pm \sqrt{\abs{\lambda_1}}}
	\label{eq:incircle-disc-v-lam}
\end{align}
after some algebra.
From 
	\eqref{eq:incircle-disc-v-lam}
	and
	\eqref{eq:incircle-disc-v-lam-P}
	we obtain 
	  \eqref{eq:h-tangents-cond-mPlam}.
	  \end{enumerate}
