\begin{enumerate}[label=\thesection.\arabic*.,ref=\thesection.\theenumi]
\numberwithin{equation}{enumi}
\item The equation of a line is given by 
\begin{align}
			\label{eq:line-school}
	y &= mx + c
	\\
	\implies \myvec{x \\ y} &= \myvec{x \\ 
	 mx + c} =\myvec{0 \\ c} + x\myvec{1 \\ m}
\end{align}
			yielding \eqref{eq:geo-param}.
\item 			\eqref{eq:line-school} can also be expressed as
\begin{align}
	y - mx &= c 
	\\
	\implies \myvec{-m & 1}\myvec{x \\ y} &= c
\end{align}
			yielding \eqref{eq:geo-normal}.
		\item The direction vector is 
\begin{align}
			\label{eq:line-school-dir}
\vec{m} = \myvec{1 \\ m}
\end{align}
and the normal vector is
\begin{align}
\vec{n}=\myvec{-m \\ 1}
			\label{eq:line-school-normal}
\end{align}
  \item From \eqref{eq:geo-param}, 
	  if $\vec{A},\vec{D}$ and $\vec{C}$ are on the same line,
		\label{prop:lin-dep}
\begin{align}
			\vec{D}=\vec{A}+q\vec{m} 
			\\ 
			\vec{C}=\vec{D}+p\vec{m} \\
			\label{eq:collinear} 
			\implies 	p\brak{\vec{D}-\vec{A}} 
			+ q\brak{\vec{D}-\vec{C}} = 0, \quad p, q \ne 0 \\ 
			\implies \vec{D} = \frac{p\vec{A}+q\vec{C}}{p+q} 
			\end{align} 
			yielding \eqref{eq:section_formula} upon substituting \begin{align} k = \frac{p}{q}. \end{align} 
			$\brak{\vec{D}-\vec{A}}, \brak{\vec{D}-\vec{C}}$ 
		are then said to be {\em linearly dependent}.
	\item If $\vec{A}, \vec{B}, \vec{C}$ are collinear,  from \eqref{eq:geo-normal}, \begin{align}
	 \vec{n}^{\top}\vec{A} &=  c 
	 \\
	 \vec{n}^{\top}\vec{B} &=  c 
	 \\
	 \vec{n}^{\top}\vec{C} &=  c 
\end{align}
which can be expressed as 
\begin{align}
		\label{prop:lin-eq}
	\myvec{ \vec{A} & \vec{B} & \vec{C}}^{\top}\vec{n} = c\myvec{1 \\ 1 \\ 1}
	\\
	\equiv \myvec{ \vec{A} & \vec{B} & \vec{C}}^{\top}\vec{n} = \myvec{1 \\ 1 \\ 1}
		\label{prop:lin-eq-unit-mat},
	\\
	\implies 
	\myvec{ 1 & 1 &1 \\ \vec{A} & \vec{B} & \vec{C}}^{\top}\myvec{\vec{n} \\ -1} &= \vec{0}
		\label{prop:lin-dep-rank}
\end{align}
yielding
			\eqref{eq:line-rank}.  Rank is defined to be the number of linearly indpendent rows or columns of a matrix.
		\item
The equation of a line can also be expressed as
\begin{align}
	 \vec{n}^{\top}\vec{x} &=   1
		\label{prop:lin-eq-unit}
\end{align}
	  \end{enumerate}
