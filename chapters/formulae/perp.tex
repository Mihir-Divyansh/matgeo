\begin{enumerate}[label=\thesubsection.\arabic*.,ref=\thesubsection.\theenumi]
\item The equation of a line is given by 
\begin{align}
			\label{eq:line-school}
	y &= mx + c
	\\
	\implies \myvec{x \\ y} &= \myvec{x \\ 
	 mx + c} =\myvec{0 \\ c} + x\myvec{1 \\ m}
\end{align}
			yielding 
\begin{align}
\label{eq:geo-param}
	\vec{x} = \vec{h} + \kappa \vec{m}.
\end{align}
where $\vec{h}$ is any point on the line and 
\begin{align}
			\label{eq:line-school-dir}
\vec{m} = \myvec{1 \\ m}
\end{align}
is the direction vector. 
\item 
	For
\begin{align}
	\vec{m}^{\top}\vec{n} = 0,
\end{align}
\eqref{eq:geo-param}
	can be expressed as
\begin{align}
	\vec{n}^{\top}\vec{x} &= \vec{n}^{\top}\vec{h} + \kappa \vec{n}^{\top}\vec{m}
	\\
	\begin{split}
	\implies	\vec{n}^{\top}\brak{\vec{x}-\vec{h}} &=0 
	\\
		\text{or, }
	\vec{n}^{\top}\vec{x} &= c
	\end{split}
	\label{eq:geo-normal}
\end{align}
for 
\begin{align}
c = 	\vec{n}^{\top}\vec{h}. 
\end{align}
where
\begin{align}
\vec{n}=\myvec{-m \\ 1}
			\label{eq:line-school-normal}
\end{align}
is defined to be the
{\em normal vector}
		of the line.  
	In 3D, \eqref{eq:geo-normal} represents a plane.
	\iffalse
\item 			\eqref{eq:line-school} can also be expressed as
\begin{align}
	y - mx &= c 
	\\
	\implies \myvec{-m & 1}\myvec{x \\ y} &= c
\end{align}
			yielding 
\begin{align}
	\label{eq:geo-normal}
\vec{n}^{\top}\vec{x} &= c
\end{align}
		\item 
and the normal vector is
\item The equation of a line passing through 
  \item From \eqref{eq:geo-param}, 
	  if $\vec{A},\vec{D}$ and $\vec{C}$ are on the same line,
		\label{prop:lin-dep}
\begin{align}
			\vec{D}=\vec{A}+q\vec{m} 
			\\ 
			\vec{C}=\vec{D}+p\vec{m} \\
			\label{eq:collinear} 
			\implies 	p\brak{\vec{D}-\vec{A}} 
			+ q\brak{\vec{D}-\vec{C}} = 0, \quad p, q \ne 0 \\ 
			\implies \vec{D} = \frac{p\vec{A}+q\vec{C}}{p+q} 
			\end{align} 
			yielding \eqref{eq:section_formula} upon substituting \begin{align} k = \frac{p}{q}. \end{align} 
			$\brak{\vec{D}-\vec{A}}, \brak{\vec{D}-\vec{C}}$ 
		are then said to be {\em linearly dependent}.
			  \fi
	\item If $\vec{A}, \vec{B}, \vec{C}$ are collinear,  from \eqref{eq:geo-normal}, \begin{align}
	 \vec{n}^{\top}\vec{A} &=  c 
	 \\
	 \vec{n}^{\top}\vec{B} &=  c 
	 \\
	 \vec{n}^{\top}\vec{C} &=  c 
\end{align}
which can be expressed as 
\begin{align}
		\label{prop:lin-eq}
	\myvec{ \vec{A} & \vec{B} & \vec{C}}^{\top}\vec{n} = c\myvec{1 \\ 1 \\ 1}
	\\
	\equiv \myvec{ \vec{A} & \vec{B} & \vec{C}}^{\top}\vec{n} = \myvec{1 \\ 1 \\ 1}
		\label{prop:lin-eq-unit-mat},
	\\
	\implies 
	\myvec{ 1 & 1 &1 \\ \vec{A} & \vec{B} & \vec{C}}^{\top}\myvec{\vec{n} \\ -1} &= \vec{0}
		\label{prop:lin-dep-rank}
\end{align}
\iffalse
yielding
		\begin{align}
			\label{eq:line-rank}
			\rank{\myvec{1 & 1 & 1 \\ \vec{A}& \vec{B}&\vec{C}}} = 2
		\end{align}
			  Rank is defined to be the number of linearly indpendent rows or columns of a matrix.
			  \fi
		\item
The equation of a line that does not pass through the origin can be expressed as
\begin{align}
	 \vec{n}^{\top}\vec{x} &=   1
		\label{prop:lin-eq-unit}
\end{align}
\item Let the perpendicular distance from the origin to a line be $p$ and the angle made by the perpendicular with the positive $x$-axis be $\theta$.
	Then 
\begin{align}
	p\myvec{\cos \theta \\ \sin \theta}
\end{align}
is a point on the line as well as the normal vector.
Hence, the equation of the line is 
\begin{align}
	p\myvec{\cos \theta & \sin \theta}
	\cbrak{\vec{x}-p\myvec{\cos \theta \\ \sin \theta}} &= 0
	\\
	\implies 
	\myvec{\cos \theta & \sin \theta}
	\vec{x} &= p
\label{eq:chapters/11/10/2/8-final}
\end{align}
\item Let $\vec{Q}$ be the foot of the perpendicular from $\vec{P}$
	to the line
\begin{align}
			\label{eq:geo-norm-app}
    \vec{n}^{\top}  \vec{x} = c
\end{align}
From
			\eqref{eq:geo-param}
\begin{align}
			\label{eq:geo-param-app}
	\vec{Q} = \vec{P} + k\vec{n}
	\\
	\implies PQ = \norm{\vec{Q} - \vec{P}}=\abs{k} \norm{\vec{n}}
			\label{eq:geo-param-app-PQ}
\end{align}
is the distance from $\vec{Q}$
to the line in 
			\eqref{eq:geo-norm-app}.
			From \eqref{eq:geo-param-app},
\begin{align}
	\vec{n}^{\top}  \vec{Q} = \vec{n}^{\top}  \vec{P} + k\norm{\vec{n}}^2
	\\
	\implies \abs{k} = 
	\frac{\abs{\vec{n}^{\top}\brak{\vec{Q} - \vec{P}}}}{\norm{\vec{n}}^2}
			\label{eq:geo-param-app-k}
			\\
	\implies PQ =\abs{k}  
		\norm{\vec{n}}	=
	\frac{\abs{\vec{n}^{\top}\vec{P} - c}}{\norm{\vec{n}}}
			\label{eq:PQ-final}
\end{align}
upon substituting from 
			\eqref{eq:geo-param-app-PQ}.
\item The foot of the perpendicular is given by
\begin{align}
	\label{eq:11/10/3/4/foot_of_perpendicular}
	\myvec{\vec{m} & \vec{n}}^\top\vec{Q} &= 
	   \myvec{
              \vec{m}^\top\vec{P}\\
	      c
	      }
\end{align}
\item The distance between the parallel lines 
\begin{align}
	\label{eq:parallel_lines}
	\begin{split}
		\vec{n}^{\top}\vec{x} &= c_1
		\\
		\vec{n}^{\top}\vec{x} &= c_2
	\end{split}
\end{align}
is given by 
\begin{align}
	\label{eq:dist_lines_2d}
	d = \frac{\abs{   c_1-c_2 }}{\norm{\vec{n}}}	
\end{align}
	\item 
The reflection of point $\vec{Q}$ w.r.t a line is given by
\begin{align}
	\label{eq:11/10/4/22}
\vec{R} = \vec{Q} -\frac{2\brak{\vec{n}^{\top}\vec{Q}-c}}{\norm{\vec{n}}}\vec{n}
\end{align}
\end{enumerate}
