\begin{enumerate}[label=\thesubsection.\arabic*.,ref=\thesubsection.\theenumi]
\item Let the perpendicular distance from the origin to a line be $p$ and the angle made by the perpendicular with the positive $x$-axis be $\theta$.
	Then 
\begin{align}
	p\myvec{\cos \theta \\ \sin \theta}
\end{align}
is a point on the line as well as the normal vector.
Hence, the equation of the line is 
\begin{align}
	p\myvec{\cos \theta & \sin \theta}
	\cbrak{\vec{x}-p\myvec{\cos \theta \\ \sin \theta}} &= 0
	\\
	\implies 
	\myvec{\cos \theta & \sin \theta}
	\vec{x} &= p
\label{eq:chapters/11/10/2/8-final}
\end{align}
\item Let $\vec{Q}$ be the foot of the perpendicular from $\vec{P}$
	to the line
\begin{align}
			\label{eq:geo-norm-app}
    \vec{n}^{\top}  \vec{x} = c
\end{align}
From
			\eqref{eq:geo-param}
\begin{align}
			\label{eq:geo-param-app}
	\vec{Q} = \vec{P} + k\vec{n}
	\\
	\implies PQ = \norm{\vec{Q} - \vec{P}}=\abs{k} \norm{\vec{n}}
			\label{eq:geo-param-app-PQ}
\end{align}
is the distance from $\vec{Q}$
to the line in 
			\eqref{eq:geo-norm-app}.
			From \eqref{eq:geo-param-app},
\begin{align}
	\vec{n}^{\top}  \vec{Q} = \vec{n}^{\top}  \vec{P} + k\norm{\vec{n}}^2
	\\
	\implies \abs{k} = 
	\frac{\abs{\vec{n}^{\top}\brak{\vec{Q} - \vec{P}}}}{\norm{\vec{n}}^2}
			\label{eq:geo-param-app-k}
			\\
	\implies PQ =\abs{k}  
		\norm{\vec{n}}	=
	\frac{\abs{\vec{n}^{\top}\vec{P} - c}}{\norm{\vec{n}}}
			\label{eq:PQ-final}
\end{align}
upon substituting from 
			\eqref{eq:geo-param-app-PQ}.
\item The foot of the perpendicular is given by
\begin{align}
	\label{eq:11/10/3/4/foot_of_perpendicular}
	\myvec{\vec{m} & \vec{n}}^\top\vec{Q} &= 
	   \myvec{
              \vec{m}^\top\vec{P}\\
	      c
	      }
\end{align}
\item The distance between the parallel lines 
\begin{align}
	\label{eq:parallel_lines}
	\begin{split}
		\vec{n}^{\top}\vec{x} &= c_1
		\\
		\vec{n}^{\top}\vec{x} &= c_2
	\end{split}
\end{align}
is given by 
\begin{align}
	\label{eq:dist_lines_2d}
	d = \frac{\abs{   c_1-c_2 }}{\norm{\vec{n}}}	
\end{align}
\item Find the equation of the line passing through the point (5,2) and perpendicular to the line joining the points (2,3) and (3, -1).
\item Find the points on the line $x+y=4$ which lie at a unit distance from the line $4x+3y=10$.
\item Find the equation of a straight line on which length of perpendicular from the origin is four units and the line makes on angle of 120$\degree$ with the positive direction of $x$-axis. [\textbf{Hint} : Use normal form, here $\omega =30\degree$.]
\item Find the equation of one of the sides of an isosceles right angled triangle whose hypotenuse is given by $3x+4y=4$ and the opposite vertex of the hypotenuse is (2,2).
\item In what direction should a line be drawn through the point (1,2) so that its point of intersection with line $x+y=4$ is at a distance $\sqrt{6}{3}$ from the given equilateral    
\item The equation of the straight line passing through the point (3,2) and perpendicular to the line $y=x$ is
\begin{enumerate}
\item $x-y=5$
\item $x+y=5$
\item $x+y=1$
\item $x-y=1$
\end{enumerate}
\item The equation of the line passing through the point (1,2) and perpendicular to the line $x+y+1=0$ is
\begin {enumerate}
\item $y-x+1=0$
\item $y-x-1=0$
\item $y-x+2=0$
\item $y-x-1=0$
\end{enumerate}
\item The distance of the point of intersection of the lines $2x-3y+5=0 \text{ and }3x+4y=0$ from the line $5x-2y=0$ is
\begin{enumerate}
\item $\frac{130}{17\sqrt{29}}$
\item $\frac{13}{7\sqrt{29}}$
\item $\frac{130}{7}$
\item none of these
\end{enumerate}
\item The equations of the lines passing through the point (1,0) and at a distance $\frac{\sqrt{3}}{2}$ from the origin, are 
\begin{enumerate}
\item $\sqrt{3}x+y-\sqrt{3}=0$, $\sqrt{3}x-y-\sqrt{3}=0$
\item $\sqrt{3}x+y+\sqrt{3}=0$, $\sqrt{3}x-y+\sqrt{3}=0$
\item $x+\sqrt{3}y-\sqrt{3}=0$, $\sqrt{3}y-\sqrt{3}=0$
\item None of these.
\end{enumerate}
\item The distance between the lines $y=mx+c$,\text{ and }$y=mx+c^2$ is
\begin{enumerate}
\item $\frac{c_1-c_2}{\sqrt{m+1}}$
\item $\frac{\abs{c_1-c_2}}{\sqrt{1+m^2}}$
\item $\frac{c^2-c^1}{\sqrt{1+m^2}}$
\item 0
\end{enumerate}
\item The coordinates of the foot of perpendiculars from the point (2,3) on the line $y=3x+4$ is given by 
\begin{enumerate} 
\item $\frac{37}{10}$, $\frac{-1}{10}$
\item $\frac{-1}{10}$, $\frac{37}{10}$
\item $\frac{10}{37}$, -10
\item $\frac{2}{3}$, $\frac{-1}{3}$
\end{enumerate}
\item A point equidistant from the lines $4x+3y+10=0$, $5x-12y+26=0$ and $7x+24y-50=0$ is
\begin{enumerate}
\item (1,-1)
\item (1,1)
\item (0,0)
\item (0,1)
\end{enumerate}
\item A line passes through (2,2) and is perpendicular to the line $3x+y=3$. Its $y$-intercept is 
\begin{enumerate}
\item $\frac{1}{3}$
\item $\frac{2}{3}$
\item 1
\item $\frac{4}{3}$
\end{enumerate}
\item The ratio in which the line $3x+4y+2=0$ divides the distance between the lines $3x+4y+5=0$ and $3x+4y-5=0$ is
\begin{enumerate}
\item 1:2
\item 3:7
\item 2:3
\item 2:5
\end{enumerate}
\end{enumerate}
