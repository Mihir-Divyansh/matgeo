\begin{enumerate}[label=\thesubsection.\arabic*,ref=\thesubsection.\theenumi]
\item
  If $L$ in \eqref{eq:conic_tangent} touches \eqref{eq:conic_quad_form} at exactly one point $\vec{q}$, 
  \begin{align}
  \vec{m}^{\top}\brak{\vec{V}\vec{q}+\vec{u}} = 0
  \label{eq:conic_tangent_mq}
  \end{align}
\begin{proof}
  In this case, \eqref{eq:conic_intercept} has exactly one root.  Hence, 
  in \eqref{eq:tangent_roots}
  \begin{align}
  \sbrak{
  \vec{m}^{\top}\brak{\vec{V}\vec{q}+\vec{u}}
  }^2 -\brak{\vec{m}^{\top}\vec{V}\vec{m}}
	  \text{g}\brak
  {
  \vec{q}
%  \vec{q}^{\top}\vec{V}\vec{q} + 2\vec{u}^{\top}\vec{q} +f
  } = 0                                                                                             
  \label{eq:conic_tangent_disc}
  \end{align}                    
  $\because \vec{q}$ is the point of contact,
	%$\vec{q}$ satisfies \eqref{eq:conic_quad_form}
%  and 
  \begin{align}
	  \text{g}\brak{  \vec{q}} = 0
%  \vec{q}^{\top}\vec{V}\vec{q} + 2\vec{u}^{\top}\vec{q} +f = 0
  \label{eq:conic_tangent_qquad}
  \end{align}
  Substituting \eqref{eq:conic_tangent_qquad} in \eqref{eq:conic_tangent_disc} and simplifying, we obtain \eqref{eq:conic_tangent_mq}.
\end{proof}
\item For a circle, the points of contact are
	\begin{align}
	\vec{q}_{ij} &= \brak{\pm r \frac{\vec{n}_j}{\norm{\vec{n}_j}}-\vec{u}}, \quad i,j = 1,2
\label{eq:conic_tangent_qk-circ}
\end{align}
\begin{proof}
	From 
\eqref{eq:conic_tangent_qk},
and 
	\eqref{eq:circ-cr},
\begin{align}
\kappa_{ij} &= \pm 
\frac{r
}
{
	\norm{\vec{n}_j}
}
\end{align}
\end{proof}

\end{enumerate}
