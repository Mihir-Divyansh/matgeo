\begin{enumerate}[label=\thesection.\arabic*,ref=\thesection.\theenumi]
\numberwithin{equation}{enumi}
\numberwithin{figure}{enumi}
\numberwithin{table}{enumi}

\item $ABCD$ is a quadrilateral in which $AB = BC$ and $AD = CD$. Show that $BD$ bisects both the angles $ABC$ and $ADC$.
\item $O$ is a point in the interior of a square $ABCD$ suchthat $OAB$ is an equilateral triangle. Show that $\triangle  OCD$ is an isosceles triangle.
\item Show that in a quadrilateral $ABCD$, 
\begin{align}
     AB + BC + CD + DA  <  (BD + AC)
\end{align} 
\item Show that in a quadrilateral $ABCD$,
\begin{align}
 AB + BC + CD + DA  >   AC + BD
\end{align}
\item Line segment joining the mid-points $M$ and $N$ of parallel sides $AB$ and $DC$, respectively of a trapezium $ABCD$ is perpendicular to both the sides $AB$ and $DC$. Prove that $AD = BC$.
\item $ABCD$ is a quadrilateral such that diagonal $AC$ bisects the angles $A$ and $C$. Prove that $AB = AD$ and $CB = CD$.
\item $AB$ and $CD$ are the smallest and largest sides of a quadrilateral $ABCD$. Out of $\angle B$ and $\angle D$ decide which is greater.
\item $ABCD$ is quadrilateral such that $AB = AD$ and $CB = CD$. Prove that $AC$ is the perpendicular bisector of $BD$.
\item A point $\vec{E} $ is taken on the side $BC$ of a parallelogram ABCD.$AE$ and  $DC$ are produced to meet at $\vec{F}$.Prove that  $ar (ADF) = ar (ABFC)$.
\item The diagonals of a parallelogram ABCD intersect at a point $\vec{O}$.Through $\vec{O}$,a line is drawn to intersect $AD$ at $\vec{P}$ and $BC$ at $\vec{Q}$.Show that $PQ$ divides the parallelogram into two parts of equal area.
\item The medians $BE$ and $CF$ of a triangle ABC intersect at $\vec{G}$.Prove that the area of $ \triangle${GBC}= area of the quadrilateral AFGE.	
\item In Fig.\ref{fig:exemplar/9.9.4/9.24},$CD \parallel AE$  and $CY \parallel BA$.Prove that  $ar (CBX) =  ar (AXY)$
\begin{figure}[h]
	\centering
	\includegraphics[width=\columnwidth]{exemplar/9.9.4/figs/Fig9.24.png}
	\caption{}
	\label{fig:exemplar/9.9.4/9.24}
\end{figure}
\item ABCD is a trapezium in which $AB \parallel DC$,$DC = 30cm$  and $AB = 50cm$.If $\vec{X}$ and $\vec{Y}$ are,respectively the mid-points of $AD$ and $BC$,prove that  $ar (DCYX) = \frac{7}{9} ar (XYBA)$.
\item  In $ \triangle${ABC},if $\vec{L}$ and $\vec{M}$ are the points on $AB$ and $AC$,respectively such that $LM \parallel BC$.Prove that $ar (LOB) = ar (MOC)$.
\item In Fig.\ref{fig:exemplar/9.9.4/9.25},ABCDE is any pentagon.$BP$ drawn parallel to $AC$ meets $DC$ produced at $\vec{P}$ and $EQ$ drawn parallel to $AD$ meets $CD$ produced at $\vec{Q}$.Prove that  $ ar (ABCDE) = ar (APQ) $.
\begin{figure}[h]
	\centering
	\includegraphics[width=\columnwidth]{exemplar/9.9.4/figs/Fig9.25.png}
	\caption{}
	\label{fig:exemplar/9.9.4/9.25}
\end{figure}
\item If the medians of a $ \triangle$ ABC  intersect at $\vec{G}$,show that
	\begin{align} 
		{ar (AGB)} &={ar (AGC)}= {ar (BGC)} = \frac{1}{3} {ar (ABC)}
	\end{align}
\item In Fig.\ref{fig:exemplar/9.9.4/9.26},$\vec{X}$ and $\vec{Y}$ are the mid-points of $AC$ and $AB$ respectively,$QP \parallel BC$ and $CYQ$ and $BXP$ are straight lines.Prove that $ ar (ABP) = ar (ACQ) $.
\begin{figure}[h]
	\centering
	\includegraphics[width=\columnwidth]{exemplar/9.9.4/figs/Fig9.26.png}
	\caption{}
	\label{fig:exemplar/9.9.4/9.26}
\end{figure}
\item In Fig.\ref{fig:exemplar/9.9.4/9.27},ABCD and AEFD are two parallelograms.Prove that $ ar (PEA) = ar (QFD) $ [Hint:Join PD].
\begin{figure}[h]
	\centering
	\includegraphics[width=\columnwidth]{exemplar/9.9.4/figs/Fig9.27.png}
	\caption{}
	\label{fig:exemplar/9.9.4/9.27}
\end{figure}
	\item ABCD is a parallelogram and $\vec{X}$ is the mid-point of AB.If $ ar(AXCD)= 24 cm^2 $ ,then $ar(ABC) =  24cm^2 $.
\item PQRS is a rectangle inscribed in a quadrant of radius 13 cm.A is any point on PQ.If PS=5 cm,then $ar(PAS)= 30 cm^2 $
\item PQRS is a parallelogram whose area is $ 180 cm^2 $ and A is any point on the diagonal QS.The area of $\triangle ASR =90 cm^2$.
\item ABC and BDE are two equilateral triangles such that $\vec{D}$is the mid-point of BC.Then ar(BDE)=$\frac{1}{4}  ar(ABC)$.
\item In Fig.\ref{fig:exemplar/9.8/9.8}, $ABCD$ and $EFGD$ are two parallelograms and $\vec{G}$ is the mid-point of $CD$. Then$ ar(DPC)=\frac{1}{2}  ar(EFGD)$.
	\begin{figure}[h]
		\centering
		\includegraphics[width=\columnwidth]{exemplar/9.9.2/figs/9.8.jpg}
		\caption{}
		\label{fig:exemplar/9.8/9.8}
	\end{figure}
\item Construct a square of side $3 cm$.
\item Construct  a rectangle whose adjacent sides are of lengths $5 cm$ and $3.5 cm$.
\item Construct a rhombus whose side is of length $3.4 cm$ and one of its angles is $45\degree$.
\item Construct a rhombus whose diagonals are 4 cm and 6 cm in lengths.
\end{enumerate}
