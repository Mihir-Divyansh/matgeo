\section{$2\times 1$ vectors}

%\begin{enumerate}[label=\arabic*.,ref=\theenumi]
\begin{enumerate}[label=\thesection.\arabic*.,ref=\thesection.\theenumi]
\numberwithin{equation}{enumi}
	\item Mathematically, 
the projection of $\vec{A}$ on $\vec{B}$ is defined as
		\begin{align}
	\vec{C} = k \vec{B},\, \text{such that}
	\brak{\vec{A}-\vec{C}}^{\top}\vec{C} = 0
\end{align}
yielding
\begin{align}
	\brak{\vec{A}-k\vec{B}}^{\top}\vec{B} = 0
	\\
	\text{or, } k = 
	\frac{\vec{A}^{\top}\vec{B}}{\norm{\vec{B}}^2}
	\implies 
	\vec{C} = 
	\frac{\vec{A}^{\top}\vec{B}}{\norm{\vec{B}}^2}
 \vec{B}
	\label{eq:12/10/3/4/proj}
\end{align}
\item If $\vec{A}, \vec{B}$ are unit vectors, 
\begin{multline}
	\brak{\vec{A}-\vec{B}}^{\top} 
	\brak{\vec{A}+\vec{B}} 
	\\
\norm{\vec{A}}^2 - \norm{\vec{B}}^2
	= 0
	\label{eq:12/10/3/11/unit}
\end{multline}
  \item If $ABCD$ be a parallelogram,
	  \label{prop:two-pgm}
  \begin{align}
	  \label{eq:two-pgm}
 \vec{B}-\vec{A} = \vec{C} -\vec{D}
  \end{align}
  \item 
If $PQRS$ is formed by joining the mid points of $ABCD$, 
\begin{align}
  \vec{P} = \frac{1}{2}\brak{\vec{A}+\vec{B}} 
  ,\,
 \vec{Q} = \frac{1}{2}\brak{\vec{B}+\vec{C}} 
 \\
 \vec{R} = \frac{1}{2}\brak{\vec{C}+\vec{D}}   
  ,\,
 \vec{S} = \frac{1}{2}\brak{\vec{D}+\vec{A}}  
 \\
	\implies 
 \vec{P}-\vec{Q} = \vec{S} -\vec{R}.
  \label{eq:10/7/4/8det2f}
\end{align}
Hence, $PQRS$ is a parallelogram
	  from \eqref{eq:two-pgm}.
  \item If 
\begin{align}
	\vec{A}^{\top}\vec{A} =\vec{I},
\label{eq:12/10/3/5/inner}
\end{align}
		then $	\vec{A}$ is an {\em orthogonal} matrix.
\item Let 
\begin{align}
  \vec{A} &= \myvec{a_1\\a_2 \\ a_3} \equiv a_1\overrightarrow{i}+a_2\overrightarrow{j}+a_3\overrightarrow{j}, 
  \\
  \vec{B} &= \myvec{b_1\\b_2 \\ b_3}, 
\end{align}
and 
\begin{align}
  \label{eq:cross3d-submat}
\begin{split}
  \vec{A}_{ij} &= \myvec{a_i\\a_j}, 
  \\
  \vec{B}_{ij} &= \myvec{b_i\\b_j}. 
\end{split}
\end{align}

\item The {\em cross product} or {\em vector product} of $\vec{A}, \vec{B}$ is defined as
\begin{align}
  \label{eq:cross3d}
	\vec{A} \times \vec{B} 
	 = \myvec{ \mydet{\vec{A}_{23} & \vec{B}_{23}} \\[1ex] \mydet{\vec{A}_{31} & \vec{B}_{31}} \\[1ex] \mydet{\vec{A}_{12}  & \vec{B}_{12}}}
\end{align}
\item Verify that
\begin{align}
  \label{eq:cross3d-commute}
  \vec{A} \times \vec{B} = -  \vec{B} \times \vec{A} 
  \\
  \label{eq:cross3d-same}
  \vec{A} \times \vec{A} = \vec{0}
\end{align}
\item If 
		\label{prop:lin-dep-cross}
\begin{align}
  \vec{A} \times \vec{B} = \vec{0},
\end{align}
  $\vec{A}$ and $ \vec{B} $ are linearly independent.
  \item 
\begin{align}
	\label{eq:cross-sin}
	\norm{ \vec{A} \times \vec{B} }
	=
	\norm{\vec{A}} \times 	\norm{\vec{B}} \sin \theta
\end{align}
where $\theta$ is the angle between the vectors.
\item 
\begin{align}
	ar\brak{ABCD} = 
         \frac{1}{2}\brak{\brak{\vec{C}-\vec{A}}\times\brak{\vec{D}-\vec{B}}} \\
        \label{eq:11/10/1/1area-diag} 
\end{align}
\item Construct a $\triangle ABC$ given $a, \angle B$ and $K = b+c$.
		\label{prob:9/11/2/1}
	\\
	\solution 
	Using the cosine formula in  $\triangle ABC$,
\begin{align}
	{b}^2&= {a}^2 + {c}^2 - 2ac\cos{B}
\\
\implies	(K-c)^2 &= {a}^2 + c^2- 2  a  c\cos{B}
\\
\implies
	c &=
	\frac{k^2-a^2}{2\brak{k- a  \cos{B}}}
		\label{eq:9/11/2/1}
\end{align}
The coordinates of $\triangle ABC$ can then be expressed as
\begin{align}
		\label{eq:9/11/2/1-final}
	\vec{A}=c\myvec{\cos B \\ \sin B},
	\vec{B} = \vec{0},
	\vec{C} =\myvec{a \\ 0}.
\end{align}
\end{enumerate}
\iffalse
\item (Cauchy-Schwarz Inequality)
    \begin{align}
        \label{eq:dot-mag-ineq}
	    \abs{\vec{a}^\top\vec{b}} &\le \norm{\vec{a}}\norm{\vec{b}}
    \end{align}
    \solution
	\begin{align}
        \norm{\vec{a}-\frac{\vec{a}^\top\vec{b}}{\norm{\vec{b}}^2}\vec{b}}^2 &\ge 0 \\
        \implies \norm{\vec{a}}^2 - 2\frac{\brak{\vec{a}^\top\vec{b}}^2}{\norm{\vec{b}}^2} + \frac{\brak{\vec{a}^\top\vec{b}}^2}{\norm{\vec{b}}^2} &\ge 0 \\
        \implies \norm{\vec{a}}^2 - \frac{\brak{\vec{a}^\top\vec{b}}^2}{\norm{\vec{b}}^2} &\ge 0 \\
        \implies \norm{\vec{a}}^2\norm{\vec{b}}^2 &\ge \brak{\vec{a}^\top\vec{b}}^2 \\
    \end{align}
    yielding
        \eqref{eq:dot-mag-ineq}.
\item (Triangle Inequality)
    \begin{align}
\norm{\vec{a}+\vec{b}} &\le \norm{\vec{a}}+\norm{\vec{b}}
        \label{eq:triangle-ineq}
    \end{align}
    \solution
    Using \eqref{eq:dot-mag-ineq},
    \begin{align}
        \vec{a}^\top\vec{b} &\le \norm{\vec{a}}\norm{\vec{b}} \\
\implies        \norm{\vec{a}}^2 + 2\vec{a}^\top\vec{b} + \norm{\vec{b}}^2 &\le \norm{\vec{a}}^2 + 2\norm{\vec{a}}\norm{\vec{b}} + \norm{\vec{b}}^2 \\
\implies               \norm{\vec{a}+\vec{b}}^2 &\le \brak{\norm{\vec{a}}+\norm{\vec{b}}}^2 
    \end{align}
    yielding
        \eqref{eq:triangle-ineq}.
\end{enumerate}

%\renewcommand{\theequation}{\theenumi}
%\begin{enumerate}[label=\arabic*.,ref=\theenumi]
\begin{enumerate}[label=\thesection.\arabic*.,ref=\thesection.\theenumi]
%\begin{enumerate}[1.]
%\begin{enumerate}
%\numberwithin{equation}{enumi}
\item Let 
\begin{align}
  \vec{A} \equiv \overrightarrow{A} &= \myvec{a_1\\a_2} 
  \\
  &\equiv a_1\overrightarrow{i}+a_2\overrightarrow{j}, 
  \\
  \vec{B} &= \myvec{b_1\\b_2}, 
\end{align}
be $2 \times 1$ vectors.
Then, the determinant of the $2 \times 2$ matrix 
\begin{align}  
  \vec{M} = \myvec{\vec{A} & \vec{B}}
\end{align}
is defined as
\begin{align}
  \label{eq:det2d}
  \mydet{\vec{M}} &= \mydet{\vec{A} & \vec{B}} 
  \\
  &= \mydet{a_1 & b_1\\a_2 & b_2} = a_1b_2 - a_2 b_1
\end{align}
%
\item The area of the triangle with vertices $\vec{A}, \vec{B}, \vec{C}$ is given by 
	\label{prop:area2d}
\begin{align}
  \label{eq:area2d}
	\frac{1}{2}\norm{\brak{\vec{A}-\vec{B}} \times \brak{\vec{A}-\vec{C}}}
 = 
 \frac{1}{2}\norm{\vec{A} \times \vec{B}+\vec{B} \times \vec{C}+\vec{C} \times \vec{A}}
  \end{align}
  \item If 
  \label{prop:area2d-norm}
\begin{align}
  \label{eq:area2d-norm}
	\norm{\vec{A}\times\vec{B}}  &= \norm{\vec{C}\times \vec{D}}, \quad \text{then}
	\\
	\vec{A}\times\vec{B}  &= \pm\brak{\vec{C}\times \vec{D}}
  \end{align}
  where the sign depends on the orientation of the vectors.
  \item The median divides the triangle into two triangles of equal area.
	  \label{prop:two-median-area}
  \item  The transpose of $\vec{A}$ is defined as
\begin{align}
  \label{eq:transpose2d}
  \vec{A}^{\top}  = \myvec{a_1 & a_2}
\end{align}
%
\item The {\em inner product} or {\em dot product} is defined as
  \label{prop:dot2d}
\begin{align}
  \label{eq:dot2d}
  \vec{A}^{\top} \vec{B} &\equiv \vec{A} \cdot \vec{B} 
  \\
  &= \myvec{a_1 & a_2} \myvec{b_1 \\ b_2}= a_1b_1+a_2b_2 
\end{align}
%
\item {\em norm} of $\vec{A}$ is defined as
\begin{align}
  \label{eq:norm2d}
  \norm{A} &\equiv \mydet{\overrightarrow{A}}
  \\
  &= \sqrt{\vec{A}^{\top} \vec{A}}= \sqrt{a_1^2+a_2^2}
\end{align}
Thus, 
\begin{align}
  \label{eq:norm2d_const}
  \norm{\lambda \vec{A}} &\equiv \mydet{\lambda\overrightarrow{A}}
  \\
  &= \abs{\lambda} \norm{\vec{A}}
\end{align}
\item The distance betwen the points $\vec{A}$ and $\vec{B}$ is given by 
\begin{align}
  \label{eq:norm2d_dist}
\norm{\vec{A}-\vec{B}} 
\end{align}
\item Let $\vec{x}$ be equidistant from the points $\vec{A}$ and $\vec{B}$.  Then 
  \begin{align}
	  \brak{\vec{A}-\vec{B}}^{\top}{\vec{x}} 
	  =  \frac{\norm{\vec{A}}^2 - \norm{\vec{B}}^2}{2}
  \label{eq:norm2d_equidist}
  \end{align}
  \solution 
\begin{align}
	\norm{\vec{x}-\vec{A}} &=
\norm{\vec{A}-\vec{B}} 
\\
	\implies \norm{\vec{x}-\vec{A}}^2 &=
\norm{\vec{x}-\vec{B}}^2 
\end{align}
which can be expressed as 
\begin{multline}
%  \label{eq:norm2d_dist}
	\brak{\vec{x}-\vec{A}}^{\top} \brak{\vec{x}-\vec{A}}=
	\brak{\vec{x}-\vec{B}}^{\top} 
\brak{\vec{x}-\vec{B}}
\\
	\implies	\norm{\vec{x}}^2-2{\vec{x}}^{\top}\vec{A} + \norm{\vec{A}}^2
	\\= \norm{\vec{x}}^2-2{\vec{x}}^{\top}\vec{B} + \norm{\vec{B}}^2
\end{multline}
which can be simplified to obtain
  \eqref{eq:norm2d_equidist}.
\item If $\vec{x}$ lies on the  $x$-axis and is  equidistant from the points $\vec{A}$ and $\vec{B}$, 
  \begin{align}
	  \vec{x} &=
	   x\vec{e}_1
  \end{align}
  where 
  \begin{align}
	  x &=\frac{\norm{\vec{A}}^2 -\norm{\vec{B}}^2 }{2\brak{\vec{A}-\vec{B}}^{\top }\vec{e}_1
}
	  \label{eq:cbse_10_x}
  \end{align}
  \solution 
  From \eqref{eq:norm2d_equidist}.
  \begin{align}
	   x\brak{\vec{A}-\vec{B}}^{\top }\vec{e}_1
		  &=
	  \frac{\norm{\vec{A}}^2 -\norm{\vec{B}}^2 }{2}
   \end{align}
	  yielding \eqref{eq:cbse_10_x}.
  \item The angle between two vectors is given by 
    \label{prop:angle2d}
  \begin{align}
%    \label{eq:angle2d}
    \theta = \cos^{-1}\frac{\vec{A}^{\top} \vec{B}}{\norm{A}\norm{B}}
  \end{align}
  \item If two vectors are orthogonal (perpendicular), 
  \begin{align}
    \label{eq:angle2d_orth}
\vec{A}^{\top} \vec{B} = 0
  \end{align}
  \item For an isoceles triangle $ABC$ ith $AB = AC$, the median $AD \perp BC$.
    \label{prop:two-isosc}
%  \begin{align}
%    \label{eq:two-isosc}
%\vec{A}^{\top} \vec{B} = 0
%  \end{align}

  \item The {\em direction vector} of the line joining two points $\vec{A},\vec{B}$ is given by 
  \begin{align}
    \label{eq:dir_vec}
    \vec{m} = \vec{A}-\vec{B}
  \end{align}
  \item The points $\vec{A}\vec{A}\vec{A}$
\item The unit vector in the direction of $\vec{m}$ is defined as
\begin{align}
    \frac{\vec{m}}{\norm{\vec{m}}}
\end{align}
\item If the direction vector of a line is expressed as 
		\label{prop:two-dir-vec}
	\begin{align}
		\label{eq:two-dir-vec}
    \vec{m} = \myvec{1\\m},
\end{align}
 the $m$ is defined to be the {\em} slope of the line. 
  \item $AB \parallel CD$ if 
	  \label{prop:two-par-dir-vec}
  \begin{align}
	  \vec{A}- \vec{B}= k\brak{\vec{C}- \vec{D}}
	  \label{eq:two-par-dir-vec}
  \end{align}
  \item The {\em normal vector} to $\vec{m}$ is defined by 
  \begin{align}
    \label{eq:normal_vec}
    \vec{m}^{\top}  \vec{n} = 0
  \end{align}
  \item  If
	  \label{prop:two-orth-para}
\begin{align}
	\vec{m}^{\top}  \vec{n}_1 &= 0
	\\
	\vec{m}^{\top}  \vec{n}_2 &= 0,
	\\
	\vec{n}_1 &\parallel \vec{n}_2
	  \label{eq:two-orth-para}
\end{align}
  \item The point $\vec{P}$ that divides the line segment $AB$ in the ratio $k:1$  is given by 

  \begin{align}
	  \vec{P}&= \frac{k\vec{B}+ \vec{A}}{k+1}
%	  \label{eq:section_formula}
  \end{align}
\item  The standard basis vectors are defined as 
	\label{def:matrix-two}

  \begin{align}
  \vec{e}_1&= \myvec{1\\0}, 
  \\
  \vec{e}_2&= \myvec{0\\1}.
  \end{align}
  \item Diagonals of a parallelogram bisect each other.
	  \label{prop:two-pgm-diag-bisect}
\item The area of the parallelogram with vertices $\vec{A}, \vec{B}, \vec{C}$ and $\vec{D}$ is given by 
  \label{prop:pgm2d}
\begin{align}
  \label{eq:pgm2d}
	\norm{\brak{\vec{A}-\vec{B}} \times \brak{\vec{A}-\vec{D}}}
 = 
 \norm{\vec{A} \times \vec{B}+\vec{B} \times \vec{C}+\vec{C} \times \vec{A}}
  \end{align}
  \item Points $\vec{A},\vec{B}$ and $\vec{C}$ form a triangle  if 
	  \label{prop:two-tri-indep}
  \begin{align}
	  p\brak{\vec{A}- \vec{B}} +q\brak{\vec{A} -\vec{C}} &= 0
	  \\
	  \label{eq:two-tri-indep}
	  \text{or, }\brak{p+q}\vec{A}- p\vec{B} -q\vec{C} &= 0
	  \\
	  \implies p=0, q=0
  \end{align}
  are linearly independent.
  \item In $\triangle ABC$, if $\vec{D}, \vec{E}$ divide the lines $AB, AC$ in the ratio $k:1$ respectively,  then $DE \parallel BC$.
	  \label{prop:two-tri-bpt}
	  \begin{proof}
		  From 
	  \eqref{eq:section_formula}, 
  \begin{align}
	  \vec{D}&= \frac{k\vec{B}+ \vec{A}}{k+1}
	  \\
	  \vec{E}&= \frac{k\vec{C}+ \vec{A}}{k+1}
	  \\
	  \implies 
	  \vec{D}-	  \vec{E}&= \frac{k}{k+1}\brak{\vec{B}- \vec{C}}
  \end{align}
  Thus, from 
		  Appendix \ref{prop:two-dir-vec}, $DE \parallel BC$.

	  \end{proof}

  \item In $\triangle ABC$, if $DE \parallel BC$, $\vec{D}$ and $\vec{E}$ divide the lines $AB, AC$ in the same ratio.  
	  \label{prop:two-tri-bpt-conv}
	  \begin{proof}
If $DE \parallel BC$,
		  from 
 \eqref{eq:two-par-dir-vec}
  \begin{align}
	  \label{prop:two-tri-bpt-conv-1}
	  \brak{\vec{B}- \vec{C}} = k\brak{\vec{D}-	  \vec{E}}
  \end{align}
Using   
	  \eqref{eq:section_formula}, 
let 
  \begin{align}
	  \vec{D}&= \frac{k_1\vec{B}+ \vec{A}}{k_1+1}
	  \\
	  \vec{E}&= \frac{k_2\vec{C}+ \vec{A}}{k_2+1}
  \end{align}
	  Subtituting the above in 
	  \eqref{prop:two-tri-bpt-conv-1}, after some algebra, we obtain 
	
  \begin{align}
\brak{p+q}\vec{A}- p\vec{B} -q\vec{C} &= 0
  \end{align}
  where
  \begin{align}
	  p = \frac{1}{k} -  \frac{k_1}{k_1+1},
	  q = \frac{1}{k} -  \frac{k_1}{k_1+1}
  \end{align}
  %
From 	  
	  \eqref{eq:two-tri-indep},
  \begin{align}
	p = q = 0
	  \\
	  \implies k_1 = k_2  = \frac{1}{k-1}
  \end{align}

	  \end{proof}
\end{enumerate}
\section{Eigenvalues and Eigenvectors}
%\renewcommand{\theequation}{\theenumi}
%\begin{enumerate}[label=\arabic*.,ref=\theenumi]
\begin{enumerate}[label=\thesection.\arabic*.,ref=\thesection.\theenumi]
%\begin{enumerate}
%\numberwithin{equation}{enumi}
\item The eigenvalue $\lambda$ and the eigenvector $\vec{x}$  for a matrix $\vec{A}$ are defined as, 
\begin{align}
  \vec{A} \vec{x} = \lambda \vec{x}
\end{align}
\item The eigenvalues are calculated by solving the
equation
\begin{align}
  \label{eq:chareq}
f\brak{\lambda} = \mydet{\lambda \vec{I}- \vec{A} } =0
\end{align}
The above equation is known as the characteristic equation.
\item According to the Cayley-Hamilton theorem,
\begin{align}
	\label{eq:cayley}
  f(\lambda) = 0 \implies f\brak{\vec{A}} = 0
\end{align}
\item The trace of a square  matrix is defined to be the sum of the diagonal elements.
\begin{align}
	\label{eq:trace}
	\text{tr}\brak{\vec{A}}=\sum_{i=1}^{N}a_{ii}.
\end{align}
	where $a_{ii}$ is the $i$th diagonal element of the matrix $\vec{A}$. 	
\item The trace of a matrix is equal to the sum of the eigenvalues
\begin{align}
	\label{eq:trace_eig}
	\text{tr}\brak{\vec{A}}=\sum_{i=1}^{N}\lambda_i
\end{align}


\end{enumerate}
\section{Determinants}
%\renewcommand{\theequation}{\theenumi}
%\begin{enumerate}[label=\arabic*.,ref=\theenumi]
\begin{enumerate}[label=\thesection.\arabic*.,ref=\thesection.\theenumi]
%\begin{enumerate}
%\numberwithin{equation}{enumi}

\item Let 
\begin{align}
	\vec{A} = \myvec{a_1 & b_1 & c_1  \\ a_2 & b_2 & c_2  \\ a_3 & b_3 & c_3}.
\end{align}
be a $3 \times 3$ matrix. 
Then, 
\begin{multline}
	\mydet{\vec{A}} = a_1 \myvec{ b_2 & c_2 \\  b_3 & c_3} - a_2\myvec{ b_1 & c_1 \\  b_3 & c_3 }  \\ + a_3\myvec{a_1 & b_1 \\ a_2 & b_2 }.
\end{multline}
\item Let $\lambda_1,\lambda_2, \dots, \lambda_n$ be the eigenvalues of a matrix $\vec{A}$.  Then,   the product of the eigenvalues is equal to the determinant of $\vec{A}$.
\begin{align}
	\mydet{\vec{A}} = \prod_{i=1}^{n}\lambda_i
\end{align}
%
\item 
\begin{align}
	\mydet{\vec{A}\vec{B}} = \mydet{\vec{A}}\mydet{\vec{B}}
\end{align}
\item If $\vec{A}$ be an $n \times n$ matrix, 
\begin{align}
	\label{eq:det_kord}
	\mydet{k\vec{A}} = k^n\mydet{\vec{A}}
\end{align}

\end{enumerate}
\section{Rank of a Matrix}
%\renewcommand{\theequation}{\theenumi}
%\begin{enumerate}[label=\arabic*.,ref=\theenumi]
\begin{enumerate}[label=\thesection.\arabic*.,ref=\thesection.\theenumi]
%\begin{enumerate}
%\numberwithin{equation}{enumi}
\end{enumerate}
\section{Inverse of a Matrix}
%\renewcommand{\theequation}{\theenumi}
%\begin{enumerate}[label=\arabic*.,ref=\theenumi]
\begin{enumerate}[label=\thesection.\arabic*.,ref=\thesection.\theenumi]
%\begin{enumerate}
%\numberwithin{equation}{enumi}
\item For a $2 \times 2$ matrix 
\begin{align}
	\vec{A} = \myvec{a_1 & b_1  \\ a_2 & b_2 },
\end{align}
the inverse is given by 
\begin{align}
	\vec{A}^{-1} = \frac{1}{\mydet{\vec{A}}}\myvec{b_2 & -b_1  \\ -a_2 & a_1 },
\end{align}
\item For higher order matrices, the inverse should be calculated using row operations.
\end{enumerate}
\section{Orthogonality}
%\renewcommand{\theequation}{\theenumi}
%\begin{enumerate}[label=\arabic*.,ref=\theenumi]
\begin{enumerate}[label=\thesection.\arabic*.,ref=\thesection.\theenumi]
%\begin{enumerate}
%\numberwithin{equation}{enumi}
\item The rotation matrix is defined as 
\begin{align}
	\vec{R}_{\theta} = \myvec{\cos \theta & -\sin \theta  \\ \sin \theta  & \cos \theta  }, \quad \theta \in \sbrak{0, 2\pi}
\end{align}
\item The rotation matrix is {\em orthogonal}
\begin{align}
	\vec{R}_{\theta}^{\top}\vec{R}_{\theta} = \vec{R}_{\theta}\vec{R}_{\theta}^{\top} = \vec{I}
\end{align}
\item If the angle of rotation is $\frac{\pi}{2}$,
	\label{prop:mat/rot/pi/2}
\begin{align}
	\vec{m}^{\top}\vec{n} = 0 \implies \vec{n} = \vec{R}_{\frac{\pi}{2}}\vec{m}
\end{align}
\item 
\begin{align}
	\label{eq:mat-nh}
	\vec{n}^{\top}\vec{h} = 1 \implies \vec{n} = \frac{\vec{e}_1}{\vec{e}_1^{\top}\vec{h}}+\mu\vec{R}_{\frac{\pi}{2}}\vec{h}, \quad \mu \in \mathbb{R}.
\end{align}


\item
	The {\em affine} transformation is given by 
    \begin{align}
	    \vec{x} &= \vec{P}\vec{y}+\vec{c} \quad \text{(Affine Transformation)}
\label{eq:conic_affine}
    \end{align}
	where $\vec{P}$ is invertible.

\item
	The eigenvalue decomposition of a symmetric matrix $\vec{V}$ is given by 
	%\cite{banchoff}
    \begin{align}
%      \label{eq:conic_parmas_eig_def}
      \vec{P}^{\top}\vec{V}\vec{P} &= \vec{D}. \quad \text{(Eigenvalue Decomposition)}
      \\
      \vec{D} &= \myvec{\lambda_1 & 0\\ 0 & \lambda_2}, 
      \\
      \vec{P} &= \myvec{\vec{p}_1 & \vec{p}_2}, \quad \vec{P}^{\top}=\vec{P}^{-1},
%      \label{eq:eigevecP}
    \end{align}


\end{enumerate}
\fi
