From the given information, 
\begin{align}
\vec{a}^{\top}\vec{d} &= 0\\
\vec{b}^{\top}\vec{d} &= 0\\
\vec{c}^{\top}\vec{d} &= 15
\end{align}
yielding
\begin{align}
\myvec{\vec{a}^{\top} \\\vec{b}^{\top}\\\vec{c}^{\top}}\vec{d} &= \myvec{0\\0\\15}\\
\implies \myvec{1&4&2 \\3&-2&7 \\2&-1&4}\vec{d} &= \myvec{0\\0\\15}
\label{eq:chapters/12/10/5/12/1}
\end{align}
%
Forming the augmented matrix, 
\begin{align}
	\myvec{1&4&2&\vrule&0\\ 3&-2&7&\vrule&0 \\ 2&-1&4&\vrule&15} 
	\xleftrightarrow[R_3\leftarrow R_3-2R_1]{R_2\leftarrow R_2-3R_1}
	\myvec{1&4&2&\vrule&0\\ 0&-14&1&\vrule&0 \\ 0&-9&0&\vrule&15}
\nonumber	\\
	\xleftrightarrow[]{R_3\leftarrow R_3-\frac{9}{14}R_2}
	\myvec{1&4&2&\vrule&0\\ 0&-14&1&\vrule&0 \\ 0&0&-\frac{9}{14}&\vrule&15}
	\label{eq:chapters/12/10/5/12/2}
\end{align}
yielding
%
\begin{align}
	\vec{d} &= \myvec{\frac{160}{3}\\[1ex]-\frac{5}{3}\\[1ex]-\frac{70}{3}}
\end{align}
upon back substitution.

