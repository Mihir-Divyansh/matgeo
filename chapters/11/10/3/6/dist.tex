\iffalse
\documentclass[12pt]{article}
\usepackage{graphicx}
%\documentclass[journal,12pt,twocolumn]{IEEEtran}
\usepackage[none]{hyphenat}
\usepackage{graphicx}
\usepackage{listings}
\usepackage[english]{babel}
\usepackage{graphicx}
\usepackage{caption}
\usepackage[parfill]{parskip}
\usepackage{hyperref}
\usepackage{booktabs}
%\usepackage{setspace}\doublespacing\pagestyle{plain}
\def\inputGnumericTable{}
\usepackage{color}                                            %%
    \usepackage{array}                                            %%
    \usepackage{longtable}                                        %%
    \usepackage{calc}                                             %%
    \usepackage{multirow}                                         %%
    \usepackage{hhline}                                           %%
    \usepackage{ifthen}
\usepackage{array}
\usepackage{amsmath}   % for having text in math mode
\usepackage{parallel,enumitem}
\usepackage{listings}
\lstset{
language=tex,
frame=single,
breaklines=true
}
%Following 2 lines were added to remove the blank page at the beginning
\usepackage{atbegshi}% http://ctan.org/pkg/atbegshi
\AtBeginDocument{\AtBeginShipoutNext{\AtBeginShipoutDiscard}}
%
%New macro definitions
\newcommand{\mydet}[1]{\ensuremath{\begin{vmatrix}#1\end{vmatrix}}}
\providecommand{\brak}[1]{\ensuremath{\left(#1\right)}}
\providecommand{\abs}[1]{\left\vert#1\right\vert}
\providecommand{\norm}[1]{\left\lVert#1\right\rVert}
\newcommand{\solution}{\noindent \textbf{Solution: }}
\newcommand{\myvec}[1]{\ensuremath{\begin{pmatrix}#1\end{pmatrix}}}
\let\vec\mathbf
\begin{document}
\begin{center}
\title{\textbf{Parallel Lines}}
\date{\vspace{-5ex}} %Not to print date automatically
\maketitle
\end{center}
\setcounter{page}{1}
\section*{11$^{th}$ Maths - Chapter 10}
This is Problem-6 from Exercise 10.3
\begin{enumerate}
\fi
\begin{enumerate}
\item 
The given lines can be expressed as
\begin{align}
\myvec{15&8}\vec{x}&=-34\\
\myvec{15&8}\vec{x}&=31\\
\implies
	\vec{n}=\myvec{15\\8},c_1&=-34,c_2= 31
\end{align}
The distance between them can then be expressed as
\begin{align}
d=\frac{\abs{c_1-c_2}}{\norm{\vec{n}}}
=\frac{\abs{-34-31}}{\sqrt{289}}
=\frac{65}{17}
\end{align}
 \item 
 The given lines can be expressed as
\begin{align}
\myvec{1&1}\vec{x}&=\frac{-p}{l}\\
\myvec{1&1}\vec{x}&=\frac{-r}{l}
\\
\implies
	\vec{n}=\myvec{1\\1}, c_1 &= \frac{-p}{l}, c_2 \frac{-r}{l}
\end{align}
The distance between them is then obtained as
\begin{align}
	d=\frac{1}{l\sqrt{2}}\abs{p-r}
\end{align}
\end{enumerate}

