\begin{enumerate}[label=\thesubsection.\arabic*, ref=\thesubsection.\theenumi]
\item Find the equation of the plane passing through the points $\brak{1, 0, -2}$,  $\brak{3, -1, 0}$ and perpendicular to the plane $2x - y + z = 8$. Also find the distance of the plane thus obtained from the origin.
\hfill (12, 2020)
    \item Find the values of $\lambda$ for which the distance of the point $(2, 1, \lambda)$ from the plane
    \begin{align}
        3x + 5y + 4z = 11
    \end{align}
    is $2\sqrt{2}$ units.
    \hfill (12, 2023)
	\item Find the distance of the point $(a, b, c)$ from the $x$-axis. \hfill (12, 2021)
	\item If the lines $\frac{x-1}{-3} = \frac{y-2}{2\lambda} = \frac{z-3}{2}$ and $\frac{x-1}{3\lambda} = \frac{y-1}{2} = \frac{z-6}{-5}$ are perpendicular, find the value of $\lambda$. Hence, determine whether the lines intersect or not. \hfill (12, 2019)
	\item Find the vector equation of the plane determined by the points $\vec{A}(3, -1, 2)$, $\vec{B}(5, 2, 4)$, and $\vec{C}(-1, -1, 6)$. Hence, find the distance of the plane, thus obtained, from the origin. \hfill (12, 2019)
	\item Find the coordinates of the foot of the perpendicular $\vec{Q}$ drawn from $\vec{P}(3, 2, 1)$ to the plane $2x - y + z + 1 = 0$. Also, find the distance $\vec{P}\vec{Q}$ and the image of the point $\vec{P}$ treating this plane as a mirror. \hfill (12, 2019)
	\item Find the value of $\lambda$ for which the following lines are perpendicular to each other:
	\begin{align*}
	\dfrac{x-5}{5(\lambda+2)} = \dfrac{2-y}{5} = \dfrac{1-z}{-1}; \quad \dfrac{x}{1} = \dfrac{y + \frac{1}{2}}{2\lambda} = \dfrac{z-1}{3}.
	\end{align*}
	Hence, find whether the lines intersect or not. \hfill (12, 2019)
	\item Find the equation of the plane passing through the point $(-1, 3, 2)$ and perpendicular to the planes $x + 2y + 3z = 5$ and $3x + 3y + z = 0$. \hfill (12, 2019)
	Gitem Find the equation of the plane passing through the point $(-1, 3, 2)$ and perpendicular to the planes $x + 2y + 3z = 5$ and $3x + 3y + z = 0$. \hfill (12, 2019)
	
	\item Find the coordinates of the foot $Q$ of the perpendicular drawn from the point $P(1, 3, 4)$ to the plane $2x - y + z + 3 = 0$. Find the distance $PQ$ and the image of $P$ treating the plane as a mirror. \hfill (12, 2019)
\item Find the coordinates of the foot of the perpendicular $\mathbf{Q}$ drawn from $\mathbf{P}(3, 2, 1)$ to the plane $2x - y + z + 1 = 0$. Also, find distance $\mathbf{PQ}$ and the image of the point $\mathbf{P}$ treating this plane as a mirror.
\hfill (12, 2018)
\item Find the vector equation of the plane determined by the points $\mathbf{A}\brak{3,-1,2}$, $\mathbf{B}\brak{5,2,4}$, $\mathbf{C}\brak{-1,-1,6}$. Hence, find the distance of the plane, thus obtained, from the origin.
\hfill (12, 2018)
\item Find the vector equation of the plane that contains the lines $r=\brak{\hat{i}+\hat{j}}+\lambda \brak{\hat{i}+2\hat{j} - \hat{j}}$ and the point $\brak{-1,3,-4}$. Also,find the length of the perpendicular drawn from the point $\brak{2,1,4}$ to the plane, thus obtained.
\hfill (12, 2018) 
\item Find the distance between the planes
      \begin{align*}
          \overrightarrow{r}.\myvec{2\hat{i}-3\hat{j}+6\hat{k} } - 4 =0
      \end{align*}
      and
      \begin{align*}
          \overrightarrow{r}.\myvec{6\hat{i}-9\hat{j} +18\hat{k}} +30 =0
      \end{align*}
      \hfill (12, 2016)
\item Find the position vector of the foot of perpendicular and the perpendicular distance from the point $P$ with position vector $2\hat{i}+3\hat{j}+\hat{k}$ to the plane
      \begin{align*}
          \vec{r}\cdot\brak{2\hat{i}+\hat{j}+3\hat{k}} - 26=0
      \end{align*}
      Also find image of $P$ in the plane. \hfill (12, 2016)
\item A line $l$ passes through point (-1,3,-2) and is perpendicular to both the lines $\frac {x}{1}=\frac{y}{2}=\frac{z}{3}$ and $\frac {x+2}{-3}=\frac{y-1}{2}=\frac{z+1}{5}$. Find the vector equation of the line $l$. Hence, obtain its distance from origin. \hfill (12, 2023)
\item Find the equation of the plane passing through the points $(2,1,0),(3,-2,-2)$ and $(1,1,7)$. Also, obtain its distance from the origin. \hfill (12, 2022)

\item The foot of a perpendicular drawn from the point $(-2,-1,-3)$ on a plane is $(1,-3,3)$. Find the equation of the plane. \hfill (12, 2022)
\item The distance between the planes $4x-4y+2z+5=0$ and $2x-2y+z+6=0$ is
	\begin{enumerate}
		\item $\dfrac{1}{6}$
		\item $\dfrac{7}{6}$
		\item $\dfrac{11}{6}$
		\item $\dfrac{16}{6}$
	\end{enumerate}
\hfill (12, 2022)
\item If the distance of the point $(1,1,1)$ from the plane $x-y+z+\lambda=0$ is $\dfrac{5}{\sqrt{3}}$, find the value(s) of $\lambda$. \hfill (12, 2022)

\item Find the distance of the point $(2,3,4)$ measured along the line $\dfrac{x-4}{3}=\dfrac{y+5}{6}=\dfrac{z+1}{2}$ from the plane $3x+2y+2z+5=0$. \hfill (12, 2022)

\item Find the distance of the point $P(4,3,2)$ from the plane determined by the points $A(-1,6,-5),B(-5,-2,3)$ and $C(2,4,-5)$. \hfill (12, 2022)

\item The distance of the line
	\begin{align}
		\vec{r}=(\hat{i}-\hat{j})+\lambda(\hat{i}+5\hat{j}+\hat{k})
	\end{align}
	from the plane
	\begin{align}
		\vec{r}\cdot(\hat{i}-\hat{j}+4\hat{k})=5
	\end{align}
	is
	\begin{enumerate}
		\item $\sqrt{2}$
		\item $\dfrac{1}{\sqrt{2}}$
		\item $\dfrac{1}{3\sqrt{2}}$
		\item $\dfrac{-2}{3\sqrt{2}}$
	\end{enumerate}
\hfill (12, 2022)

\item Find the values of $\lambda$, for which the distance of point $(2,1,\lambda)$ from plane $3x+5y+4z=11$ is $2\sqrt{2}$ units. \hfill (12, 2022)
\item If the distance of the point $(1,1,1)$ from the plane $x-y+z+\lambda=0$ is $\frac{5}{\sqrt{3}}$, find the value(s) of $\lambda$. \hfill (12, 2022)
\item If the line $\frac{x-1}{-3} = \frac{y-2}{2\lambda} = \frac{z-3}{2} $ and $\frac{x-1}{3\lambda} = \frac{y-1}{2}  = \frac{z-6}{-5} $ are perpendicular, find the value of $\lambda$. Hence find whether the lines are intersecting or not. \hfill (12, 2018)
\item Find the distance between the planes $2x - y + 2z = 5$ and $5x - 2.5y + 5z = 20$. \hfill (12, 2017)
\item Find the coordinates of the foot of perpendicular drawn from the point
      $A(-1, 8, 4)$ to the line joining the points $B(0, -1, 3)$ and $C(2,-3,-1)$. Hence
      find the image of the point $A$ in the line $BC$. \hfill (12, 2016)

\end{enumerate}
