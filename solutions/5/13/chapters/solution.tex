\item {Proof: } Let 
\begin{align}
\vec{O} = \myvec{0\\0}
\end{align}
%
be the circumcentre of $\triangle ABC$ and let $r$ be the radius.  Assuming that
\begin{align}
\label{eq:8.5.13_A}
\vec{A} &= r\myvec{\cos \theta_1\\\sin\theta_1},
\vec{B} = r\myvec{\cos \theta_2\\\sin\theta_2}
\\
\vec{C} &= r\myvec{\cos \theta_3\\\sin\theta_3},
\vec{D} = k\myvec{\cos \theta_4\\\sin\theta_4}
\label{eq:8.5.13_Dp}
\end{align}
in Fig. \ref{fig:8.5.13_C_circle}, from the given information
\begin{align}
 \frac{\brak{\vec{A}-\vec{B}}^T\brak{\vec{A}-\vec{C}}}{\norm{\vec{A}-\vec{B}}\norm{\vec{A}-\vec{C}}} = 
 \frac{\brak{\vec{D}-\vec{B}}^T\brak{\vec{D}-\vec{C}}}{\norm{\vec{D}-\vec{B}}\norm{\vec{D}-\vec{C}}} 
\label{eq:8.5.13_inner}
\end{align}
\begin{multline}
\because \brak{\vec{A}-\vec{B}}^T\brak{\vec{A}-\vec{C}} = \norm{\vec{A}}^2 - \vec{A}^T\vec{B}
\\
- \vec{B}^T\vec{A}+ \vec{B}^T\vec{C}
\label{eq:8.5.13_inner_expand}
\end{multline}
from \eqref{eq:8.5.13_A}-\eqref{eq:8.5.13_Dp}, \eqref{eq:8.5.13_inner_expand} can be expressed as
\begin{multline}
r^2\lsbrak{1 - \cos \brak{\theta_1-\theta_2} }
\\
\rsbrak{-  \cos \brak{\theta_1-\theta_3}+ \cos \brak{\theta_2-\theta_3}}
\\
=  r^2\brak{1-p_{12}- p_{13}+ p_{23}}
\label{eq:8.5.13_abc_inner}
\end{multline}
which can be expressed as
\begin{multline}
2r^2\lsbrak{ \sin^2 \brak{\frac{\theta_1-\theta_2}{2}} }
\\
+
\rsbrak{  \sin \brak{\frac{\theta_1-\theta_2}{2}}\sin \brak{\frac{\theta_1+\theta_2}{2} - \theta_3}}
\\
= 2r^2 \sin \brak{\frac{\theta_1-\theta_2}{2}} 
\\
\times
\sbrak{\sin \brak{\frac{\theta_1-\theta_2}{2}}
 + \sin \brak{\frac{\theta_1+\theta_2}{2} - \theta_3}}
\\
= 4r^2 \sin \brak{\frac{\theta_1-\theta_2}{2}} 
\\
\times \sin \brak{\frac{\theta_1-\theta_3}{2}} \cos \brak{\frac{\theta_2-\theta_3}{2}}
\label{eq:8.5.13_abc_innercos}
\end{multline}
%
Similarly, 
\begin{multline}
 \brak{\vec{D}-\vec{B}}^T\brak{\vec{D}-\vec{C}} = \norm{\vec{D}}^2 - \vec{D}^T\vec{B}
\\
- \vec{B}^T\vec{D}+ \vec{B}^T\vec{C}
\end{multline}
which can be expressed using \eqref{eq:8.5.13_A}-\eqref{eq:8.5.13_Dp} as
\begin{multline}
 k^2 -rk \cos\brak{\theta_2-\theta_4}
\\
-rk \cos\brak{\theta_3-\theta_4}+r^2 \cos\brak{\theta_2-\theta_3}
\\
=  k^2 -rk p_{24}-rk p_{34}+r^2  p_{23}
\label{eq:8.5.13_inner_dexpand}
\end{multline}
%
Similarly, 
\begin{align}
\norm{\vec{A}-\vec{B}}^2 &= 2r^2\sbrak{1 -  \cos \brak{\theta_1-\theta_2} } 
\\
&= 2r^2\brak{1 - p_{12}}
\label{eq:8.5.13_inner_normab}
\\
&=2 \sin^2 \brak{\frac{\theta_1-\theta_2}{2}} 
\label{eq:8.5.13_inner_normabcos}
\\
\norm{\vec{A}-\vec{C}}^2 &= 2r^2\sbrak{1 -  \cos \brak{\theta_1-\theta_3} } 
\\
&= 2r^2\brak{1 - p_{13}}
\label{eq:8.5.13_inner_normac}
\\
&=2 \sin^2 \brak{\frac{\theta_1-\theta_3}{2}} 
\label{eq:8.5.13_inner_normaccos}
\\
\norm{\vec{D}-\vec{B}}^2 &= k^2 + r^2 - 2kr \cos\brak{\theta_2-\theta_4} 
\\
&= k^2+r^2 - 2kr p_{24}
\label{eq:8.5.13_inner_normdb}
\\
\norm{\vec{D}-\vec{C}}^2 &= k^2 + r^2 - 2kr \cos\brak{\theta_3-\theta_4} 
\\
&= k^2+r^2 - 2kr p_{34}
\label{eq:8.5.13_inner_normdc}
\end{align}
%
Substituting from \eqref{eq:8.5.13_inner_dexpand}, \eqref{eq:8.5.13_abc_inner},
\eqref{eq:8.5.13_inner_normab},
\eqref{eq:8.5.13_inner_normac},
\eqref{eq:8.5.13_inner_normdb},
\eqref{eq:8.5.13_inner_normdc}
 in \eqref{eq:8.5.13_inner}, 
\begin{multline}
\frac{r^2\brak{1-p_{12}- p_{13}+ p_{23}}}{\sqrt{2r^2\brak{1 - p_{12}}}\sqrt{2r^2\brak{1 - p_{13}}}} 
\\
= \frac{k^2 -rk p_{24}-rk p_{34}+r^2  p_{23}}{\sqrt{k^2+r^2 - 2kr p_{24}}\sqrt{k^2+r^2 - 2kr p_{34}}}
\end{multline}
which can be expressed as 
\begin{multline}
\frac{1-p_{12}- p_{13}+ p_{23}}{2\sqrt{\brak{1 - p_{12}}\brak{1 - p_{13}}}}
\\
= \frac{x^2 -x p_{24}-x p_{34}+  p_{23}}{\sqrt{\brak{x^2+1 - 2x p_{24}}\brak{x^2+1 - 2x p_{34}}}}
\label{eq:8.5.13_inner_x}
\end{multline}
upon substituting
\begin{align}
x =\frac{k}{r}.
\label{eq:8.5.13_xkr}
\end{align}
From \eqref{eq:8.5.13_abc_innercos}, \eqref{eq:8.5.13_inner_normabcos}
and \eqref{eq:8.5.13_inner_normaccos},
\begin{multline}
\frac{1-p_{12}- p_{13}+ p_{23}}{2\sqrt{\brak{1 - p_{12}}\brak{1 - p_{13}}}} 
\\
= \cos \brak{\frac{\theta_2-\theta_3}{2}}
\label{eq:8.5.13_inner_123}
\end{multline}
Similarly, it can be shown that
\begin{multline}
\frac{1-p_{24}- p_{34}+ p_{23}}{2\sqrt{\brak{1 - p_{24}}\brak{1 - p_{34}}}} 
\\
= \cos \brak{\frac{\theta_2-\theta_3}{2}}
\label{eq:8.5.13_inner_234}
\end{multline}
From \eqref{eq:8.5.13_inner_123} and \eqref{eq:8.5.13_inner_234}, \eqref{eq:8.5.13_inner_x}
can be expressed as
\begin{multline}
\frac{1-p_{24}- p_{34}+ p_{23}}{2\sqrt{\brak{1 - p_{24}}\brak{1 - p_{34}}}} 
\\
= \frac{x^2 -x p_{24}-x p_{34}+  p_{23}}{\sqrt{\brak{x^2+1 - 2x p_{24}}\brak{x^2+1 - 2x p_{34}}}}
\label{eq:8.5.13_inner_x23}
\end{multline}
It is obvious that 
\begin{align}
x = 1
\end{align}
%
is a solution of \eqref{eq:8.5.13_inner_x23}
\begin{align}
\implies k = r
\end{align}
from \eqref{eq:8.5.13_xkr}.

