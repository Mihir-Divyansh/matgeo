\solution
From 
		\eqref{eq:geo-dir-vec-ab},
		\eqref{eq:geo-dir-vec-bc},
		\eqref{eq:geo-dir-vec-ca},
	\eqref{eq:median-d},
	\eqref{eq:median-e}
	and
	\eqref{eq:median-f},
\begin{align}
\vec{\frac{\vec{B}+\vec{C}}{2}} &= \frac{1}{2}\myvec{-{7} \\ 1},\,
\vec{B}-\vec{C} = \myvec{-1 \\ 11} 
\\
\vec{\frac{\vec{A}+\vec{B}}{2}}&=\frac{1}{2}\myvec{-{3} \\{5}},\,
\vec{A}-\vec{B}=\myvec{5\\ -7} \\
\vec{\frac{\vec{C}+\vec{A}}{2}} &= \myvec{-1\\-3},\,
\vec{C}-\vec{A} = \myvec{-4\\-4} \\
\end{align}
yielding
\begin{alignat}{2}
  \brak{\vec{B}-\vec{C}}^{\top}\brak{\frac{\vec{B}+\vec{C}}{2}}
	&=\myvec{-1&11}\myvec{-\frac{7}{2} \\ \frac{1}{2}}
	&&=9
  \\
\brak{\vec{A}-\vec{B}}^{\top}\brak{\frac{\vec{A}+\vec{B}}{2}}
	&=\myvec{5&-7}\myvec{-\frac{3}{2} \\\frac{5}{2}}
	&&=-25
  \\
\brak{\vec{C}-\vec{A}}^{\top}\brak{\frac{\vec{C}+\vec{A}}{2}}
	&=\myvec{-4&-4}\myvec{-1\\-3}
	&&=16
\end{alignat}
Thus, the perpendicular bisectors are obtained from 
			\eqref{eq:tri-perp-bisect}
			as
		\begin{alignat}{2}
			\label{eq:tri-perp-bisect-bc}
			BC&: \quad \myvec{-1&11}\vec{x}&&=9
\\
			\label{eq:tri-perp-bisect-ca}
			CA&: \quad \myvec{5&-7}\vec{x}&&=-25
\\
			\label{eq:tri-perp-bisect-ab}
			AB&: \quad \myvec{1&1}\vec{x}&&=-4
		\end{alignat}



